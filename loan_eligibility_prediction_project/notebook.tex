
% Default to the notebook output style

    


% Inherit from the specified cell style.




    
\documentclass[11pt]{article}

    
    
    \usepackage[T1]{fontenc}
    % Nicer default font (+ math font) than Computer Modern for most use cases
    \usepackage{mathpazo}

    % Basic figure setup, for now with no caption control since it's done
    % automatically by Pandoc (which extracts ![](path) syntax from Markdown).
    \usepackage{graphicx}
    % We will generate all images so they have a width \maxwidth. This means
    % that they will get their normal width if they fit onto the page, but
    % are scaled down if they would overflow the margins.
    \makeatletter
    \def\maxwidth{\ifdim\Gin@nat@width>\linewidth\linewidth
    \else\Gin@nat@width\fi}
    \makeatother
    \let\Oldincludegraphics\includegraphics
    % Set max figure width to be 80% of text width, for now hardcoded.
    \renewcommand{\includegraphics}[1]{\Oldincludegraphics[width=.8\maxwidth]{#1}}
    % Ensure that by default, figures have no caption (until we provide a
    % proper Figure object with a Caption API and a way to capture that
    % in the conversion process - todo).
    \usepackage{caption}
    \DeclareCaptionLabelFormat{nolabel}{}
    \captionsetup{labelformat=nolabel}

    \usepackage{adjustbox} % Used to constrain images to a maximum size 
    \usepackage{xcolor} % Allow colors to be defined
    \usepackage{enumerate} % Needed for markdown enumerations to work
    \usepackage{geometry} % Used to adjust the document margins
    \usepackage{amsmath} % Equations
    \usepackage{amssymb} % Equations
    \usepackage{textcomp} % defines textquotesingle
    % Hack from http://tex.stackexchange.com/a/47451/13684:
    \AtBeginDocument{%
        \def\PYZsq{\textquotesingle}% Upright quotes in Pygmentized code
    }
    \usepackage{upquote} % Upright quotes for verbatim code
    \usepackage{eurosym} % defines \euro
    \usepackage[mathletters]{ucs} % Extended unicode (utf-8) support
    \usepackage[utf8x]{inputenc} % Allow utf-8 characters in the tex document
    \usepackage{fancyvrb} % verbatim replacement that allows latex
    \usepackage{grffile} % extends the file name processing of package graphics 
                         % to support a larger range 
    % The hyperref package gives us a pdf with properly built
    % internal navigation ('pdf bookmarks' for the table of contents,
    % internal cross-reference links, web links for URLs, etc.)
    \usepackage{hyperref}
    \usepackage{longtable} % longtable support required by pandoc >1.10
    \usepackage{booktabs}  % table support for pandoc > 1.12.2
    \usepackage[inline]{enumitem} % IRkernel/repr support (it uses the enumerate* environment)
    \usepackage[normalem]{ulem} % ulem is needed to support strikethroughs (\sout)
                                % normalem makes italics be italics, not underlines
    

    
    
    % Colors for the hyperref package
    \definecolor{urlcolor}{rgb}{0,.145,.698}
    \definecolor{linkcolor}{rgb}{.71,0.21,0.01}
    \definecolor{citecolor}{rgb}{.12,.54,.11}

    % ANSI colors
    \definecolor{ansi-black}{HTML}{3E424D}
    \definecolor{ansi-black-intense}{HTML}{282C36}
    \definecolor{ansi-red}{HTML}{E75C58}
    \definecolor{ansi-red-intense}{HTML}{B22B31}
    \definecolor{ansi-green}{HTML}{00A250}
    \definecolor{ansi-green-intense}{HTML}{007427}
    \definecolor{ansi-yellow}{HTML}{DDB62B}
    \definecolor{ansi-yellow-intense}{HTML}{B27D12}
    \definecolor{ansi-blue}{HTML}{208FFB}
    \definecolor{ansi-blue-intense}{HTML}{0065CA}
    \definecolor{ansi-magenta}{HTML}{D160C4}
    \definecolor{ansi-magenta-intense}{HTML}{A03196}
    \definecolor{ansi-cyan}{HTML}{60C6C8}
    \definecolor{ansi-cyan-intense}{HTML}{258F8F}
    \definecolor{ansi-white}{HTML}{C5C1B4}
    \definecolor{ansi-white-intense}{HTML}{A1A6B2}

    % commands and environments needed by pandoc snippets
    % extracted from the output of `pandoc -s`
    \providecommand{\tightlist}{%
      \setlength{\itemsep}{0pt}\setlength{\parskip}{0pt}}
    \DefineVerbatimEnvironment{Highlighting}{Verbatim}{commandchars=\\\{\}}
    % Add ',fontsize=\small' for more characters per line
    \newenvironment{Shaded}{}{}
    \newcommand{\KeywordTok}[1]{\textcolor[rgb]{0.00,0.44,0.13}{\textbf{{#1}}}}
    \newcommand{\DataTypeTok}[1]{\textcolor[rgb]{0.56,0.13,0.00}{{#1}}}
    \newcommand{\DecValTok}[1]{\textcolor[rgb]{0.25,0.63,0.44}{{#1}}}
    \newcommand{\BaseNTok}[1]{\textcolor[rgb]{0.25,0.63,0.44}{{#1}}}
    \newcommand{\FloatTok}[1]{\textcolor[rgb]{0.25,0.63,0.44}{{#1}}}
    \newcommand{\CharTok}[1]{\textcolor[rgb]{0.25,0.44,0.63}{{#1}}}
    \newcommand{\StringTok}[1]{\textcolor[rgb]{0.25,0.44,0.63}{{#1}}}
    \newcommand{\CommentTok}[1]{\textcolor[rgb]{0.38,0.63,0.69}{\textit{{#1}}}}
    \newcommand{\OtherTok}[1]{\textcolor[rgb]{0.00,0.44,0.13}{{#1}}}
    \newcommand{\AlertTok}[1]{\textcolor[rgb]{1.00,0.00,0.00}{\textbf{{#1}}}}
    \newcommand{\FunctionTok}[1]{\textcolor[rgb]{0.02,0.16,0.49}{{#1}}}
    \newcommand{\RegionMarkerTok}[1]{{#1}}
    \newcommand{\ErrorTok}[1]{\textcolor[rgb]{1.00,0.00,0.00}{\textbf{{#1}}}}
    \newcommand{\NormalTok}[1]{{#1}}
    
    % Additional commands for more recent versions of Pandoc
    \newcommand{\ConstantTok}[1]{\textcolor[rgb]{0.53,0.00,0.00}{{#1}}}
    \newcommand{\SpecialCharTok}[1]{\textcolor[rgb]{0.25,0.44,0.63}{{#1}}}
    \newcommand{\VerbatimStringTok}[1]{\textcolor[rgb]{0.25,0.44,0.63}{{#1}}}
    \newcommand{\SpecialStringTok}[1]{\textcolor[rgb]{0.73,0.40,0.53}{{#1}}}
    \newcommand{\ImportTok}[1]{{#1}}
    \newcommand{\DocumentationTok}[1]{\textcolor[rgb]{0.73,0.13,0.13}{\textit{{#1}}}}
    \newcommand{\AnnotationTok}[1]{\textcolor[rgb]{0.38,0.63,0.69}{\textbf{\textit{{#1}}}}}
    \newcommand{\CommentVarTok}[1]{\textcolor[rgb]{0.38,0.63,0.69}{\textbf{\textit{{#1}}}}}
    \newcommand{\VariableTok}[1]{\textcolor[rgb]{0.10,0.09,0.49}{{#1}}}
    \newcommand{\ControlFlowTok}[1]{\textcolor[rgb]{0.00,0.44,0.13}{\textbf{{#1}}}}
    \newcommand{\OperatorTok}[1]{\textcolor[rgb]{0.40,0.40,0.40}{{#1}}}
    \newcommand{\BuiltInTok}[1]{{#1}}
    \newcommand{\ExtensionTok}[1]{{#1}}
    \newcommand{\PreprocessorTok}[1]{\textcolor[rgb]{0.74,0.48,0.00}{{#1}}}
    \newcommand{\AttributeTok}[1]{\textcolor[rgb]{0.49,0.56,0.16}{{#1}}}
    \newcommand{\InformationTok}[1]{\textcolor[rgb]{0.38,0.63,0.69}{\textbf{\textit{{#1}}}}}
    \newcommand{\WarningTok}[1]{\textcolor[rgb]{0.38,0.63,0.69}{\textbf{\textit{{#1}}}}}
    
    
    % Define a nice break command that doesn't care if a line doesn't already
    % exist.
    \def\br{\hspace*{\fill} \\* }
    % Math Jax compatability definitions
    \def\gt{>}
    \def\lt{<}
    % Document parameters
    \title{loan\_prediction1}
    
    
    

    % Pygments definitions
    
\makeatletter
\def\PY@reset{\let\PY@it=\relax \let\PY@bf=\relax%
    \let\PY@ul=\relax \let\PY@tc=\relax%
    \let\PY@bc=\relax \let\PY@ff=\relax}
\def\PY@tok#1{\csname PY@tok@#1\endcsname}
\def\PY@toks#1+{\ifx\relax#1\empty\else%
    \PY@tok{#1}\expandafter\PY@toks\fi}
\def\PY@do#1{\PY@bc{\PY@tc{\PY@ul{%
    \PY@it{\PY@bf{\PY@ff{#1}}}}}}}
\def\PY#1#2{\PY@reset\PY@toks#1+\relax+\PY@do{#2}}

\expandafter\def\csname PY@tok@w\endcsname{\def\PY@tc##1{\textcolor[rgb]{0.73,0.73,0.73}{##1}}}
\expandafter\def\csname PY@tok@c\endcsname{\let\PY@it=\textit\def\PY@tc##1{\textcolor[rgb]{0.25,0.50,0.50}{##1}}}
\expandafter\def\csname PY@tok@cp\endcsname{\def\PY@tc##1{\textcolor[rgb]{0.74,0.48,0.00}{##1}}}
\expandafter\def\csname PY@tok@k\endcsname{\let\PY@bf=\textbf\def\PY@tc##1{\textcolor[rgb]{0.00,0.50,0.00}{##1}}}
\expandafter\def\csname PY@tok@kp\endcsname{\def\PY@tc##1{\textcolor[rgb]{0.00,0.50,0.00}{##1}}}
\expandafter\def\csname PY@tok@kt\endcsname{\def\PY@tc##1{\textcolor[rgb]{0.69,0.00,0.25}{##1}}}
\expandafter\def\csname PY@tok@o\endcsname{\def\PY@tc##1{\textcolor[rgb]{0.40,0.40,0.40}{##1}}}
\expandafter\def\csname PY@tok@ow\endcsname{\let\PY@bf=\textbf\def\PY@tc##1{\textcolor[rgb]{0.67,0.13,1.00}{##1}}}
\expandafter\def\csname PY@tok@nb\endcsname{\def\PY@tc##1{\textcolor[rgb]{0.00,0.50,0.00}{##1}}}
\expandafter\def\csname PY@tok@nf\endcsname{\def\PY@tc##1{\textcolor[rgb]{0.00,0.00,1.00}{##1}}}
\expandafter\def\csname PY@tok@nc\endcsname{\let\PY@bf=\textbf\def\PY@tc##1{\textcolor[rgb]{0.00,0.00,1.00}{##1}}}
\expandafter\def\csname PY@tok@nn\endcsname{\let\PY@bf=\textbf\def\PY@tc##1{\textcolor[rgb]{0.00,0.00,1.00}{##1}}}
\expandafter\def\csname PY@tok@ne\endcsname{\let\PY@bf=\textbf\def\PY@tc##1{\textcolor[rgb]{0.82,0.25,0.23}{##1}}}
\expandafter\def\csname PY@tok@nv\endcsname{\def\PY@tc##1{\textcolor[rgb]{0.10,0.09,0.49}{##1}}}
\expandafter\def\csname PY@tok@no\endcsname{\def\PY@tc##1{\textcolor[rgb]{0.53,0.00,0.00}{##1}}}
\expandafter\def\csname PY@tok@nl\endcsname{\def\PY@tc##1{\textcolor[rgb]{0.63,0.63,0.00}{##1}}}
\expandafter\def\csname PY@tok@ni\endcsname{\let\PY@bf=\textbf\def\PY@tc##1{\textcolor[rgb]{0.60,0.60,0.60}{##1}}}
\expandafter\def\csname PY@tok@na\endcsname{\def\PY@tc##1{\textcolor[rgb]{0.49,0.56,0.16}{##1}}}
\expandafter\def\csname PY@tok@nt\endcsname{\let\PY@bf=\textbf\def\PY@tc##1{\textcolor[rgb]{0.00,0.50,0.00}{##1}}}
\expandafter\def\csname PY@tok@nd\endcsname{\def\PY@tc##1{\textcolor[rgb]{0.67,0.13,1.00}{##1}}}
\expandafter\def\csname PY@tok@s\endcsname{\def\PY@tc##1{\textcolor[rgb]{0.73,0.13,0.13}{##1}}}
\expandafter\def\csname PY@tok@sd\endcsname{\let\PY@it=\textit\def\PY@tc##1{\textcolor[rgb]{0.73,0.13,0.13}{##1}}}
\expandafter\def\csname PY@tok@si\endcsname{\let\PY@bf=\textbf\def\PY@tc##1{\textcolor[rgb]{0.73,0.40,0.53}{##1}}}
\expandafter\def\csname PY@tok@se\endcsname{\let\PY@bf=\textbf\def\PY@tc##1{\textcolor[rgb]{0.73,0.40,0.13}{##1}}}
\expandafter\def\csname PY@tok@sr\endcsname{\def\PY@tc##1{\textcolor[rgb]{0.73,0.40,0.53}{##1}}}
\expandafter\def\csname PY@tok@ss\endcsname{\def\PY@tc##1{\textcolor[rgb]{0.10,0.09,0.49}{##1}}}
\expandafter\def\csname PY@tok@sx\endcsname{\def\PY@tc##1{\textcolor[rgb]{0.00,0.50,0.00}{##1}}}
\expandafter\def\csname PY@tok@m\endcsname{\def\PY@tc##1{\textcolor[rgb]{0.40,0.40,0.40}{##1}}}
\expandafter\def\csname PY@tok@gh\endcsname{\let\PY@bf=\textbf\def\PY@tc##1{\textcolor[rgb]{0.00,0.00,0.50}{##1}}}
\expandafter\def\csname PY@tok@gu\endcsname{\let\PY@bf=\textbf\def\PY@tc##1{\textcolor[rgb]{0.50,0.00,0.50}{##1}}}
\expandafter\def\csname PY@tok@gd\endcsname{\def\PY@tc##1{\textcolor[rgb]{0.63,0.00,0.00}{##1}}}
\expandafter\def\csname PY@tok@gi\endcsname{\def\PY@tc##1{\textcolor[rgb]{0.00,0.63,0.00}{##1}}}
\expandafter\def\csname PY@tok@gr\endcsname{\def\PY@tc##1{\textcolor[rgb]{1.00,0.00,0.00}{##1}}}
\expandafter\def\csname PY@tok@ge\endcsname{\let\PY@it=\textit}
\expandafter\def\csname PY@tok@gs\endcsname{\let\PY@bf=\textbf}
\expandafter\def\csname PY@tok@gp\endcsname{\let\PY@bf=\textbf\def\PY@tc##1{\textcolor[rgb]{0.00,0.00,0.50}{##1}}}
\expandafter\def\csname PY@tok@go\endcsname{\def\PY@tc##1{\textcolor[rgb]{0.53,0.53,0.53}{##1}}}
\expandafter\def\csname PY@tok@gt\endcsname{\def\PY@tc##1{\textcolor[rgb]{0.00,0.27,0.87}{##1}}}
\expandafter\def\csname PY@tok@err\endcsname{\def\PY@bc##1{\setlength{\fboxsep}{0pt}\fcolorbox[rgb]{1.00,0.00,0.00}{1,1,1}{\strut ##1}}}
\expandafter\def\csname PY@tok@kc\endcsname{\let\PY@bf=\textbf\def\PY@tc##1{\textcolor[rgb]{0.00,0.50,0.00}{##1}}}
\expandafter\def\csname PY@tok@kd\endcsname{\let\PY@bf=\textbf\def\PY@tc##1{\textcolor[rgb]{0.00,0.50,0.00}{##1}}}
\expandafter\def\csname PY@tok@kn\endcsname{\let\PY@bf=\textbf\def\PY@tc##1{\textcolor[rgb]{0.00,0.50,0.00}{##1}}}
\expandafter\def\csname PY@tok@kr\endcsname{\let\PY@bf=\textbf\def\PY@tc##1{\textcolor[rgb]{0.00,0.50,0.00}{##1}}}
\expandafter\def\csname PY@tok@bp\endcsname{\def\PY@tc##1{\textcolor[rgb]{0.00,0.50,0.00}{##1}}}
\expandafter\def\csname PY@tok@fm\endcsname{\def\PY@tc##1{\textcolor[rgb]{0.00,0.00,1.00}{##1}}}
\expandafter\def\csname PY@tok@vc\endcsname{\def\PY@tc##1{\textcolor[rgb]{0.10,0.09,0.49}{##1}}}
\expandafter\def\csname PY@tok@vg\endcsname{\def\PY@tc##1{\textcolor[rgb]{0.10,0.09,0.49}{##1}}}
\expandafter\def\csname PY@tok@vi\endcsname{\def\PY@tc##1{\textcolor[rgb]{0.10,0.09,0.49}{##1}}}
\expandafter\def\csname PY@tok@vm\endcsname{\def\PY@tc##1{\textcolor[rgb]{0.10,0.09,0.49}{##1}}}
\expandafter\def\csname PY@tok@sa\endcsname{\def\PY@tc##1{\textcolor[rgb]{0.73,0.13,0.13}{##1}}}
\expandafter\def\csname PY@tok@sb\endcsname{\def\PY@tc##1{\textcolor[rgb]{0.73,0.13,0.13}{##1}}}
\expandafter\def\csname PY@tok@sc\endcsname{\def\PY@tc##1{\textcolor[rgb]{0.73,0.13,0.13}{##1}}}
\expandafter\def\csname PY@tok@dl\endcsname{\def\PY@tc##1{\textcolor[rgb]{0.73,0.13,0.13}{##1}}}
\expandafter\def\csname PY@tok@s2\endcsname{\def\PY@tc##1{\textcolor[rgb]{0.73,0.13,0.13}{##1}}}
\expandafter\def\csname PY@tok@sh\endcsname{\def\PY@tc##1{\textcolor[rgb]{0.73,0.13,0.13}{##1}}}
\expandafter\def\csname PY@tok@s1\endcsname{\def\PY@tc##1{\textcolor[rgb]{0.73,0.13,0.13}{##1}}}
\expandafter\def\csname PY@tok@mb\endcsname{\def\PY@tc##1{\textcolor[rgb]{0.40,0.40,0.40}{##1}}}
\expandafter\def\csname PY@tok@mf\endcsname{\def\PY@tc##1{\textcolor[rgb]{0.40,0.40,0.40}{##1}}}
\expandafter\def\csname PY@tok@mh\endcsname{\def\PY@tc##1{\textcolor[rgb]{0.40,0.40,0.40}{##1}}}
\expandafter\def\csname PY@tok@mi\endcsname{\def\PY@tc##1{\textcolor[rgb]{0.40,0.40,0.40}{##1}}}
\expandafter\def\csname PY@tok@il\endcsname{\def\PY@tc##1{\textcolor[rgb]{0.40,0.40,0.40}{##1}}}
\expandafter\def\csname PY@tok@mo\endcsname{\def\PY@tc##1{\textcolor[rgb]{0.40,0.40,0.40}{##1}}}
\expandafter\def\csname PY@tok@ch\endcsname{\let\PY@it=\textit\def\PY@tc##1{\textcolor[rgb]{0.25,0.50,0.50}{##1}}}
\expandafter\def\csname PY@tok@cm\endcsname{\let\PY@it=\textit\def\PY@tc##1{\textcolor[rgb]{0.25,0.50,0.50}{##1}}}
\expandafter\def\csname PY@tok@cpf\endcsname{\let\PY@it=\textit\def\PY@tc##1{\textcolor[rgb]{0.25,0.50,0.50}{##1}}}
\expandafter\def\csname PY@tok@c1\endcsname{\let\PY@it=\textit\def\PY@tc##1{\textcolor[rgb]{0.25,0.50,0.50}{##1}}}
\expandafter\def\csname PY@tok@cs\endcsname{\let\PY@it=\textit\def\PY@tc##1{\textcolor[rgb]{0.25,0.50,0.50}{##1}}}

\def\PYZbs{\char`\\}
\def\PYZus{\char`\_}
\def\PYZob{\char`\{}
\def\PYZcb{\char`\}}
\def\PYZca{\char`\^}
\def\PYZam{\char`\&}
\def\PYZlt{\char`\<}
\def\PYZgt{\char`\>}
\def\PYZsh{\char`\#}
\def\PYZpc{\char`\%}
\def\PYZdl{\char`\$}
\def\PYZhy{\char`\-}
\def\PYZsq{\char`\'}
\def\PYZdq{\char`\"}
\def\PYZti{\char`\~}
% for compatibility with earlier versions
\def\PYZat{@}
\def\PYZlb{[}
\def\PYZrb{]}
\makeatother


    % Exact colors from NB
    \definecolor{incolor}{rgb}{0.0, 0.0, 0.5}
    \definecolor{outcolor}{rgb}{0.545, 0.0, 0.0}



    
    % Prevent overflowing lines due to hard-to-break entities
    \sloppy 
    % Setup hyperref package
    \hypersetup{
      breaklinks=true,  % so long urls are correctly broken across lines
      colorlinks=true,
      urlcolor=urlcolor,
      linkcolor=linkcolor,
      citecolor=citecolor,
      }
    % Slightly bigger margins than the latex defaults
    
    \geometry{verbose,tmargin=1in,bmargin=1in,lmargin=1in,rmargin=1in}
    
    

    \begin{document}
    
    
    \maketitle
    
    

    
    \subsubsection{Project Objective}\label{project-objective}

To automate the loan eligibility process (real time) based on customer
detail provided while filling online application form. These details are
Gender, Marital Status, Education, Number of Dependents, Income, Loan
Amount, Credit History and others. To automate this process, they have
given a problem to identify the customers segments, those are eligible
for loan amount so that they can specifically target these customers.

\textbf{The aim of this project is to predict whether a loan will be
approved or not}

    \subsubsection{Background Information}\label{background-information}

Loan prediction is a very common real-life problem that each retail bank
faces atleast once in its lifetime. If done correctly, it can save a lot
of man hours at the end of a retail bank.

    It is a classification problem where we have to predict whether a loan
would be approved or not. In a classification problem, we have to
predict discrete values based on a given set of independent variable(s).

    \subsubsection{Hypthosis generation}\label{hypthosis-generation}

Below are some of the factors which I think can affect the Loan Approval
(dependent variable for this loan prediction problem):

\begin{quote}
\textbf{Salary}: Applicants with high income should have more chances of
loan approval.
\end{quote}

\begin{quote}
\textbf{Previous history}: Applicants who have repayed their previous
debts should have higher chances of loan approval.
\end{quote}

\begin{quote}
\textbf{Loan amount}: Loan approval should also depend on the loan
amount. If the loan amount is less, chances of loan approval should be
high.
\end{quote}

\begin{quote}
\textbf{Loan term}: Loan for less time period and less amount should
have higher chances of approval.
\end{quote}

\begin{quote}
\textbf{Equated Monthly Installment (EMI)}: Lesser the amount to be paid
monthly to repay the loan, higher the chances of loan approval.
\end{quote}

    \subsubsection{The sample Dataset}\label{the-sample-dataset}

\begin{itemize}
\item
  \textbf{Train file} will be used for training the model, i.e. our
  model will learn from this file. It contains all the independent
  variables and the target variable.
\item
  \textbf{Test file} contains all the independent variables, but not the
  target variable. I will apply the model to predict the target variable
  for the test data.
\end{itemize}

    \subsubsection{Data Dictionary}\label{data-dictionary}

\begin{figure}
\centering
\includegraphics{attachment:image.png}
\caption{image.png}
\end{figure}

    \begin{Verbatim}[commandchars=\\\{\}]
{\color{incolor}In [{\color{incolor}147}]:} \PY{c+c1}{\PYZsh{} Importing required Packages}
          \PY{k+kn}{import} \PY{n+nn}{pandas} \PY{k}{as} \PY{n+nn}{pd}
          \PY{k+kn}{import} \PY{n+nn}{numpy} \PY{k}{as} \PY{n+nn}{np}
          \PY{k+kn}{import} \PY{n+nn}{seaborn} \PY{k}{as} \PY{n+nn}{sns}
          \PY{k+kn}{import} \PY{n+nn}{matplotlib}\PY{n+nn}{.}\PY{n+nn}{pyplot} \PY{k}{as} \PY{n+nn}{plt}
          \PY{o}{\PYZpc{}}\PY{k}{matplotlib} inline
\end{Verbatim}


    \begin{Verbatim}[commandchars=\\\{\}]
{\color{incolor}In [{\color{incolor}148}]:} \PY{k+kn}{import} \PY{n+nn}{warnings}
          \PY{n}{warnings}\PY{o}{.}\PY{n}{filterwarnings}\PY{p}{(}\PY{l+s+s2}{\PYZdq{}}\PY{l+s+s2}{ignore}\PY{l+s+s2}{\PYZdq{}}\PY{p}{)}
\end{Verbatim}


    \begin{Verbatim}[commandchars=\\\{\}]
{\color{incolor}In [{\color{incolor}149}]:} \PY{c+c1}{\PYZsh{} Read Train}
          \PY{n}{train}\PY{o}{=}\PY{n}{pd}\PY{o}{.}\PY{n}{read\PYZus{}csv}\PY{p}{(}\PY{l+s+s2}{\PYZdq{}}\PY{l+s+s2}{train\PYZus{}u6lujuX\PYZus{}CVtuZ9i.csv}\PY{l+s+s2}{\PYZdq{}}\PY{p}{)}
\end{Verbatim}


    \begin{Verbatim}[commandchars=\\\{\}]
{\color{incolor}In [{\color{incolor}150}]:} \PY{n+nb}{print}\PY{p}{(}\PY{n}{f}\PY{l+s+s1}{\PYZsq{}}\PY{l+s+s1}{The train dataset has }\PY{l+s+si}{\PYZob{}train.shape[0]\PYZcb{}}\PY{l+s+s1}{ observations and }\PY{l+s+si}{\PYZob{}train.shape[1]\PYZcb{}}\PY{l+s+s1}{ features}\PY{l+s+s1}{\PYZsq{}}\PY{p}{)}
\end{Verbatim}


    \begin{Verbatim}[commandchars=\\\{\}]
The train dataset has 614 observations and 13 features

    \end{Verbatim}

    \begin{Verbatim}[commandchars=\\\{\}]
{\color{incolor}In [{\color{incolor}151}]:} \PY{c+c1}{\PYZsh{} Read Test}
          \PY{n}{test}\PY{o}{=}\PY{n}{pd}\PY{o}{.}\PY{n}{read\PYZus{}csv}\PY{p}{(}\PY{l+s+s2}{\PYZdq{}}\PY{l+s+s2}{test\PYZus{}Y3wMUE5\PYZus{}7gLdaTN.csv}\PY{l+s+s2}{\PYZdq{}}\PY{p}{)}
\end{Verbatim}


    \begin{Verbatim}[commandchars=\\\{\}]
{\color{incolor}In [{\color{incolor}152}]:} \PY{n+nb}{print}\PY{p}{(}\PY{n}{f}\PY{l+s+s1}{\PYZsq{}}\PY{l+s+s1}{The train dataset has }\PY{l+s+si}{\PYZob{}test.shape[0]\PYZcb{}}\PY{l+s+s1}{ observations and }\PY{l+s+si}{\PYZob{}test.shape[1]\PYZcb{}}\PY{l+s+s1}{ features}\PY{l+s+s1}{\PYZsq{}}\PY{p}{)}
          \PY{n}{f}\PY{l+s+s1}{\PYZsq{}}\PY{l+s+s1}{The target feature is excluded}\PY{l+s+s1}{\PYZsq{}}
\end{Verbatim}


    \begin{Verbatim}[commandchars=\\\{\}]
The train dataset has 367 observations and 12 features

    \end{Verbatim}

\begin{Verbatim}[commandchars=\\\{\}]
{\color{outcolor}Out[{\color{outcolor}152}]:} 'The target feature is excluded'
\end{Verbatim}
            
    \begin{Verbatim}[commandchars=\\\{\}]
{\color{incolor}In [{\color{incolor}153}]:} \PY{c+c1}{\PYZsh{} first five observation for train}
          \PY{n}{train}\PY{o}{.}\PY{n}{head}\PY{p}{(}\PY{p}{)}
\end{Verbatim}


\begin{Verbatim}[commandchars=\\\{\}]
{\color{outcolor}Out[{\color{outcolor}153}]:}     Loan\_ID Gender Married Dependents     Education Self\_Employed  \textbackslash{}
          0  LP001002   Male      No          0      Graduate            No   
          1  LP001003   Male     Yes          1      Graduate            No   
          2  LP001005   Male     Yes          0      Graduate           Yes   
          3  LP001006   Male     Yes          0  Not Graduate            No   
          4  LP001008   Male      No          0      Graduate            No   
          
             ApplicantIncome  CoapplicantIncome  LoanAmount  Loan\_Amount\_Term  \textbackslash{}
          0             5849                0.0         NaN             360.0   
          1             4583             1508.0       128.0             360.0   
          2             3000                0.0        66.0             360.0   
          3             2583             2358.0       120.0             360.0   
          4             6000                0.0       141.0             360.0   
          
             Credit\_History Property\_Area Loan\_Status  
          0             1.0         Urban           Y  
          1             1.0         Rural           N  
          2             1.0         Urban           Y  
          3             1.0         Urban           Y  
          4             1.0         Urban           Y  
\end{Verbatim}
            
    \begin{Verbatim}[commandchars=\\\{\}]
{\color{incolor}In [{\color{incolor}154}]:} \PY{c+c1}{\PYZsh{} first five observation for train}
          \PY{n}{test}\PY{o}{.}\PY{n}{head}\PY{p}{(}\PY{p}{)}
\end{Verbatim}


\begin{Verbatim}[commandchars=\\\{\}]
{\color{outcolor}Out[{\color{outcolor}154}]:}     Loan\_ID Gender Married Dependents     Education Self\_Employed  \textbackslash{}
          0  LP001015   Male     Yes          0      Graduate            No   
          1  LP001022   Male     Yes          1      Graduate            No   
          2  LP001031   Male     Yes          2      Graduate            No   
          3  LP001035   Male     Yes          2      Graduate            No   
          4  LP001051   Male      No          0  Not Graduate            No   
          
             ApplicantIncome  CoapplicantIncome  LoanAmount  Loan\_Amount\_Term  \textbackslash{}
          0             5720                  0       110.0             360.0   
          1             3076               1500       126.0             360.0   
          2             5000               1800       208.0             360.0   
          3             2340               2546       100.0             360.0   
          4             3276                  0        78.0             360.0   
          
             Credit\_History Property\_Area  
          0             1.0         Urban  
          1             1.0         Urban  
          2             1.0         Urban  
          3             NaN         Urban  
          4             1.0         Urban  
\end{Verbatim}
            
    \texttt{Fractional\ sample\ of\ the\ dataset\ at\ the\ different\ point.\ This\ select\ a\ random\ sample\ every\ time\ the\ code\ is\ run.\ Its\ essence\ is\ to\ obtain\ a\ more\ specific\ overview\ of\ each\ features.}

    \begin{Verbatim}[commandchars=\\\{\}]
{\color{incolor}In [{\color{incolor}155}]:}  \PY{c+c1}{\PYZsh{} obtain fractional sample of the dataset at the diff. point}
          \PY{n}{train}\PY{o}{.}\PY{n}{sample}\PY{p}{(}\PY{n}{frac}\PY{o}{=}\PY{l+m+mf}{0.1}\PY{p}{,}\PY{n}{axis}\PY{o}{=}\PY{l+m+mi}{0}\PY{p}{)}
\end{Verbatim}


\begin{Verbatim}[commandchars=\\\{\}]
{\color{outcolor}Out[{\color{outcolor}155}]:}       Loan\_ID  Gender Married Dependents     Education Self\_Employed  \textbackslash{}
          252  LP001841    Male      No          0  Not Graduate           Yes   
          282  LP001915    Male     Yes          2      Graduate            No   
          175  LP001606    Male     Yes          0      Graduate            No   
          80   LP001265  Female      No          0      Graduate            No   
          590  LP002928    Male     Yes          0      Graduate            No   
          546  LP002768    Male      No          0  Not Graduate            No   
          480  LP002534  Female      No          0  Not Graduate            No   
          96   LP001327  Female     Yes          0      Graduate            No   
          378  LP002224    Male      No          0      Graduate            No   
          589  LP002926    Male     Yes          2      Graduate           Yes   
          476  LP002529    Male     Yes          2      Graduate            No   
          334  LP002103     NaN     Yes          1      Graduate           Yes   
          558  LP002798    Male     Yes          0      Graduate            No   
          464  LP002493    Male      No          0      Graduate            No   
          77   LP001259    Male     Yes          1      Graduate           Yes   
          598  LP002945    Male     Yes          0      Graduate           Yes   
          310  LP002002  Female      No          0      Graduate            No   
          58   LP001198    Male     Yes          1      Graduate            No   
          561  LP002813  Female     Yes          1      Graduate           Yes   
          434  LP002390    Male      No          0      Graduate            No   
          253  LP001843    Male     Yes          1  Not Graduate            No   
          399  LP002287  Female      No          0      Graduate            No   
          404  LP002301  Female      No          0      Graduate           Yes   
          401  LP002296    Male      No          0  Not Graduate            No   
          498  LP002600    Male     Yes          1      Graduate           Yes   
          428  LP002369    Male     Yes          0      Graduate            No   
          524  LP002697    Male      No          0      Graduate            No   
          432  LP002386    Male      No          0      Graduate           NaN   
          199  LP001673    Male      No          0      Graduate           Yes   
          0    LP001002    Male      No          0      Graduate            No   
          ..        {\ldots}     {\ldots}     {\ldots}        {\ldots}           {\ldots}           {\ldots}   
          544  LP002757  Female     Yes          0  Not Graduate            No   
          483  LP002541    Male     Yes          0      Graduate            No   
          136  LP001489  Female     Yes          0      Graduate            No   
          145  LP001514  Female     Yes          0      Graduate            No   
          112  LP001391    Male     Yes          0  Not Graduate            No   
          335  LP002106    Male     Yes        NaN      Graduate           Yes   
          176  LP001608    Male     Yes          2      Graduate            No   
          220  LP001736    Male     Yes          0      Graduate            No   
          144  LP001508    Male     Yes          2      Graduate            No   
          529  LP002716    Male      No          0  Not Graduate            No   
          462  LP002487    Male     Yes          0      Graduate            No   
          370  LP002194  Female      No          0      Graduate           Yes   
          207  LP001698    Male      No          0  Not Graduate            No   
          250  LP001835    Male     Yes          0  Not Graduate            No   
          564  LP002832    Male     Yes          2      Graduate            No   
          140  LP001497    Male     Yes          2      Graduate            No   
          509  LP002634  Female      No          1      Graduate            No   
          347  LP002131    Male     Yes          2  Not Graduate            No   
          122  LP001431  Female      No          0      Graduate            No   
          27   LP001073    Male     Yes          2  Not Graduate            No   
          543  LP002755    Male     Yes          1  Not Graduate            No   
          235  LP001784    Male     Yes          1      Graduate            No   
          111  LP001387  Female     Yes          0      Graduate           NaN   
          583  LP002898    Male     Yes          1      Graduate            No   
          506  LP002624    Male     Yes          0      Graduate            No   
          418  LP002345    Male     Yes          0      Graduate            No   
          605  LP002960    Male     Yes          0  Not Graduate            No   
          256  LP001849    Male      No          0  Not Graduate            No   
          602  LP002953    Male     Yes         3+      Graduate            No   
          295  LP001949    Male     Yes         3+      Graduate           NaN   
          
               ApplicantIncome  CoapplicantIncome  LoanAmount  Loan\_Amount\_Term  \textbackslash{}
          252             2583        2167.000000       104.0             360.0   
          282             2301         985.799988        78.0             180.0   
          175             3497        1964.000000       116.0             360.0   
          80              3846           0.000000       111.0             360.0   
          590             3000        3416.000000        56.0             180.0   
          546             3358           0.000000        80.0              36.0   
          480             4350           0.000000       154.0             360.0   
          96              2484        2302.000000       137.0             360.0   
          378             3069           0.000000        71.0             480.0   
          589             2726           0.000000       106.0             360.0   
          476             6700        1750.000000       230.0             300.0   
          334             9833        1833.000000       182.0             180.0   
          558             3887        2669.000000       162.0             360.0   
          464             4166           0.000000        98.0             360.0   
          77              1000        3022.000000       110.0             360.0   
          598             9963           0.000000       180.0             360.0   
          310             2917           0.000000        84.0             360.0   
          58              8080        2250.000000       180.0             360.0   
          561            19484           0.000000       600.0             360.0   
          434             3750           0.000000       100.0             360.0   
          253             2661        7101.000000       279.0             180.0   
          399             1500        1800.000000       103.0             360.0   
          404             7441           0.000000       194.0             360.0   
          401             2755           0.000000        65.0             300.0   
          498             2895           0.000000        95.0             360.0   
          428             2920          16.120001        87.0             360.0   
          524             4680        2087.000000         NaN             360.0   
          432            12876           0.000000       405.0             360.0   
          199            11000           0.000000        83.0             360.0   
          0               5849           0.000000         NaN             360.0   
          ..               {\ldots}                {\ldots}         {\ldots}               {\ldots}   
          544             3017         663.000000       102.0             360.0   
          483            10833           0.000000       234.0             360.0   
          136             4583           0.000000        84.0             360.0   
          145             2330        4486.000000       100.0             360.0   
          112             3572        4114.000000       152.0               NaN   
          335             5503        4490.000000        70.0               NaN   
          176             2045        1619.000000       101.0             360.0   
          220             2221           0.000000        60.0             360.0   
          144            11757           0.000000       187.0             180.0   
          529             6783           0.000000       130.0             360.0   
          462             3015        2188.000000       153.0             360.0   
          370            15759           0.000000        55.0             360.0   
          207             3975        2531.000000        55.0             360.0   
          250             1668        3890.000000       201.0             360.0   
          564             8799           0.000000       258.0             360.0   
          140             5042        2083.000000       185.0             360.0   
          509            13262           0.000000        40.0             360.0   
          347             3083        2168.000000       126.0             360.0   
          122             2137        8980.000000       137.0             360.0   
          27              4226        1040.000000       110.0             360.0   
          543             2239        2524.000000       128.0             360.0   
          235             5500        1260.000000       170.0             360.0   
          111             2929        2333.000000       139.0             360.0   
          583             1880           0.000000        61.0             360.0   
          506            20833        6667.000000       480.0             360.0   
          418             1025        2773.000000       112.0             360.0   
          605             2400        3800.000000         NaN             180.0   
          256             6045           0.000000       115.0             360.0   
          602             5703           0.000000       128.0             360.0   
          295             4416        1250.000000       110.0             360.0   
          
               Credit\_History Property\_Area Loan\_Status  
          252             1.0         Rural           Y  
          282             1.0         Urban           Y  
          175             1.0         Rural           Y  
          80              1.0     Semiurban           Y  
          590             1.0     Semiurban           Y  
          546             1.0     Semiurban           N  
          480             1.0         Rural           Y  
          96              1.0     Semiurban           Y  
          378             1.0         Urban           N  
          589             0.0     Semiurban           N  
          476             1.0     Semiurban           Y  
          334             1.0         Urban           Y  
          558             1.0     Semiurban           Y  
          464             0.0     Semiurban           N  
          77              1.0         Urban           N  
          598             1.0         Rural           Y  
          310             1.0     Semiurban           Y  
          58              1.0         Urban           Y  
          561             1.0     Semiurban           Y  
          434             1.0         Urban           Y  
          253             1.0     Semiurban           Y  
          399             0.0     Semiurban           N  
          404             1.0         Rural           N  
          401             1.0         Rural           N  
          498             1.0     Semiurban           Y  
          428             1.0         Rural           Y  
          524             1.0     Semiurban           N  
          432             1.0     Semiurban           Y  
          199             1.0         Urban           N  
          0               1.0         Urban           Y  
          ..              {\ldots}           {\ldots}         {\ldots}  
          544             NaN     Semiurban           Y  
          483             1.0     Semiurban           Y  
          136             1.0         Rural           N  
          145             1.0     Semiurban           Y  
          112             0.0         Rural           N  
          335             1.0     Semiurban           Y  
          176             1.0         Rural           Y  
          220             0.0         Urban           N  
          144             1.0         Urban           Y  
          529             1.0     Semiurban           Y  
          462             1.0         Rural           Y  
          370             1.0     Semiurban           Y  
          207             1.0         Rural           Y  
          250             0.0     Semiurban           N  
          564             0.0         Urban           N  
          140             1.0         Rural           N  
          509             1.0         Urban           Y  
          347             1.0         Urban           Y  
          122             0.0     Semiurban           Y  
          27              1.0         Urban           Y  
          543             1.0         Urban           Y  
          235             1.0         Rural           Y  
          111             1.0     Semiurban           Y  
          583             NaN         Rural           N  
          506             NaN         Urban           Y  
          418             1.0         Rural           Y  
          605             1.0         Urban           N  
          256             0.0         Rural           N  
          602             1.0         Urban           Y  
          295             1.0         Urban           Y  
          
          [61 rows x 13 columns]
\end{Verbatim}
            
    \emph{A series of data showing the value counts of
\texttt{Coapplicant\ Income}}

    \begin{Verbatim}[commandchars=\\\{\}]
{\color{incolor}In [{\color{incolor}156}]:} \PY{c+c1}{\PYZsh{} obtain the value of the categorical type that has the highest frequency}
          \PY{n+nb}{print}\PY{p}{(}\PY{n}{train}\PY{p}{[}\PY{l+s+s1}{\PYZsq{}}\PY{l+s+s1}{CoapplicantIncome}\PY{l+s+s1}{\PYZsq{}}\PY{p}{]}\PY{o}{.}\PY{n}{value\PYZus{}counts}\PY{p}{(}\PY{p}{)}\PY{p}{)}
\end{Verbatim}


    \begin{Verbatim}[commandchars=\\\{\}]
0.0        273
1666.0       5
2083.0       5
2500.0       5
1750.0       3
1459.0       3
2333.0       3
1800.0       3
1625.0       3
2250.0       3
5625.0       3
2451.0       2
1640.0       2
2917.0       2
1560.0       2
20000.0      2
1717.0       2
1950.0       2
1843.0       2
2569.0       2
3750.0       2
4167.0       2
3500.0       2
2925.0       2
1300.0       2
1430.0       2
754.0        2
3167.0       2
1667.0       2
4416.0       2
          {\ldots} 
1881.0       1
1041.0       1
3583.0       1
3013.0       1
7250.0       1
3300.0       1
1302.0       1
1287.0       1
2340.0       1
1710.0       1
2330.0       1
4648.0       1
3447.0       1
1774.0       1
3796.0       1
1425.0       1
5701.0       1
2079.0       1
3033.0       1
5302.0       1
1131.0       1
7101.0       1
5500.0       1
1779.0       1
1863.0       1
7166.0       1
2138.0       1
2166.0       1
3541.0       1
3021.0       1
Name: CoapplicantIncome, Length: 287, dtype: int64

    \end{Verbatim}

    \begin{Verbatim}[commandchars=\\\{\}]
{\color{incolor}In [{\color{incolor}157}]:} \PY{n+nb}{print}\PY{p}{(}\PY{n}{f}\PY{l+s+s1}{\PYZsq{}\PYZsq{}\PYZsq{}}\PY{l+s+s1}{It was observed that }\PY{l+s+s1}{\PYZob{}}\PY{l+s+s1}{train[}\PY{l+s+s1}{\PYZsq{}}\PY{l+s+s1}{CoapplicantIncome}\PY{l+s+s1}{\PYZsq{}}\PY{l+s+s1}{].value\PYZus{}counts().max()\PYZcb{} loan obervations have }
          \PY{l+s+s1}{      }\PY{l+s+s1}{\PYZob{}}\PY{l+s+s1}{train[}\PY{l+s+s1}{\PYZsq{}}\PY{l+s+s1}{CoapplicantIncome}\PY{l+s+s1}{\PYZsq{}}\PY{l+s+s1}{].value\PYZus{}counts().idxmax()\PYZcb{} coapplicantIncome}\PY{l+s+s1}{\PYZsq{}\PYZsq{}\PYZsq{}}\PY{p}{)}
          
          
          \PY{n+nb}{print}\PY{p}{(}\PY{l+s+s1}{\PYZsq{}\PYZsq{}\PYZsq{}}\PY{l+s+se}{\PYZbs{}n}\PY{l+s+s1}{This can implies that those loan requests with 0.0 coapplicant income involved only one person}
          \PY{l+s+s1}{OR the coapplicant has no income value OR he/she is unemployed}\PY{l+s+s1}{\PYZsq{}\PYZsq{}\PYZsq{}}\PY{p}{)}
\end{Verbatim}


    \begin{Verbatim}[commandchars=\\\{\}]
It was observed that 273 loan obervations have 
      0.0 coapplicantIncome

This can implies that those loan requests with 0.0 coapplicant income involved only one person
OR the coapplicant has no income value OR he/she is unemployed

    \end{Verbatim}

    \begin{Verbatim}[commandchars=\\\{\}]
{\color{incolor}In [{\color{incolor}158}]:} \PY{c+c1}{\PYZsh{} Copy of original data so that the original is retained even after making changes}
          \PY{n}{train\PYZus{}original}\PY{o}{=}\PY{n}{train}\PY{o}{.}\PY{n}{copy}\PY{p}{(}\PY{p}{)}
          \PY{n}{test\PYZus{}original}\PY{o}{=}\PY{n}{test}\PY{o}{.}\PY{n}{copy}\PY{p}{(}\PY{p}{)}
\end{Verbatim}


    \textbf{\emph{Features in the train dataset}}

    \begin{Verbatim}[commandchars=\\\{\}]
{\color{incolor}In [{\color{incolor}159}]:} \PY{c+c1}{\PYZsh{} Features in the dataset}
          \PY{n+nb}{print}\PY{p}{(}\PY{n}{train}\PY{o}{.}\PY{n}{columns}\PY{o}{.}\PY{n}{to\PYZus{}frame}\PY{p}{(}\PY{n}{index}\PY{o}{=}\PY{k+kc}{False}\PY{p}{)}\PY{p}{)}
\end{Verbatim}


    \begin{Verbatim}[commandchars=\\\{\}]
                    0
0             Loan\_ID
1              Gender
2             Married
3          Dependents
4           Education
5       Self\_Employed
6     ApplicantIncome
7   CoapplicantIncome
8          LoanAmount
9    Loan\_Amount\_Term
10     Credit\_History
11      Property\_Area
12        Loan\_Status

    \end{Verbatim}

    \emph{A proof that loan status is missing in the test dataset}

    \begin{Verbatim}[commandchars=\\\{\}]
{\color{incolor}In [{\color{incolor}160}]:} \PY{c+c1}{\PYZsh{} loan status missing in the test dataset}
          \PY{n+nb}{print}\PY{p}{(}\PY{n}{test}\PY{o}{.}\PY{n}{columns}\PY{o}{.}\PY{n}{to\PYZus{}frame}\PY{p}{(}\PY{n}{index}\PY{o}{=}\PY{k+kc}{False}\PY{p}{)}\PY{p}{)}
\end{Verbatim}


    \begin{Verbatim}[commandchars=\\\{\}]
                    0
0             Loan\_ID
1              Gender
2             Married
3          Dependents
4           Education
5       Self\_Employed
6     ApplicantIncome
7   CoapplicantIncome
8          LoanAmount
9    Loan\_Amount\_Term
10     Credit\_History
11      Property\_Area

    \end{Verbatim}

    \textbf{Print data types for each variable}

    \begin{Verbatim}[commandchars=\\\{\}]
{\color{incolor}In [{\color{incolor}161}]:} \PY{c+c1}{\PYZsh{} Print data types for each variable}
          \PY{n}{train}\PY{o}{.}\PY{n}{dtypes}
\end{Verbatim}


\begin{Verbatim}[commandchars=\\\{\}]
{\color{outcolor}Out[{\color{outcolor}161}]:} Loan\_ID               object
          Gender                object
          Married               object
          Dependents            object
          Education             object
          Self\_Employed         object
          ApplicantIncome        int64
          CoapplicantIncome    float64
          LoanAmount           float64
          Loan\_Amount\_Term     float64
          Credit\_History       float64
          Property\_Area         object
          Loan\_Status           object
          dtype: object
\end{Verbatim}
            
    Data types for each features is okay by me and they can be use for
further action

The types of variables in the dataset are Categorical, ordinal and
numerical.

\begin{quote}
\textbf{Categorical features}: These features have categories (Gender,
Married, Self\_Employed, Credit\_History, Loan\_Status)
\end{quote}

\begin{quote}
\textbf{Ordinal features}: Variables in categorical features having some
order involved (Dependents, Education, Property\_Area)
\end{quote}

\begin{quote}
\textbf{Numerical features}: These features have numerical values
(ApplicantIncome, CoapplicantIncome, LoanAmount, Loan\_Amount\_Term)
\end{quote}

    \texttt{The\ number\ of\ unique\ observation\ we\ have\ in\ each\ feature}

    \begin{Verbatim}[commandchars=\\\{\}]
{\color{incolor}In [{\color{incolor}162}]:} \PY{c+c1}{\PYZsh{} get the number of unique observation}
          \PY{n}{train}\PY{o}{.}\PY{n}{nunique}\PY{p}{(}\PY{p}{)}
\end{Verbatim}


\begin{Verbatim}[commandchars=\\\{\}]
{\color{outcolor}Out[{\color{outcolor}162}]:} Loan\_ID              614
          Gender                 2
          Married                2
          Dependents             4
          Education              2
          Self\_Employed          2
          ApplicantIncome      505
          CoapplicantIncome    287
          LoanAmount           203
          Loan\_Amount\_Term      10
          Credit\_History         2
          Property\_Area          3
          Loan\_Status            2
          dtype: int64
\end{Verbatim}
            
    \texttt{Table\ showing\ the\ different\ unique\ observations\ for\ Loan\_Amount\_Term}

    \begin{Verbatim}[commandchars=\\\{\}]
{\color{incolor}In [{\color{incolor}163}]:} \PY{c+c1}{\PYZsh{} check the different unique observation for Loan\PYZus{}Amount\PYZus{}Term}
          \PY{n+nb}{print}\PY{p}{(}\PY{l+s+s1}{\PYZsq{}}\PY{l+s+s1}{Loan\PYZus{}Amount\PYZus{}Term}\PY{l+s+s1}{\PYZsq{}}\PY{p}{)}
          \PY{p}{(}\PY{n}{train}\PY{o}{.}\PY{n}{Loan\PYZus{}Amount\PYZus{}Term}
                            \PY{o}{.}\PY{n}{value\PYZus{}counts}\PY{p}{(}\PY{p}{)}\PY{o}{.}\PY{n}{to\PYZus{}frame}\PY{p}{(}\PY{n}{name}\PY{o}{=}\PY{l+s+s1}{\PYZsq{}}\PY{l+s+s1}{count}\PY{l+s+s1}{\PYZsq{}}\PY{p}{)}\PY{p}{)}
\end{Verbatim}


    \begin{Verbatim}[commandchars=\\\{\}]
Loan\_Amount\_Term

    \end{Verbatim}

\begin{Verbatim}[commandchars=\\\{\}]
{\color{outcolor}Out[{\color{outcolor}163}]:}        count
          360.0    512
          180.0     44
          480.0     15
          300.0     13
          84.0       4
          240.0      4
          120.0      3
          36.0       2
          60.0       2
          12.0       1
\end{Verbatim}
            
    \begin{Verbatim}[commandchars=\\\{\}]
{\color{incolor}In [{\color{incolor}164}]:} \PY{c+c1}{\PYZsh{} get a dictionary of value counts to make a quick check for outlier, mistyped parameter, recurring values}
          \PY{n}{dict\PYZus{}of\PYZus{}value\PYZus{}counts} \PY{o}{=} \PY{p}{\PYZob{}}\PY{n}{a}\PY{p}{:}\PY{n}{b}\PY{o}{.}\PY{n}{value\PYZus{}counts}\PY{p}{(}\PY{p}{)} \PY{k}{for} \PY{n}{a}\PY{p}{,}\PY{n}{b} \PY{o+ow}{in} \PY{n}{train}\PY{o}{.}\PY{n}{items}\PY{p}{(}\PY{p}{)}\PY{p}{\PYZcb{}}
\end{Verbatim}


    \begin{Verbatim}[commandchars=\\\{\}]
{\color{incolor}In [{\color{incolor}172}]:} \PY{c+c1}{\PYZsh{} visualize the Loan\PYZus{}Amount\PYZus{}Term to get glimpse of the picture}
          \PY{n}{fig}\PY{o}{=}\PY{n}{plt}\PY{o}{.}\PY{n}{figure}\PY{p}{(}\PY{n}{figsize}\PY{o}{=}\PY{p}{(}\PY{l+m+mi}{8}\PY{p}{,}\PY{l+m+mi}{6}\PY{p}{)}\PY{p}{)}
          \PY{n}{sns}\PY{o}{.}\PY{n}{set}\PY{p}{(}\PY{n}{style}\PY{o}{=}\PY{l+s+s2}{\PYZdq{}}\PY{l+s+s2}{dark}\PY{l+s+s2}{\PYZdq{}}\PY{p}{)}
          \PY{n}{ax}\PY{o}{=}\PY{n}{sns}\PY{o}{.}\PY{n}{countplot}\PY{p}{(}\PY{n}{x}\PY{o}{=}\PY{l+s+s1}{\PYZsq{}}\PY{l+s+s1}{Loan\PYZus{}Amount\PYZus{}Term}\PY{l+s+s1}{\PYZsq{}}\PY{p}{,}\PY{n}{data}\PY{o}{=}\PY{n}{train}\PY{p}{,}
                       \PY{n}{order}\PY{o}{=}\PY{n+nb}{list}\PY{p}{(}\PY{n}{dict\PYZus{}of\PYZus{}value\PYZus{}counts}\PY{p}{[}\PY{l+s+s1}{\PYZsq{}}\PY{l+s+s1}{Loan\PYZus{}Amount\PYZus{}Term}\PY{l+s+s1}{\PYZsq{}}\PY{p}{]}\PY{o}{.}\PY{n}{index}\PY{p}{)}\PY{p}{)}
          \PY{n}{ax}\PY{o}{.}\PY{n}{set\PYZus{}title}\PY{p}{(}\PY{l+s+s1}{\PYZsq{}}\PY{l+s+s1}{Distribution of Loan\PYZus{}Amount\PYZus{}Term in months}\PY{l+s+s1}{\PYZsq{}}\PY{p}{,} \PY{n}{fontsize}\PY{o}{=}\PY{l+m+mi}{15}\PY{p}{)}
          \PY{n}{ax}\PY{o}{.}\PY{n}{set\PYZus{}xlabel}\PY{p}{(}\PY{l+s+s1}{\PYZsq{}}\PY{l+s+s1}{Loan\PYZus{}Amount\PYZus{}Term in months}\PY{l+s+s1}{\PYZsq{}}\PY{p}{,} \PY{n}{fontsize}\PY{o}{=}\PY{l+m+mi}{14}\PY{p}{)}
          \PY{n}{sns}\PY{o}{.}\PY{n}{despine}\PY{p}{(}\PY{n}{left}\PY{o}{=}\PY{k+kc}{True}\PY{p}{,}\PY{n}{bottom}\PY{o}{=}\PY{k+kc}{True}\PY{p}{)}
          \PY{k}{for} \PY{n}{p} \PY{o+ow}{in} \PY{n}{ax}\PY{o}{.}\PY{n}{patches}\PY{p}{:}
              \PY{n}{ax}\PY{o}{.}\PY{n}{annotate}\PY{p}{(}\PY{l+s+s1}{\PYZsq{}}\PY{l+s+si}{\PYZob{}:\PYZcb{}}\PY{l+s+s1}{\PYZsq{}}\PY{o}{.}\PY{n}{format}\PY{p}{(}\PY{n}{p}\PY{o}{.}\PY{n}{get\PYZus{}height}\PY{p}{(}\PY{p}{)}\PY{p}{)}\PY{p}{,} \PY{p}{(}\PY{n}{p}\PY{o}{.}\PY{n}{get\PYZus{}x}\PY{p}{(}\PY{p}{)}\PY{o}{+}\PY{l+m+mf}{0.15}\PY{p}{,} \PY{n}{p}\PY{o}{.}\PY{n}{get\PYZus{}height}\PY{p}{(}\PY{p}{)}\PY{o}{+}\PY{l+m+mi}{1}\PY{p}{)}\PY{p}{)}
\end{Verbatim}


    \begin{center}
    \adjustimage{max size={0.9\linewidth}{0.9\paperheight}}{output_32_0.png}
    \end{center}
    { \hspace*{\fill} \\}
    
    \textbf{\emph{It was observed that majority of the loan request indicate
360 months (30 years) paying back term implying that most of the loan
request are long term loan.}}

    \texttt{Table\ showing\ the\ different\ unique\ observations\ for\ Dependents}

    \begin{Verbatim}[commandchars=\\\{\}]
{\color{incolor}In [{\color{incolor}166}]:} \PY{c+c1}{\PYZsh{} check the different unique observation for Dependent}
          \PY{n}{train}\PY{o}{.}\PY{n}{Dependents}\PY{o}{.}\PY{n}{value\PYZus{}counts}\PY{p}{(}\PY{p}{)}\PY{o}{.}\PY{n}{to\PYZus{}frame}\PY{p}{(}\PY{p}{)}
\end{Verbatim}


\begin{Verbatim}[commandchars=\\\{\}]
{\color{outcolor}Out[{\color{outcolor}166}]:}     Dependents
          0          345
          1          102
          2          101
          3+          51
\end{Verbatim}
            
    \begin{Verbatim}[commandchars=\\\{\}]
{\color{incolor}In [{\color{incolor}173}]:} \PY{c+c1}{\PYZsh{} visualize the Dependents to get glimpse of the picture}
          \PY{n}{fig}\PY{o}{=}\PY{n}{plt}\PY{o}{.}\PY{n}{figure}\PY{p}{(}\PY{n}{figsize}\PY{o}{=}\PY{p}{(}\PY{l+m+mi}{8}\PY{p}{,}\PY{l+m+mi}{6}\PY{p}{)}\PY{p}{)}
          \PY{n}{sns}\PY{o}{.}\PY{n}{set}\PY{p}{(}\PY{n}{style}\PY{o}{=}\PY{l+s+s2}{\PYZdq{}}\PY{l+s+s2}{dark}\PY{l+s+s2}{\PYZdq{}}\PY{p}{)}
          \PY{n}{ax}\PY{o}{=}\PY{n}{sns}\PY{o}{.}\PY{n}{countplot}\PY{p}{(}\PY{n}{x}\PY{o}{=}\PY{l+s+s1}{\PYZsq{}}\PY{l+s+s1}{Dependents}\PY{l+s+s1}{\PYZsq{}}\PY{p}{,}\PY{n}{data}\PY{o}{=}\PY{n}{train}\PY{p}{,}
                       \PY{n}{order}\PY{o}{=}\PY{n+nb}{list}\PY{p}{(}\PY{n}{dict\PYZus{}of\PYZus{}value\PYZus{}counts}\PY{p}{[}\PY{l+s+s1}{\PYZsq{}}\PY{l+s+s1}{Dependents}\PY{l+s+s1}{\PYZsq{}}\PY{p}{]}\PY{o}{.}\PY{n}{index}\PY{p}{)}\PY{p}{)}
          \PY{n}{ax}\PY{o}{.}\PY{n}{set\PYZus{}title}\PY{p}{(}\PY{l+s+s1}{\PYZsq{}}\PY{l+s+s1}{Distribution of Dependents}\PY{l+s+s1}{\PYZsq{}}\PY{p}{,} \PY{n}{fontsize}\PY{o}{=}\PY{l+m+mi}{15}\PY{p}{)}
          \PY{n}{sns}\PY{o}{.}\PY{n}{despine}\PY{p}{(}\PY{n}{left}\PY{o}{=}\PY{k+kc}{True}\PY{p}{,}\PY{n}{bottom}\PY{o}{=}\PY{k+kc}{True}\PY{p}{)}
          \PY{k}{for} \PY{n}{p} \PY{o+ow}{in} \PY{n}{ax}\PY{o}{.}\PY{n}{patches}\PY{p}{:}
              \PY{n}{ax}\PY{o}{.}\PY{n}{annotate}\PY{p}{(}\PY{l+s+s1}{\PYZsq{}}\PY{l+s+si}{\PYZob{}:\PYZcb{}}\PY{l+s+s1}{\PYZsq{}}\PY{o}{.}\PY{n}{format}\PY{p}{(}\PY{n}{p}\PY{o}{.}\PY{n}{get\PYZus{}height}\PY{p}{(}\PY{p}{)}\PY{p}{)}\PY{p}{,} \PY{p}{(}\PY{n}{p}\PY{o}{.}\PY{n}{get\PYZus{}x}\PY{p}{(}\PY{p}{)}\PY{o}{+}\PY{l+m+mf}{0.20}\PY{p}{,} \PY{n}{p}\PY{o}{.}\PY{n}{get\PYZus{}height}\PY{p}{(}\PY{p}{)}\PY{o}{+}\PY{l+m+mi}{1}\PY{p}{)}\PY{p}{)}
\end{Verbatim}


    \begin{center}
    \adjustimage{max size={0.9\linewidth}{0.9\paperheight}}{output_36_0.png}
    \end{center}
    { \hspace*{\fill} \\}
    Majority of the loan were requested from person with no or one dependent.

This factor can be connected to the fact that less external issue are liable to affect paying-back ability i.e. they are likely to pay back.
    \begin{Verbatim}[commandchars=\\\{\}]
{\color{incolor}In [{\color{incolor}179}]:} \PY{c+c1}{\PYZsh{} visualize the Credit\PYZus{}History to get glimpse of the picture}
          \PY{n}{fig}\PY{o}{=}\PY{n}{plt}\PY{o}{.}\PY{n}{figure}\PY{p}{(}\PY{n}{figsize}\PY{o}{=}\PY{p}{(}\PY{l+m+mi}{8}\PY{p}{,}\PY{l+m+mi}{6}\PY{p}{)}\PY{p}{)}
          \PY{n}{sns}\PY{o}{.}\PY{n}{set}\PY{p}{(}\PY{n}{style}\PY{o}{=}\PY{l+s+s2}{\PYZdq{}}\PY{l+s+s2}{dark}\PY{l+s+s2}{\PYZdq{}}\PY{p}{)}
          \PY{n}{ax}\PY{o}{=}\PY{n}{sns}\PY{o}{.}\PY{n}{countplot}\PY{p}{(}\PY{n}{x}\PY{o}{=}\PY{l+s+s1}{\PYZsq{}}\PY{l+s+s1}{Credit\PYZus{}History}\PY{l+s+s1}{\PYZsq{}}\PY{p}{,}\PY{n}{data}\PY{o}{=}\PY{n}{train}\PY{p}{,}
                       \PY{n}{order}\PY{o}{=}\PY{n+nb}{list}\PY{p}{(}\PY{n}{dict\PYZus{}of\PYZus{}value\PYZus{}counts}\PY{p}{[}\PY{l+s+s1}{\PYZsq{}}\PY{l+s+s1}{Credit\PYZus{}History}\PY{l+s+s1}{\PYZsq{}}\PY{p}{]}\PY{o}{.}\PY{n}{index}\PY{p}{)}\PY{p}{)}
          \PY{n}{ax}\PY{o}{.}\PY{n}{set\PYZus{}title}\PY{p}{(}\PY{l+s+s1}{\PYZsq{}}\PY{l+s+s1}{Distribution of Credit History}\PY{l+s+s1}{\PYZsq{}}\PY{p}{,} \PY{n}{fontsize}\PY{o}{=}\PY{l+m+mi}{15}\PY{p}{)}
          \PY{n}{ax}\PY{o}{.}\PY{n}{legend}\PY{p}{(}\PY{p}{[}\PY{l+s+s1}{\PYZsq{}}\PY{l+s+s1}{Applicants meet previous loan guidelines}\PY{l+s+s1}{\PYZsq{}}\PY{p}{]}\PY{p}{)}
          \PY{n}{sns}\PY{o}{.}\PY{n}{despine}\PY{p}{(}\PY{n}{left}\PY{o}{=}\PY{k+kc}{True}\PY{p}{,}\PY{n}{bottom}\PY{o}{=}\PY{k+kc}{True}\PY{p}{)}
          \PY{k}{for} \PY{n}{p} \PY{o+ow}{in} \PY{n}{ax}\PY{o}{.}\PY{n}{patches}\PY{p}{:}
              \PY{n}{ax}\PY{o}{.}\PY{n}{annotate}\PY{p}{(}\PY{l+s+s1}{\PYZsq{}}\PY{l+s+si}{\PYZob{}:.2f\PYZcb{}}\PY{l+s+s1}{\PYZpc{}}\PY{l+s+s1}{\PYZsq{}}\PY{o}{.}\PY{n}{format}\PY{p}{(}\PY{p}{(}\PY{n}{p}\PY{o}{.}\PY{n}{get\PYZus{}height}\PY{p}{(}\PY{p}{)}\PY{o}{/}\PY{n}{dict\PYZus{}of\PYZus{}value\PYZus{}counts}\PY{p}{[}\PY{l+s+s1}{\PYZsq{}}\PY{l+s+s1}{Credit\PYZus{}History}\PY{l+s+s1}{\PYZsq{}}\PY{p}{]}\PY{o}{.}\PY{n}{sum}\PY{p}{(}\PY{p}{)}\PY{p}{)}\PY{o}{*}\PY{l+m+mi}{100}\PY{p}{)}\PY{p}{,}
                          \PY{p}{(}\PY{n}{p}\PY{o}{.}\PY{n}{get\PYZus{}x}\PY{p}{(}\PY{p}{)}\PY{o}{+}\PY{l+m+mf}{0.30}\PY{p}{,} \PY{n}{p}\PY{o}{.}\PY{n}{get\PYZus{}height}\PY{p}{(}\PY{p}{)}\PY{o}{+}\PY{l+m+mi}{1}\PY{p}{)}\PY{p}{)}
\end{Verbatim}


    \begin{center}
    \adjustimage{max size={0.9\linewidth}{0.9\paperheight}}{output_38_0.png}
    \end{center}
    { \hspace*{\fill} \\}
    
    \texttt{It\ was\ observed\ that\ 84\%\ of\ those\ requesting\ for\ loan\ have\ once\ met\ a\ previous\ loan\ guidelines\ i.e.\ have\ repaid\ their\ debts.}

    \subsubsection{Other Categorical
Variable}\label{other-categorical-variable}

    \begin{Verbatim}[commandchars=\\\{\}]
{\color{incolor}In [{\color{incolor}181}]:} \PY{n}{plt}\PY{o}{.}\PY{n}{figure}\PY{p}{(}\PY{n}{figsize}\PY{o}{=}\PY{p}{(}\PY{l+m+mi}{10}\PY{p}{,}\PY{l+m+mi}{12}\PY{p}{)}\PY{p}{)}
          \PY{n}{a} \PY{o}{=} \PY{p}{[}\PY{l+s+s1}{\PYZsq{}}\PY{l+s+s1}{Gender}\PY{l+s+s1}{\PYZsq{}}\PY{p}{,}\PY{l+s+s1}{\PYZsq{}}\PY{l+s+s1}{Married}\PY{l+s+s1}{\PYZsq{}}\PY{p}{,}\PY{l+s+s1}{\PYZsq{}}\PY{l+s+s1}{Self\PYZus{}Employed}\PY{l+s+s1}{\PYZsq{}}\PY{p}{,}\PY{l+s+s1}{\PYZsq{}}\PY{l+s+s1}{Loan\PYZus{}Status}\PY{l+s+s1}{\PYZsq{}}\PY{p}{]}
          \PY{k}{for} \PY{n}{x}\PY{p}{,}\PY{n}{y} \PY{o+ow}{in} \PY{n+nb}{enumerate}\PY{p}{(}\PY{n+nb}{range}\PY{p}{(}\PY{l+m+mi}{221}\PY{p}{,}\PY{l+m+mi}{225}\PY{p}{)}\PY{p}{)}\PY{p}{:}
              \PY{n}{plt}\PY{o}{.}\PY{n}{subplot}\PY{p}{(}\PY{n}{y}\PY{p}{)}
              \PY{n}{ax}\PY{o}{=}\PY{n}{sns}\PY{o}{.}\PY{n}{countplot}\PY{p}{(}\PY{n}{x}\PY{o}{=}\PY{n}{a}\PY{p}{[}\PY{n}{x}\PY{p}{]}\PY{p}{,}\PY{n}{data}\PY{o}{=}\PY{n}{train}\PY{p}{,}
                           \PY{n}{order}\PY{o}{=}\PY{n+nb}{list}\PY{p}{(}\PY{n}{dict\PYZus{}of\PYZus{}value\PYZus{}counts}\PY{p}{[}\PY{n}{a}\PY{p}{[}\PY{n}{x}\PY{p}{]}\PY{p}{]}\PY{o}{.}\PY{n}{index}\PY{p}{)}\PY{p}{)}
              \PY{n}{ax}\PY{o}{.}\PY{n}{set\PYZus{}title}\PY{p}{(}\PY{l+s+s1}{\PYZsq{}}\PY{l+s+s1}{Distribution of }\PY{l+s+si}{\PYZob{}\PYZcb{}}\PY{l+s+s1}{\PYZsq{}}\PY{o}{.}\PY{n}{format}\PY{p}{(}\PY{n}{a}\PY{p}{[}\PY{n}{x}\PY{p}{]}\PY{p}{)}\PY{p}{,} \PY{n}{fontsize}\PY{o}{=}\PY{l+m+mi}{15}\PY{p}{)}
              \PY{n}{sns}\PY{o}{.}\PY{n}{despine}\PY{p}{(}\PY{n}{left}\PY{o}{=}\PY{k+kc}{True}\PY{p}{,}\PY{n}{bottom}\PY{o}{=}\PY{k+kc}{True}\PY{p}{)}
              \PY{k}{for} \PY{n}{p} \PY{o+ow}{in} \PY{n}{ax}\PY{o}{.}\PY{n}{patches}\PY{p}{:}
                  \PY{n}{ax}\PY{o}{.}\PY{n}{annotate}\PY{p}{(}\PY{l+s+s1}{\PYZsq{}}\PY{l+s+si}{\PYZob{}:.2f\PYZcb{}}\PY{l+s+s1}{\PYZpc{}}\PY{l+s+s1}{\PYZsq{}}\PY{o}{.}\PY{n}{format}\PY{p}{(}\PY{p}{(}\PY{n}{p}\PY{o}{.}\PY{n}{get\PYZus{}height}\PY{p}{(}\PY{p}{)}\PY{o}{/}\PY{n}{dict\PYZus{}of\PYZus{}value\PYZus{}counts}\PY{p}{[}\PY{n}{a}\PY{p}{[}\PY{n}{x}\PY{p}{]}\PY{p}{]}\PY{o}{.}\PY{n}{sum}\PY{p}{(}\PY{p}{)}\PY{p}{)}\PY{o}{*}\PY{l+m+mi}{100}\PY{p}{)}\PY{p}{,}
                              \PY{p}{(}\PY{n}{p}\PY{o}{.}\PY{n}{get\PYZus{}x}\PY{p}{(}\PY{p}{)}\PY{o}{+}\PY{l+m+mf}{0.30}\PY{p}{,} \PY{n}{p}\PY{o}{.}\PY{n}{get\PYZus{}height}\PY{p}{(}\PY{p}{)}\PY{o}{+}\PY{l+m+mi}{1}\PY{p}{)}\PY{p}{)}
              
\end{Verbatim}


    \begin{center}
    \adjustimage{max size={0.9\linewidth}{0.9\paperheight}}{output_41_0.png}
    \end{center}
    { \hspace*{\fill} \\}
    
    \begin{Verbatim}[commandchars=\\\{\}]
{\color{incolor}In [{\color{incolor}182}]:} \PY{n}{plt}\PY{o}{.}\PY{n}{figure}\PY{p}{(}\PY{n}{figsize}\PY{o}{=}\PY{p}{(}\PY{l+m+mi}{8}\PY{p}{,}\PY{l+m+mi}{4}\PY{p}{)}\PY{p}{)}
          \PY{n}{a} \PY{o}{=} \PY{p}{[}\PY{l+s+s1}{\PYZsq{}}\PY{l+s+s1}{Education}\PY{l+s+s1}{\PYZsq{}}\PY{p}{,}\PY{l+s+s1}{\PYZsq{}}\PY{l+s+s1}{Property\PYZus{}Area}\PY{l+s+s1}{\PYZsq{}}\PY{p}{]}
          \PY{k}{for} \PY{n}{x}\PY{p}{,}\PY{n}{y} \PY{o+ow}{in} \PY{n+nb}{enumerate}\PY{p}{(}\PY{n+nb}{range}\PY{p}{(}\PY{l+m+mi}{121}\PY{p}{,}\PY{l+m+mi}{123}\PY{p}{)}\PY{p}{)}\PY{p}{:}
              \PY{n}{plt}\PY{o}{.}\PY{n}{subplot}\PY{p}{(}\PY{n}{y}\PY{p}{)}
              \PY{n}{ax}\PY{o}{=}\PY{n}{sns}\PY{o}{.}\PY{n}{countplot}\PY{p}{(}\PY{n}{x}\PY{o}{=}\PY{n}{a}\PY{p}{[}\PY{n}{x}\PY{p}{]}\PY{p}{,}\PY{n}{data}\PY{o}{=}\PY{n}{train}\PY{p}{,}
                           \PY{n}{order}\PY{o}{=}\PY{n+nb}{list}\PY{p}{(}\PY{n}{dict\PYZus{}of\PYZus{}value\PYZus{}counts}\PY{p}{[}\PY{n}{a}\PY{p}{[}\PY{n}{x}\PY{p}{]}\PY{p}{]}\PY{o}{.}\PY{n}{index}\PY{p}{)}\PY{p}{)}
              \PY{n}{ax}\PY{o}{.}\PY{n}{set\PYZus{}ylabel}\PY{p}{(}\PY{l+s+s1}{\PYZsq{}}\PY{l+s+s1}{\PYZsq{}}\PY{p}{)}
              \PY{n}{ax}\PY{o}{.}\PY{n}{set\PYZus{}title}\PY{p}{(}\PY{l+s+s1}{\PYZsq{}}\PY{l+s+s1}{Distribution of }\PY{l+s+si}{\PYZob{}\PYZcb{}}\PY{l+s+s1}{\PYZsq{}}\PY{o}{.}\PY{n}{format}\PY{p}{(}\PY{n}{a}\PY{p}{[}\PY{n}{x}\PY{p}{]}\PY{p}{)}\PY{p}{,} \PY{n}{fontsize}\PY{o}{=}\PY{l+m+mi}{15}\PY{p}{)}
              \PY{n}{sns}\PY{o}{.}\PY{n}{despine}\PY{p}{(}\PY{n}{left}\PY{o}{=}\PY{k+kc}{True}\PY{p}{,}\PY{n}{bottom}\PY{o}{=}\PY{k+kc}{True}\PY{p}{)}
              \PY{k}{for} \PY{n}{p} \PY{o+ow}{in} \PY{n}{ax}\PY{o}{.}\PY{n}{patches}\PY{p}{:}
                  \PY{n}{ax}\PY{o}{.}\PY{n}{annotate}\PY{p}{(}\PY{l+s+s1}{\PYZsq{}}\PY{l+s+si}{\PYZob{}:.2f\PYZcb{}}\PY{l+s+s1}{\PYZpc{}}\PY{l+s+s1}{\PYZsq{}}\PY{o}{.}\PY{n}{format}\PY{p}{(}\PY{p}{(}\PY{n}{p}\PY{o}{.}\PY{n}{get\PYZus{}height}\PY{p}{(}\PY{p}{)}\PY{o}{/}\PY{n}{dict\PYZus{}of\PYZus{}value\PYZus{}counts}\PY{p}{[}\PY{n}{a}\PY{p}{[}\PY{n}{x}\PY{p}{]}\PY{p}{]}\PY{o}{.}\PY{n}{sum}\PY{p}{(}\PY{p}{)}\PY{p}{)}\PY{o}{*}\PY{l+m+mi}{100}\PY{p}{)}\PY{p}{,}
                              \PY{p}{(}\PY{n}{p}\PY{o}{.}\PY{n}{get\PYZus{}x}\PY{p}{(}\PY{p}{)}\PY{o}{+}\PY{l+m+mf}{0.30}\PY{p}{,} \PY{n}{p}\PY{o}{.}\PY{n}{get\PYZus{}height}\PY{p}{(}\PY{p}{)}\PY{o}{+}\PY{l+m+mi}{1}\PY{p}{)}\PY{p}{)}
              
\end{Verbatim}


    \begin{center}
    \adjustimage{max size={0.9\linewidth}{0.9\paperheight}}{output_42_0.png}
    \end{center}
    { \hspace*{\fill} \\}
    
    \textbf{The following inferences can be made from the above plots:}

\begin{itemize}
\item
  \texttt{Most\ of\ the\ applicants\ don’t\ have\ any\ dependents.}
\item
  \texttt{Around\ 80\%\ of\ the\ applicants\ are\ Graduate.}
\item
  \texttt{Most\ of\ the\ applicants\ are\ from\ Semiurban\ area.}
\end{itemize}

    \textbf{Descriptive statistic of categorical features shown below}

    \begin{Verbatim}[commandchars=\\\{\}]
{\color{incolor}In [{\color{incolor}183}]:} \PY{n}{train}\PY{o}{.}\PY{n}{describe}\PY{p}{(}\PY{n}{include}\PY{o}{=}\PY{p}{[}\PY{l+s+s1}{\PYZsq{}}\PY{l+s+s1}{O}\PY{l+s+s1}{\PYZsq{}}\PY{p}{]}\PY{p}{)}
\end{Verbatim}


\begin{Verbatim}[commandchars=\\\{\}]
{\color{outcolor}Out[{\color{outcolor}183}]:}          Loan\_ID Gender Married Dependents Education Self\_Employed  \textbackslash{}
          count        614    601     611        599       614           582   
          unique       614      2       2          4         2             2   
          top     LP002422   Male     Yes          0  Graduate            No   
          freq           1    489     398        345       480           500   
          
                 Property\_Area Loan\_Status  
          count            614         614  
          unique             3           2  
          top        Semiurban           Y  
          freq             233         422  
\end{Verbatim}
            
    \texttt{From\ the\ above\ table,\ it\ was\ revealed\ that\ Gender,\ Married,\ Dependants,\ Self\_Employed\ have\ missing\ values.}

    \textbf{\emph{Checking for missing values in categorical variables.}}

    \begin{Verbatim}[commandchars=\\\{\}]
{\color{incolor}In [{\color{incolor}26}]:} \PY{c+c1}{\PYZsh{} check for missing values in categorical variables}
         \PY{n}{train}\PY{o}{.}\PY{n}{select\PYZus{}dtypes}\PY{p}{(}\PY{n}{exclude}\PY{o}{=}\PY{n}{np}\PY{o}{.}\PY{n}{number}\PY{p}{)}\PY{o}{.}\PY{n}{info}\PY{p}{(}\PY{n}{memory\PYZus{}usage}\PY{o}{=}\PY{l+s+s1}{\PYZsq{}}\PY{l+s+s1}{deep}\PY{l+s+s1}{\PYZsq{}}\PY{p}{)}
\end{Verbatim}


    \begin{Verbatim}[commandchars=\\\{\}]
<class 'pandas.core.frame.DataFrame'>
RangeIndex: 614 entries, 0 to 613
Data columns (total 8 columns):
Loan\_ID          614 non-null object
Gender           601 non-null object
Married          611 non-null object
Dependents       599 non-null object
Education        614 non-null object
Self\_Employed    582 non-null object
Property\_Area    614 non-null object
Loan\_Status      614 non-null object
dtypes: object(8)
memory usage: 332.5 KB

    \end{Verbatim}

    \begin{Verbatim}[commandchars=\\\{\}]
{\color{incolor}In [{\color{incolor}186}]:} \PY{c+c1}{\PYZsh{} obtain column with missing values,their dtypes and frequency}
          \PY{n}{num\PYZus{}of\PYZus{}missing\PYZus{}values}\PY{o}{=}\PY{n}{train}\PY{o}{.}\PY{n}{isnull}\PY{p}{(}\PY{p}{)}\PY{o}{.}\PY{n}{apply}\PY{p}{(}\PY{k}{lambda} \PY{n}{x}\PY{p}{:}\PY{n+nb}{sum}\PY{p}{(}\PY{n}{x}\PY{p}{)}\PY{p}{)}
          
          \PY{n}{col\PYZus{}with\PYZus{}missing\PYZus{}values}\PY{o}{=}\PY{n}{num\PYZus{}of\PYZus{}missing\PYZus{}values}\PY{o}{.}\PY{n}{loc}\PY{p}{[}\PY{n}{num\PYZus{}of\PYZus{}missing\PYZus{}values}\PY{o}{\PYZgt{}}\PY{l+m+mi}{0}\PY{p}{]}
          \PY{n+nb}{print}\PY{p}{(}\PY{l+s+s1}{\PYZsq{}}\PY{l+s+s1}{column with missing values,their dtypes and frequency}\PY{l+s+s1}{\PYZsq{}}\PY{p}{)}
          
          \PY{p}{[}\PY{n}{f}\PY{l+s+s1}{\PYZsq{}}\PY{l+s+si}{\PYZob{}a\PYZcb{}}\PY{l+s+s1}{ : }\PY{l+s+si}{\PYZob{}train[a].dtypes\PYZcb{}}\PY{l+s+s1}{: }\PY{l+s+si}{\PYZob{}num\PYZus{}of\PYZus{}missing\PYZus{}values[a]\PYZcb{}}\PY{l+s+s1}{\PYZsq{}} 
           \PY{k}{for} \PY{n}{a} \PY{o+ow}{in} \PY{n}{col\PYZus{}with\PYZus{}missing\PYZus{}values}\PY{o}{.}\PY{n}{index}\PY{p}{]} 
\end{Verbatim}


    \begin{Verbatim}[commandchars=\\\{\}]
column with missing values,their dtypes and frequency

    \end{Verbatim}

\begin{Verbatim}[commandchars=\\\{\}]
{\color{outcolor}Out[{\color{outcolor}186}]:} ['Gender : object: 13',
           'Married : object: 3',
           'Dependents : object: 15',
           'Self\_Employed : object: 32',
           'LoanAmount : float64: 22',
           'Loan\_Amount\_Term : float64: 14',
           'Credit\_History : float64: 50']
\end{Verbatim}
            
    \begin{Verbatim}[commandchars=\\\{\}]
{\color{incolor}In [{\color{incolor}187}]:} \PY{c+c1}{\PYZsh{} display it as a dataframe}
          \PY{n}{col\PYZus{}with\PYZus{}missing\PYZus{}values}\PY{o}{=}\PY{n}{col\PYZus{}with\PYZus{}missing\PYZus{}values}\PY{o}{.}\PY{n}{to\PYZus{}frame}\PY{p}{(}\PY{n}{name}\PY{o}{=}\PY{l+s+s1}{\PYZsq{}}\PY{l+s+s1}{num\PYZus{}missing\PYZus{}val}\PY{l+s+s1}{\PYZsq{}}\PY{p}{)}
          \PY{n}{col\PYZus{}with\PYZus{}missing\PYZus{}values}
\end{Verbatim}


\begin{Verbatim}[commandchars=\\\{\}]
{\color{outcolor}Out[{\color{outcolor}187}]:}                   num\_missing\_val
          Gender                         13
          Married                         3
          Dependents                     15
          Self\_Employed                  32
          LoanAmount                     22
          Loan\_Amount\_Term               14
          Credit\_History                 50
\end{Verbatim}
            
    \subsubsection{Bivariate Analysis of Categorical Independent variables
and the target
variable}\label{bivariate-analysis-of-categorical-independent-variables-and-the-target-variable}

\textbf{Since the target variable is a categorical variable, and we
interested to find the relationship between this response and other
categorical varaiable (Categorical vs Catgorical)}

\textbf{\emph{The question of interest will be how associated the two
features are?}}

\textbf{This finding can easily be obtained with a chart (mosaic chart)}

    \subsubsection{Background Knowledge on finding the association between
two categorical
variables}\label{background-knowledge-on-finding-the-association-between-two-categorical-variables}

\textbf{Stacked Column chart} is a useful graph to visualize the
relationship between two categorical variables. It compares the
percentage that each category from one variable contributes to a total
across categories of the second variable.

\textbf{The chi-square test} can be used to determine the association
between categorical variables. It is based on the difference between the
expected frequencies (e) and the observed frequencies (n) in one or more
categories in the frequency table. The chi-square distribution returns a
probability for the computed chi-square and the degree of freedom. A
probability of zero shows a complete dependency between two categorical
variables and a probability of one means that two categorical variables
are completely independent. Tchouproff Contingency Coefficient measures
the amount of dependency between two categorical variables.

\begin{figure}
\centering
\includegraphics{attachment:image.png}
\caption{image.png}
\end{figure}

    \begin{figure}
\centering
\includegraphics{attachment:image.png}
\caption{image.png}
\end{figure}

    \subsubsection{Interpreting Chi square and more fact to know about
it}\label{interpreting-chi-square-and-more-fact-to-know-about-it}

A large value of the \textbf{Chi Square statistic} indicates a large
deviation from what we would expect with no association.

This test is invalid when the observed or expected frequencies in each
category are too small. A \textbf{typical rule is that all of the
observed and expected frequencies should be at least 5}.

The default degrees of freedom, k-1, are for the case when no parameters
of the distribution are estimated. If p parameters are estimated by
efficient maximum likelihood then the correct degrees of freedom are
k-1-p. If the parameters are estimated in a different way, then the dof
can be between k-1-p and k-1. However, it is also possible that the
asymptotic distribution is not a chisquare, in which case this test is
not appropriate.

\textbf{Chi square goodness of fit} tests whether the distribution of
sample categorical data matches an expected distribution. For example,
you could use a chi-squared goodness-of-fit test to check whether the
race demographics of members at your church or school match that of the
entire U.S. population or whether the computer browser preferences of
your friends match those of Internet uses as a whole.

When working with categorical data, the values themselves aren't of much
use for statistical testing because categories like "male", "female,"
and "other" have no mathematical meaning. Tests dealing with categorical
variables are based on variable counts instead of the actual value of
the variables themselves.

\textbf{Chi-Squared Test of Independence}

\emph{Independence} is a key concept in probability that describes a
situation where knowing the value of one variable tells you nothing
about the value of another. For instance, the month you were born
probably doesn't tell you anything about which web browser you use, so
we'd expect birth month and browser preference to be independent. On the
other hand, your month of birth might be related to whether you excelled
at sports in school, so month of birth and sports performance might not
be independent.

The \textbf{chi-squared test of independence tests whether two
categorical variables are independent. The test of independence is
commonly used to determine whether variables like education, political
views and other preferences vary based on demographic factors like
gender, race and religion}.

\href{http://hamelg.blogspot.com/2015/11/python-for-data-analysis-part-25-chi.html}{Read
more here}

    \begin{quote}
\textbf{Relationship of Gender vs Loan\_Status (how associated gender is
to Loan\_Status)?}
\end{quote}

    \texttt{Table\ showing\ the\ proportion\ of\ Gender\ to\ Loan\_Status\ in\ \%}

    \begin{Verbatim}[commandchars=\\\{\}]
{\color{incolor}In [{\color{incolor}29}]:} \PY{c+c1}{\PYZsh{} Table showing the proportion of Gender to Loan\PYZus{}Status in \PYZpc{} }
         \PY{n}{pd}\PY{o}{.}\PY{n}{crosstab}\PY{p}{(}\PY{n}{train}\PY{o}{.}\PY{n}{Gender}\PY{p}{,}\PY{n}{train}\PY{o}{.}\PY{n}{Loan\PYZus{}Status}\PY{p}{,}\PY{n}{normalize}\PY{o}{=}\PY{k+kc}{True}\PY{p}{)}
\end{Verbatim}


\begin{Verbatim}[commandchars=\\\{\}]
{\color{outcolor}Out[{\color{outcolor}29}]:} Loan\_Status         N         Y
         Gender                         
         Female       0.061564  0.124792
         Male         0.249584  0.564060
\end{Verbatim}
            
    \texttt{Table\ showing\ the\ actual\ proportion\ of\ Gender\ to\ Loan\_Status}

    \begin{Verbatim}[commandchars=\\\{\}]
{\color{incolor}In [{\color{incolor}30}]:} \PY{c+c1}{\PYZsh{} print the cross tabulation of Loan\PYZus{}Status vs Gender}
         \PY{n}{gender} \PY{o}{=} \PY{n}{pd}\PY{o}{.}\PY{n}{crosstab}\PY{p}{(}\PY{n}{train}\PY{o}{.}\PY{n}{Gender}\PY{p}{,}\PY{n}{train}\PY{o}{.}\PY{n}{Loan\PYZus{}Status}\PY{p}{)}
         \PY{n+nb}{print}\PY{p}{(}\PY{n}{gender}\PY{p}{)}
\end{Verbatim}


    \begin{Verbatim}[commandchars=\\\{\}]
Loan\_Status    N    Y
Gender               
Female        37   75
Male         150  339

    \end{Verbatim}

    \texttt{Table\ showing\ the\ percentage\ proportion\ of\ each\ Loan\ Status\ per\ Gender}

    \begin{Verbatim}[commandchars=\\\{\}]
{\color{incolor}In [{\color{incolor}31}]:} \PY{c+c1}{\PYZsh{} obtain the percentage of each Loan\PYZus{}Status per Gender}
         \PY{n}{gender}\PY{o}{.}\PY{n}{divide}\PY{p}{(}\PY{n}{gender}\PY{o}{.}\PY{n}{sum}\PY{p}{(}\PY{n}{axis}\PY{o}{=}\PY{l+m+mi}{1}\PY{p}{)}
              \PY{o}{.}\PY{n}{astype}\PY{p}{(}\PY{n+nb}{float}\PY{p}{)}\PY{p}{,}\PY{n}{axis}\PY{o}{=}\PY{l+m+mi}{0}\PY{p}{)}
\end{Verbatim}


\begin{Verbatim}[commandchars=\\\{\}]
{\color{outcolor}Out[{\color{outcolor}31}]:} Loan\_Status         N         Y
         Gender                         
         Female       0.330357  0.669643
         Male         0.306748  0.693252
\end{Verbatim}
            
    \begin{Verbatim}[commandchars=\\\{\}]
{\color{incolor}In [{\color{incolor}188}]:} \PY{n}{sns}\PY{o}{.}\PY{n}{set\PYZus{}style}\PY{p}{(}\PY{l+s+s1}{\PYZsq{}}\PY{l+s+s1}{dark}\PY{l+s+s1}{\PYZsq{}}\PY{p}{)}
          \PY{p}{(}\PY{n}{gender}\PY{o}{.}\PY{n}{divide}\PY{p}{(}\PY{n}{gender}\PY{o}{.}\PY{n}{sum}\PY{p}{(}\PY{n}{axis}\PY{o}{=}\PY{l+m+mi}{1}\PY{p}{)}
                         \PY{o}{.}\PY{n}{astype}\PY{p}{(}\PY{n+nb}{float}\PY{p}{)}\PY{p}{,} \PY{n}{axis}\PY{o}{=}\PY{l+m+mi}{0}\PY{p}{)}
                          \PY{o}{.}\PY{n}{plot}\PY{p}{(}\PY{n}{kind}\PY{o}{=}\PY{l+s+s2}{\PYZdq{}}\PY{l+s+s2}{bar}\PY{l+s+s2}{\PYZdq{}}\PY{p}{,} \PY{n}{stacked}\PY{o}{=}\PY{k+kc}{True}\PY{p}{,} \PY{n}{figsize}\PY{o}{=}\PY{p}{(}\PY{l+m+mi}{6}\PY{p}{,}\PY{l+m+mi}{4}\PY{p}{)}\PY{p}{)}\PY{p}{)}
          \PY{n}{plt}\PY{o}{.}\PY{n}{title}\PY{p}{(}\PY{l+s+s1}{\PYZsq{}}\PY{l+s+s1}{Relationship between Gender and Loan Status}\PY{l+s+s1}{\PYZsq{}}\PY{p}{)}
          \PY{n}{plt}\PY{o}{.}\PY{n}{xlabel}\PY{p}{(}\PY{l+s+s1}{\PYZsq{}}\PY{l+s+s1}{Gender}\PY{l+s+s1}{\PYZsq{}}\PY{p}{)}
          \PY{n}{plt}\PY{o}{.}\PY{n}{ylabel}\PY{p}{(}\PY{l+s+s1}{\PYZsq{}}\PY{l+s+s1}{Percentage}\PY{l+s+s1}{\PYZsq{}}\PY{p}{)}
          \PY{n}{plt}\PY{o}{.}\PY{n}{text}
\end{Verbatim}


\begin{Verbatim}[commandchars=\\\{\}]
{\color{outcolor}Out[{\color{outcolor}188}]:} <function matplotlib.pyplot.text>
\end{Verbatim}
            
    \begin{center}
    \adjustimage{max size={0.9\linewidth}{0.9\paperheight}}{output_62_1.png}
    \end{center}
    { \hspace*{\fill} \\}
    
    \texttt{It\ can\ be\ inferred\ that\ the\ proportion\ of\ male\ and\ female\ applicants\ is\ more\ or\ less\ same\ for\ both\ approved\ and\ unapproved\ loans.}

    \textbf{Test of Independence with chi square test of Independence}

    \texttt{Expected\ Count\ table\ for\ \ gender\ in\ respect\ to\ Loan\ Status}

    \begin{Verbatim}[commandchars=\\\{\}]
{\color{incolor}In [{\color{incolor}33}]:} \PY{c+c1}{\PYZsh{} We can quickly get the expected counts for all cells in the table }
         \PY{c+c1}{\PYZsh{} by taking the row totals and column totals of the table, }
         \PY{c+c1}{\PYZsh{} performing an outer product on them with the np.outer() function and dividing by the number of observations:}
         
         \PY{n}{col\PYZus{}sum\PYZus{}gender}\PY{o}{=}\PY{n}{gender}\PY{o}{.}\PY{n}{sum}\PY{p}{(}\PY{n}{axis}\PY{o}{=}\PY{l+m+mi}{1}\PY{p}{)}
         \PY{n}{row\PYZus{}sum\PYZus{}gender}\PY{o}{=}\PY{n}{gender}\PY{o}{.}\PY{n}{sum}\PY{p}{(}\PY{n}{axis}\PY{o}{=}\PY{l+m+mi}{0}\PY{p}{)}
         \PY{n}{expected} \PY{o}{=} \PY{n}{np}\PY{o}{.}\PY{n}{outer}\PY{p}{(}\PY{n}{col\PYZus{}sum\PYZus{}gender}\PY{p}{,}\PY{n}{row\PYZus{}sum\PYZus{}gender}\PY{p}{)}\PY{o}{/}\PY{n}{col\PYZus{}sum\PYZus{}gender}\PY{o}{.}\PY{n}{sum}\PY{p}{(}\PY{p}{)}
         \PY{n}{expected} \PY{o}{=} \PY{n}{pd}\PY{o}{.}\PY{n}{DataFrame}\PY{p}{(}\PY{n}{expected}\PY{p}{)}
         \PY{n}{expected}\PY{o}{.}\PY{n}{columns} \PY{o}{=} \PY{p}{[}\PY{l+s+s1}{\PYZsq{}}\PY{l+s+s1}{N}\PY{l+s+s1}{\PYZsq{}}\PY{p}{,}\PY{l+s+s1}{\PYZsq{}}\PY{l+s+s1}{Y}\PY{l+s+s1}{\PYZsq{}}\PY{p}{]}
         \PY{n}{expected}\PY{o}{.}\PY{n}{index} \PY{o}{=} \PY{p}{[}\PY{l+s+s1}{\PYZsq{}}\PY{l+s+s1}{Female}\PY{l+s+s1}{\PYZsq{}}\PY{p}{,}\PY{l+s+s1}{\PYZsq{}}\PY{l+s+s1}{Male}\PY{l+s+s1}{\PYZsq{}}\PY{p}{]}
         \PY{n}{expected}
\end{Verbatim}


\begin{Verbatim}[commandchars=\\\{\}]
{\color{outcolor}Out[{\color{outcolor}33}]:}                  N           Y
         Female   34.848586   77.151414
         Male    152.151414  336.848586
\end{Verbatim}
            
    \emph{Expected counts table is designed to reflect what the sample data
counts would be if the two variables were independent.}

The statistical question becomes, \textbf{"Are the observed counts so
different from the expected counts that we can conclude a relationship
between the two variables?"}

\emph{In this particular case they are no huge difference} \textbf{so we
can initially/partially conclude the variable are independents (no
association between them)}

    \begin{Verbatim}[commandchars=\\\{\}]
{\color{incolor}In [{\color{incolor}189}]:} \PY{c+c1}{\PYZsh{} calculate the chi\PYZhy{}square\PYZhy{}statistic}
          \PY{n}{chi\PYZus{}squared\PYZus{}stat} \PY{o}{=} \PY{p}{(}\PY{p}{(}\PY{p}{(}\PY{p}{(}\PY{n}{gender}\PY{o}{\PYZhy{}}\PY{n}{expected}\PY{p}{)}\PY{o}{*}\PY{o}{*}\PY{l+m+mi}{2}\PY{p}{)}\PY{o}{/}\PY{n}{expected}\PY{p}{)}\PY{o}{.}\PY{n}{sum}\PY{p}{(}\PY{p}{)}\PY{o}{.}\PY{n}{sum}\PY{p}{(}\PY{p}{)}\PY{p}{)}
          \PY{n+nb}{print}\PY{p}{(}\PY{l+s+s1}{\PYZsq{}}\PY{l+s+s1}{chi\PYZus{}squared\PYZus{}stat}\PY{l+s+s1}{\PYZsq{}}\PY{p}{)}
          \PY{n+nb}{print}\PY{p}{(}\PY{n}{chi\PYZus{}squared\PYZus{}stat}\PY{p}{)}
\end{Verbatim}


    \begin{Verbatim}[commandchars=\\\{\}]
chi\_squared\_stat
0.23697508750826923

    \end{Verbatim}

    \begin{Verbatim}[commandchars=\\\{\}]
{\color{incolor}In [{\color{incolor}35}]:} \PY{k+kn}{import} \PY{n+nn}{scipy}\PY{n+nn}{.}\PY{n+nn}{stats} \PY{k}{as} \PY{n+nn}{stats}
         \PY{n}{crit} \PY{o}{=} \PY{n}{stats}\PY{o}{.}\PY{n}{chi2}\PY{o}{.}\PY{n}{ppf}\PY{p}{(}\PY{n}{q} \PY{o}{=} \PY{l+m+mf}{0.95}\PY{p}{,} \PY{c+c1}{\PYZsh{} Find the critical value for 95\PYZpc{} confidence*}
                               \PY{n}{df} \PY{o}{=} \PY{l+m+mi}{1}\PY{p}{)}   \PY{c+c1}{\PYZsh{} *}
         \PY{n}{stats}\PY{o}{.}\PY{n}{chi2}
         \PY{n+nb}{print}\PY{p}{(}\PY{l+s+s2}{\PYZdq{}}\PY{l+s+s2}{Critical value}\PY{l+s+s2}{\PYZdq{}}\PY{p}{)}
         \PY{n+nb}{print}\PY{p}{(}\PY{n}{crit}\PY{p}{)}
         
         \PY{n}{p\PYZus{}value} \PY{o}{=} \PY{l+m+mi}{1} \PY{o}{\PYZhy{}} \PY{n}{stats}\PY{o}{.}\PY{n}{chi2}\PY{o}{.}\PY{n}{cdf}\PY{p}{(}\PY{n}{x}\PY{o}{=}\PY{n}{chi\PYZus{}squared\PYZus{}stat}\PY{p}{,}  \PY{c+c1}{\PYZsh{} Find the p\PYZhy{}value}
                                      \PY{n}{df}\PY{o}{=}\PY{l+m+mi}{1}\PY{p}{)}
         \PY{n+nb}{print}\PY{p}{(}\PY{l+s+s2}{\PYZdq{}}\PY{l+s+s2}{P value}\PY{l+s+s2}{\PYZdq{}}\PY{p}{)}
         \PY{n+nb}{print}\PY{p}{(}\PY{n}{p\PYZus{}value}\PY{p}{)}
\end{Verbatim}


    \begin{Verbatim}[commandchars=\\\{\}]
Critical value
3.8414588206941236
P value
0.6263994534115932

    \end{Verbatim}

    \subsubsection{The decision making
approach}\label{the-decision-making-approach}

We make our decision by either comparing the value of the test statistic
to a critical value (rejection region approach), or by finding the
probability of getting this test statistic value or one more extreme
(p-value approach).

    \textbf{We accept the null hypotheses because critical value of 3.84
\textgreater{} 0.236 of chi squared statistic.}

\emph{We have statistically significant evidence at a =0.05 to show that
null hypotheses is true or that gender and Loan status are independent
(i.e., they are not dependent or related), p \textless{} 0.005. }

    \begin{Verbatim}[commandchars=\\\{\}]
{\color{incolor}In [{\color{incolor}190}]:} \PY{n}{g}\PY{p}{,} \PY{n}{p}\PY{p}{,} \PY{n}{dof}\PY{p}{,} \PY{n}{expctd} \PY{o}{=} \PY{n}{stats}\PY{o}{.}\PY{n}{chi2\PYZus{}contingency}\PY{p}{(}\PY{n}{observed}\PY{o}{=} \PY{n}{gender}\PY{p}{)}
\end{Verbatim}


    \begin{Verbatim}[commandchars=\\\{\}]
{\color{incolor}In [{\color{incolor}191}]:} \PY{n+nb}{print}\PY{p}{(}\PY{n}{f}\PY{l+s+s1}{\PYZsq{}}\PY{l+s+s1}{The chi square statistic is }\PY{l+s+si}{\PYZob{}g\PYZcb{}}\PY{l+s+s1}{\PYZsq{}}\PY{p}{)}
          \PY{n+nb}{print}\PY{p}{(}\PY{n}{f}\PY{l+s+s1}{\PYZsq{}}\PY{l+s+s1}{The P\PYZhy{}value is }\PY{l+s+si}{\PYZob{}p\PYZcb{}}\PY{l+s+s1}{\PYZsq{}}\PY{p}{)}
          \PY{n+nb}{print}\PY{p}{(}\PY{n}{f}\PY{l+s+s1}{\PYZsq{}}\PY{l+s+s1}{The degree of freedom is }\PY{l+s+si}{\PYZob{}dof\PYZcb{}}\PY{l+s+s1}{\PYZsq{}}\PY{p}{)}
          \PY{n+nb}{print}\PY{p}{(}\PY{n}{f}\PY{l+s+s1}{\PYZsq{}}\PY{l+s+s1}{The expected count table }\PY{l+s+s1}{\PYZsq{}}\PY{p}{)}
          \PY{n}{expctd}
\end{Verbatim}


    \begin{Verbatim}[commandchars=\\\{\}]
The chi square statistic is 0.13962612116543877
The P-value is 0.7086529816451106
The degree of freedom is 1
The expected count table 

    \end{Verbatim}

\begin{Verbatim}[commandchars=\\\{\}]
{\color{outcolor}Out[{\color{outcolor}191}]:} array([[ 34.84858569,  77.15141431],
                 [152.15141431, 336.84858569]])
\end{Verbatim}
            
    \emph{We don't have a significant result at 5\% significance level since
the p-value (0.71) is greater than 0.05.}

\textbf{We accept the null hypothesis that Loan\_Status of applicant and
their gender on a loan eligibility process are independent. We conclude
that they are independent, that there is no association between the two
variables.}

    \begin{Verbatim}[commandchars=\\\{\}]
{\color{incolor}In [{\color{incolor}192}]:} \PY{c+c1}{\PYZsh{} Married vs Loan\PYZus{}Status}
          \PY{n}{married} \PY{o}{=} \PY{n}{pd}\PY{o}{.}\PY{n}{crosstab}\PY{p}{(}\PY{n}{train}\PY{o}{.}\PY{n}{Married}\PY{p}{,}\PY{n}{train}\PY{o}{.}\PY{n}{Loan\PYZus{}Status}\PY{p}{)}
          \PY{n+nb}{print}\PY{p}{(}\PY{n}{married}\PY{p}{)}
          
          \PY{p}{(}\PY{n}{married}\PY{o}{.}\PY{n}{divide}\PY{p}{(}\PY{n}{married}\PY{o}{.}\PY{n}{sum}\PY{p}{(}\PY{n}{axis}\PY{o}{=}\PY{l+m+mi}{1}\PY{p}{)}
                         \PY{o}{.}\PY{n}{astype}\PY{p}{(}\PY{n+nb}{float}\PY{p}{)}\PY{p}{,} \PY{n}{axis}\PY{o}{=}\PY{l+m+mi}{0}\PY{p}{)}
                          \PY{o}{.}\PY{n}{plot}\PY{p}{(}\PY{n}{kind}\PY{o}{=}\PY{l+s+s2}{\PYZdq{}}\PY{l+s+s2}{bar}\PY{l+s+s2}{\PYZdq{}}\PY{p}{,} \PY{n}{stacked}\PY{o}{=}\PY{k+kc}{True}\PY{p}{,} \PY{n}{figsize}\PY{o}{=}\PY{p}{(}\PY{l+m+mi}{6}\PY{p}{,}\PY{l+m+mi}{4}\PY{p}{)}\PY{p}{)}\PY{p}{)}
          \PY{n}{plt}\PY{o}{.}\PY{n}{title}\PY{p}{(}\PY{l+s+s1}{\PYZsq{}}\PY{l+s+s1}{Relationship between Married and Loan Status}\PY{l+s+s1}{\PYZsq{}}\PY{p}{)}
          \PY{n}{plt}\PY{o}{.}\PY{n}{xlabel}\PY{p}{(}\PY{l+s+s1}{\PYZsq{}}\PY{l+s+s1}{Married}\PY{l+s+s1}{\PYZsq{}}\PY{p}{)}
          \PY{n}{plt}\PY{o}{.}\PY{n}{ylabel}\PY{p}{(}\PY{l+s+s1}{\PYZsq{}}\PY{l+s+s1}{Percentage}\PY{l+s+s1}{\PYZsq{}}\PY{p}{)}
          \PY{n}{sns}\PY{o}{.}\PY{n}{despine}\PY{p}{(}\PY{n}{left}\PY{o}{=}\PY{k+kc}{True}\PY{p}{,}\PY{n}{bottom}\PY{o}{=}\PY{k+kc}{True}\PY{p}{)}
          \PY{n}{plt}\PY{o}{.}\PY{n}{text}
\end{Verbatim}


    \begin{Verbatim}[commandchars=\\\{\}]
Loan\_Status    N    Y
Married              
No            79  134
Yes          113  285

    \end{Verbatim}

\begin{Verbatim}[commandchars=\\\{\}]
{\color{outcolor}Out[{\color{outcolor}192}]:} <function matplotlib.pyplot.text>
\end{Verbatim}
            
    \begin{center}
    \adjustimage{max size={0.9\linewidth}{0.9\paperheight}}{output_75_2.png}
    \end{center}
    { \hspace*{\fill} \\}
    
    \begin{Verbatim}[commandchars=\\\{\}]
{\color{incolor}In [{\color{incolor}193}]:} \PY{n}{g}\PY{p}{,} \PY{n}{p}\PY{p}{,} \PY{n}{dof}\PY{p}{,} \PY{n}{expctd} \PY{o}{=} \PY{n}{stats}\PY{o}{.}\PY{n}{chi2\PYZus{}contingency}\PY{p}{(}\PY{n}{observed}\PY{o}{=} \PY{n}{married}\PY{p}{)}
          
          \PY{n+nb}{print}\PY{p}{(}\PY{n}{f}\PY{l+s+s1}{\PYZsq{}}\PY{l+s+s1}{The chi square statistic is }\PY{l+s+si}{\PYZob{}g\PYZcb{}}\PY{l+s+s1}{\PYZsq{}}\PY{p}{)}
          \PY{n+nb}{print}\PY{p}{(}\PY{n}{f}\PY{l+s+s1}{\PYZsq{}}\PY{l+s+s1}{The P\PYZhy{}value is }\PY{l+s+si}{\PYZob{}p\PYZcb{}}\PY{l+s+s1}{\PYZsq{}}\PY{p}{)}
          \PY{n+nb}{print}\PY{p}{(}\PY{n}{f}\PY{l+s+s1}{\PYZsq{}}\PY{l+s+s1}{The degree of freedom is }\PY{l+s+si}{\PYZob{}dof\PYZcb{}}\PY{l+s+s1}{\PYZsq{}}\PY{p}{)}
          \PY{n+nb}{print}\PY{p}{(}\PY{n}{f}\PY{l+s+s1}{\PYZsq{}}\PY{l+s+s1}{The expected count table }\PY{l+s+s1}{\PYZsq{}}\PY{p}{)}
          \PY{n}{expctd}
\end{Verbatim}


    \begin{Verbatim}[commandchars=\\\{\}]
The chi square statistic is 4.475019348315097
The P-value is 0.03439381301579988
The degree of freedom is 1
The expected count table 

    \end{Verbatim}

\begin{Verbatim}[commandchars=\\\{\}]
{\color{outcolor}Out[{\color{outcolor}193}]:} array([[ 66.93289689, 146.06710311],
                 [125.06710311, 272.93289689]])
\end{Verbatim}
            
    \begin{Verbatim}[commandchars=\\\{\}]
{\color{incolor}In [{\color{incolor}194}]:} \PY{n}{expctd} \PY{o}{=} \PY{n}{pd}\PY{o}{.}\PY{n}{DataFrame}\PY{p}{(}\PY{n}{expctd}\PY{p}{)}
          \PY{n}{expctd}
\end{Verbatim}


\begin{Verbatim}[commandchars=\\\{\}]
{\color{outcolor}Out[{\color{outcolor}194}]:}             0           1
          0   66.932897  146.067103
          1  125.067103  272.932897
\end{Verbatim}
            
    \begin{Verbatim}[commandchars=\\\{\}]
{\color{incolor}In [{\color{incolor}195}]:} \PY{c+c1}{\PYZsh{} calculate the chi\PYZhy{}square\PYZhy{}statistic}
          \PY{n}{chi\PYZus{}squared\PYZus{}stat} \PY{o}{=} \PY{p}{(}\PY{p}{(}\PY{p}{(}\PY{p}{(}\PY{n}{married}\PY{o}{\PYZhy{}}\PY{n}{expctd}\PY{p}{)}\PY{o}{*}\PY{o}{*}\PY{l+m+mi}{2}\PY{p}{)}\PY{o}{/}\PY{n}{expctd}\PY{p}{)}\PY{o}{.}\PY{n}{sum}\PY{p}{(}\PY{p}{)}\PY{o}{.}\PY{n}{sum}\PY{p}{(}\PY{p}{)}\PY{p}{)}
          
          \PY{c+c1}{\PYZsh{} print(chi\PYZus{}squared\PYZus{}stat)}
          \PY{n}{crit} \PY{o}{=} \PY{n}{stats}\PY{o}{.}\PY{n}{chi2}\PY{o}{.}\PY{n}{ppf}\PY{p}{(}\PY{n}{q} \PY{o}{=} \PY{l+m+mf}{0.95}\PY{p}{,} \PY{c+c1}{\PYZsh{} Find the critical value for 95\PYZpc{} confidence*}
                                \PY{n}{df} \PY{o}{=} \PY{l+m+mi}{1}\PY{p}{)}   \PY{c+c1}{\PYZsh{} *}
          \PY{n+nb}{print}\PY{p}{(}\PY{l+s+s2}{\PYZdq{}}\PY{l+s+s2}{Critical value}\PY{l+s+s2}{\PYZdq{}}\PY{p}{)}
          \PY{n+nb}{print}\PY{p}{(}\PY{n}{crit}\PY{p}{)}
          
          \PY{n}{p\PYZus{}value} \PY{o}{=} \PY{l+m+mi}{1} \PY{o}{\PYZhy{}} \PY{n}{stats}\PY{o}{.}\PY{n}{chi2}\PY{o}{.}\PY{n}{cdf}\PY{p}{(}\PY{n}{x}\PY{o}{=}\PY{n}{chi\PYZus{}squared\PYZus{}stat}\PY{p}{,}  \PY{c+c1}{\PYZsh{} Find the p\PYZhy{}value}
                                       \PY{n}{df}\PY{o}{=}\PY{l+m+mi}{1}\PY{p}{)}
          \PY{c+c1}{\PYZsh{} print(\PYZdq{}P value\PYZdq{})}
          \PY{c+c1}{\PYZsh{} print(p\PYZus{}value)}
\end{Verbatim}


    \begin{Verbatim}[commandchars=\\\{\}]
Critical value
3.8414588206941236

    \end{Verbatim}

    It can be inferred that the Proportion of married applicants is higher
for the approved loans.

\textbf{Critical value of 3.84 \textless{} 4.41 of chi square statistic,
We have a significant result at 5\% significance level since the p-value
(0.03) is less than 0.05.}

\texttt{We\ can\ conclude\ that\ Loan\ Status\ of\ an\ applicant\ is\ associated\ with\ their\ married\ status.}

    \begin{Verbatim}[commandchars=\\\{\}]
{\color{incolor}In [{\color{incolor}196}]:} \PY{c+c1}{\PYZsh{} Dependents vs Loan\PYZus{}Status}
          \PY{n}{dependents} \PY{o}{=} \PY{n}{pd}\PY{o}{.}\PY{n}{crosstab}\PY{p}{(}\PY{n}{train}\PY{o}{.}\PY{n}{Dependents}\PY{p}{,}\PY{n}{train}\PY{o}{.}\PY{n}{Loan\PYZus{}Status}\PY{p}{)}
          \PY{n+nb}{print}\PY{p}{(}\PY{n}{dependents}\PY{p}{)}
          
          \PY{p}{(}\PY{n}{dependents}\PY{o}{.}\PY{n}{divide}\PY{p}{(}\PY{n}{dependents}\PY{o}{.}\PY{n}{sum}\PY{p}{(}\PY{n}{axis}\PY{o}{=}\PY{l+m+mi}{1}\PY{p}{)}
                         \PY{o}{.}\PY{n}{astype}\PY{p}{(}\PY{n+nb}{float}\PY{p}{)}\PY{p}{,} \PY{n}{axis}\PY{o}{=}\PY{l+m+mi}{0}\PY{p}{)}
                          \PY{o}{.}\PY{n}{plot}\PY{p}{(}\PY{n}{kind}\PY{o}{=}\PY{l+s+s2}{\PYZdq{}}\PY{l+s+s2}{bar}\PY{l+s+s2}{\PYZdq{}}\PY{p}{,} \PY{n}{stacked}\PY{o}{=}\PY{k+kc}{True}\PY{p}{,} \PY{n}{figsize}\PY{o}{=}\PY{p}{(}\PY{l+m+mi}{12}\PY{p}{,}\PY{l+m+mi}{4}\PY{p}{)}\PY{p}{)}\PY{p}{)}
          \PY{n}{plt}\PY{o}{.}\PY{n}{title}\PY{p}{(}\PY{l+s+s1}{\PYZsq{}}\PY{l+s+s1}{Relationship between Dependents and Loan Status}\PY{l+s+s1}{\PYZsq{}}\PY{p}{)}
          \PY{n}{plt}\PY{o}{.}\PY{n}{xlabel}\PY{p}{(}\PY{l+s+s1}{\PYZsq{}}\PY{l+s+s1}{Dependents}\PY{l+s+s1}{\PYZsq{}}\PY{p}{)}
          \PY{n}{plt}\PY{o}{.}\PY{n}{ylabel}\PY{p}{(}\PY{l+s+s1}{\PYZsq{}}\PY{l+s+s1}{Percentage}\PY{l+s+s1}{\PYZsq{}}\PY{p}{)}
          \PY{n}{sns}\PY{o}{.}\PY{n}{despine}\PY{p}{(}\PY{n}{left}\PY{o}{=}\PY{k+kc}{True}\PY{p}{,}\PY{n}{bottom}\PY{o}{=}\PY{k+kc}{True}\PY{p}{)}
          \PY{n}{plt}\PY{o}{.}\PY{n}{text}
\end{Verbatim}


    \begin{Verbatim}[commandchars=\\\{\}]
Loan\_Status    N    Y
Dependents           
0            107  238
1             36   66
2             25   76
3+            18   33

    \end{Verbatim}

\begin{Verbatim}[commandchars=\\\{\}]
{\color{outcolor}Out[{\color{outcolor}196}]:} <function matplotlib.pyplot.text>
\end{Verbatim}
            
    \begin{center}
    \adjustimage{max size={0.9\linewidth}{0.9\paperheight}}{output_80_2.png}
    \end{center}
    { \hspace*{\fill} \\}
    
    \begin{Verbatim}[commandchars=\\\{\}]
{\color{incolor}In [{\color{incolor}197}]:} \PY{n}{g}\PY{p}{,} \PY{n}{p}\PY{p}{,} \PY{n}{dof}\PY{p}{,} \PY{n}{expctd} \PY{o}{=} \PY{n}{stats}\PY{o}{.}\PY{n}{chi2\PYZus{}contingency}\PY{p}{(}\PY{n}{observed}\PY{o}{=} \PY{n}{dependents}\PY{p}{)}
          
          \PY{n+nb}{print}\PY{p}{(}\PY{n}{f}\PY{l+s+s1}{\PYZsq{}}\PY{l+s+s1}{The chi square statistic is }\PY{l+s+si}{\PYZob{}g\PYZcb{}}\PY{l+s+s1}{\PYZsq{}}\PY{p}{)}
          \PY{n+nb}{print}\PY{p}{(}\PY{n}{f}\PY{l+s+s1}{\PYZsq{}}\PY{l+s+s1}{The P\PYZhy{}value is }\PY{l+s+si}{\PYZob{}p\PYZcb{}}\PY{l+s+s1}{\PYZsq{}}\PY{p}{)}
          \PY{n+nb}{print}\PY{p}{(}\PY{n}{f}\PY{l+s+s1}{\PYZsq{}}\PY{l+s+s1}{The degree of freedom is }\PY{l+s+si}{\PYZob{}dof\PYZcb{}}\PY{l+s+s1}{\PYZsq{}}\PY{p}{)}
          \PY{n+nb}{print}\PY{p}{(}\PY{n}{f}\PY{l+s+s1}{\PYZsq{}}\PY{l+s+s1}{The expected count table }\PY{l+s+s1}{\PYZsq{}}\PY{p}{)}
          \PY{n}{expctd}
\end{Verbatim}


    \begin{Verbatim}[commandchars=\\\{\}]
The chi square statistic is 3.158339770698263
The P-value is 0.3678506740863211
The degree of freedom is 3
The expected count table 

    \end{Verbatim}

\begin{Verbatim}[commandchars=\\\{\}]
{\color{outcolor}Out[{\color{outcolor}197}]:} array([[107.12854758, 237.87145242],
                 [ 31.67278798,  70.32721202],
                 [ 31.36227045,  69.63772955],
                 [ 15.83639399,  35.16360601]])
\end{Verbatim}
            
    Distribution of applicants with 1 or 3+ dependents is similar across
both the categories of Loan\_Status.

    \begin{Verbatim}[commandchars=\\\{\}]
{\color{incolor}In [{\color{incolor}198}]:} \PY{c+c1}{\PYZsh{} calculate the chi\PYZhy{}square\PYZhy{}statistic}
          \PY{n}{expctd} \PY{o}{=} \PY{n}{pd}\PY{o}{.}\PY{n}{DataFrame}\PY{p}{(}\PY{n}{expctd}\PY{p}{)}
          \PY{n}{expctd}
          \PY{n}{chi\PYZus{}squared\PYZus{}stat} \PY{o}{=} \PY{p}{(}\PY{p}{(}\PY{p}{(}\PY{p}{(}\PY{n}{dependents}\PY{o}{\PYZhy{}}\PY{n}{expctd}\PY{p}{)}\PY{o}{*}\PY{o}{*}\PY{l+m+mi}{2}\PY{p}{)}\PY{o}{/}\PY{n}{expctd}\PY{p}{)}\PY{o}{.}\PY{n}{sum}\PY{p}{(}\PY{p}{)}\PY{o}{.}\PY{n}{sum}\PY{p}{(}\PY{p}{)}\PY{p}{)}
          
          \PY{c+c1}{\PYZsh{} print(chi\PYZus{}squared\PYZus{}stat)}
          \PY{n}{crit} \PY{o}{=} \PY{n}{stats}\PY{o}{.}\PY{n}{chi2}\PY{o}{.}\PY{n}{ppf}\PY{p}{(}\PY{n}{q} \PY{o}{=} \PY{l+m+mf}{0.95}\PY{p}{,} \PY{c+c1}{\PYZsh{} Find the critical value for 95\PYZpc{} confidence*}
                                \PY{n}{df} \PY{o}{=} \PY{l+m+mi}{3}\PY{p}{)}   \PY{c+c1}{\PYZsh{} *}
          \PY{n+nb}{print}\PY{p}{(}\PY{l+s+s2}{\PYZdq{}}\PY{l+s+s2}{Critical value}\PY{l+s+s2}{\PYZdq{}}\PY{p}{)}
          \PY{n+nb}{print}\PY{p}{(}\PY{n}{crit}\PY{p}{)}
          
          \PY{c+c1}{\PYZsh{} p\PYZus{}value = 1 \PYZhy{} stats.chi2.cdf(x=chi\PYZus{}squared\PYZus{}stat,  \PYZsh{} Find the p\PYZhy{}value}
          \PY{c+c1}{\PYZsh{}                              df=3)}
          \PY{c+c1}{\PYZsh{} print(\PYZdq{}P value\PYZdq{})}
          \PY{c+c1}{\PYZsh{} print(p\PYZus{}value)}
\end{Verbatim}


    \begin{Verbatim}[commandchars=\\\{\}]
Critical value
7.8147279032511765

    \end{Verbatim}

    \begin{Verbatim}[commandchars=\\\{\}]
{\color{incolor}In [{\color{incolor}199}]:} \PY{c+c1}{\PYZsh{} Education vs Loan\PYZus{}Status}
          \PY{n}{education} \PY{o}{=} \PY{n}{pd}\PY{o}{.}\PY{n}{crosstab}\PY{p}{(}\PY{n}{train}\PY{o}{.}\PY{n}{Education}\PY{p}{,}\PY{n}{train}\PY{o}{.}\PY{n}{Loan\PYZus{}Status}\PY{p}{)}
          \PY{n+nb}{print}\PY{p}{(}\PY{n}{education}\PY{p}{)}
          
          \PY{p}{(}\PY{n}{education}\PY{o}{.}\PY{n}{divide}\PY{p}{(}\PY{n}{education}\PY{o}{.}\PY{n}{sum}\PY{p}{(}\PY{n}{axis}\PY{o}{=}\PY{l+m+mi}{1}\PY{p}{)}
                         \PY{o}{.}\PY{n}{astype}\PY{p}{(}\PY{n+nb}{float}\PY{p}{)}\PY{p}{,} \PY{n}{axis}\PY{o}{=}\PY{l+m+mi}{0}\PY{p}{)}
                          \PY{o}{.}\PY{n}{plot}\PY{p}{(}\PY{n}{kind}\PY{o}{=}\PY{l+s+s2}{\PYZdq{}}\PY{l+s+s2}{bar}\PY{l+s+s2}{\PYZdq{}}\PY{p}{,} \PY{n}{stacked}\PY{o}{=}\PY{k+kc}{True}\PY{p}{,} \PY{n}{figsize}\PY{o}{=}\PY{p}{(}\PY{l+m+mi}{6}\PY{p}{,}\PY{l+m+mi}{4}\PY{p}{)}\PY{p}{)}\PY{p}{)}
          \PY{n}{plt}\PY{o}{.}\PY{n}{title}\PY{p}{(}\PY{l+s+s1}{\PYZsq{}}\PY{l+s+s1}{Relationship between Education and Loan Status}\PY{l+s+s1}{\PYZsq{}}\PY{p}{)}
          \PY{n}{plt}\PY{o}{.}\PY{n}{xlabel}\PY{p}{(}\PY{l+s+s1}{\PYZsq{}}\PY{l+s+s1}{Education}\PY{l+s+s1}{\PYZsq{}}\PY{p}{)}
          \PY{n}{plt}\PY{o}{.}\PY{n}{ylabel}\PY{p}{(}\PY{l+s+s1}{\PYZsq{}}\PY{l+s+s1}{Percentage}\PY{l+s+s1}{\PYZsq{}}\PY{p}{)}
          \PY{n}{sns}\PY{o}{.}\PY{n}{despine}\PY{p}{(}\PY{n}{left}\PY{o}{=}\PY{k+kc}{True}\PY{p}{,}\PY{n}{bottom}\PY{o}{=}\PY{k+kc}{True}\PY{p}{)}
          \PY{n}{plt}\PY{o}{.}\PY{n}{text}
\end{Verbatim}


    \begin{Verbatim}[commandchars=\\\{\}]
Loan\_Status     N    Y
Education             
Graduate      140  340
Not Graduate   52   82

    \end{Verbatim}

\begin{Verbatim}[commandchars=\\\{\}]
{\color{outcolor}Out[{\color{outcolor}199}]:} <function matplotlib.pyplot.text>
\end{Verbatim}
            
    \begin{center}
    \adjustimage{max size={0.9\linewidth}{0.9\paperheight}}{output_84_2.png}
    \end{center}
    { \hspace*{\fill} \\}
    
    \begin{Verbatim}[commandchars=\\\{\}]
{\color{incolor}In [{\color{incolor}200}]:} \PY{n}{g}\PY{p}{,} \PY{n}{p}\PY{p}{,} \PY{n}{dof}\PY{p}{,} \PY{n}{expctd} \PY{o}{=} \PY{n}{stats}\PY{o}{.}\PY{n}{chi2\PYZus{}contingency}\PY{p}{(}\PY{n}{observed}\PY{o}{=} \PY{n}{education}\PY{p}{)}
          
          \PY{n+nb}{print}\PY{p}{(}\PY{n}{f}\PY{l+s+s1}{\PYZsq{}}\PY{l+s+s1}{The chi square statistic is }\PY{l+s+si}{\PYZob{}g\PYZcb{}}\PY{l+s+s1}{\PYZsq{}}\PY{p}{)}
          \PY{n+nb}{print}\PY{p}{(}\PY{n}{f}\PY{l+s+s1}{\PYZsq{}}\PY{l+s+s1}{The P\PYZhy{}value is }\PY{l+s+si}{\PYZob{}p\PYZcb{}}\PY{l+s+s1}{\PYZsq{}}\PY{p}{)}
          \PY{n+nb}{print}\PY{p}{(}\PY{n}{f}\PY{l+s+s1}{\PYZsq{}}\PY{l+s+s1}{The degree of freedom is }\PY{l+s+si}{\PYZob{}dof\PYZcb{}}\PY{l+s+s1}{\PYZsq{}}\PY{p}{)}
          \PY{n+nb}{print}\PY{p}{(}\PY{n}{f}\PY{l+s+s1}{\PYZsq{}}\PY{l+s+s1}{The expected count table }\PY{l+s+s1}{\PYZsq{}}\PY{p}{)}
          \PY{n}{expctd}
          
          \PY{c+c1}{\PYZsh{} calculate the chi\PYZhy{}square\PYZhy{}statistic}
          \PY{n}{expctd} \PY{o}{=} \PY{n}{pd}\PY{o}{.}\PY{n}{DataFrame}\PY{p}{(}\PY{n}{expctd}\PY{p}{)}
          \PY{n+nb}{print}\PY{p}{(}\PY{n}{expctd}\PY{p}{)}
          \PY{n}{chi\PYZus{}squared\PYZus{}stat} \PY{o}{=} \PY{p}{(}\PY{p}{(}\PY{p}{(}\PY{p}{(}\PY{n}{education}\PY{o}{\PYZhy{}}\PY{n}{expctd}\PY{p}{)}\PY{o}{*}\PY{o}{*}\PY{l+m+mi}{2}\PY{p}{)}\PY{o}{/}\PY{n}{expctd}\PY{p}{)}\PY{o}{.}\PY{n}{sum}\PY{p}{(}\PY{p}{)}\PY{o}{.}\PY{n}{sum}\PY{p}{(}\PY{p}{)}\PY{p}{)}
          
          \PY{c+c1}{\PYZsh{} print(chi\PYZus{}squared\PYZus{}stat)}
          \PY{n}{crit} \PY{o}{=} \PY{n}{stats}\PY{o}{.}\PY{n}{chi2}\PY{o}{.}\PY{n}{ppf}\PY{p}{(}\PY{n}{q} \PY{o}{=} \PY{l+m+mf}{0.95}\PY{p}{,} \PY{c+c1}{\PYZsh{} Find the critical value for 95\PYZpc{} confidence*}
                                \PY{n}{df} \PY{o}{=} \PY{l+m+mi}{1}\PY{p}{)}   \PY{c+c1}{\PYZsh{} *}
          \PY{n+nb}{print}\PY{p}{(}\PY{l+s+s2}{\PYZdq{}}\PY{l+s+s2}{Critical value}\PY{l+s+s2}{\PYZdq{}}\PY{p}{)}
          \PY{n+nb}{print}\PY{p}{(}\PY{n}{crit}\PY{p}{)}
\end{Verbatim}


    \begin{Verbatim}[commandchars=\\\{\}]
The chi square statistic is 4.091490413303621
The P-value is 0.04309962129357355
The degree of freedom is 1
The expected count table 
           0          1
0  150.09772  329.90228
1   41.90228   92.09772
Critical value
3.8414588206941236

    \end{Verbatim}

    \textbf{Critical value of 3.84 \textless{} 4.09 of chi square statistic,
We have a significant result at 5\% significance level since the p-value
(0.04) is less than 0.05.}

\texttt{We\ can\ conclude\ that\ Loan\ Status\ of\ an\ applicant\ is\ associated\ to\ their\ education\ level.}

    \begin{Verbatim}[commandchars=\\\{\}]
{\color{incolor}In [{\color{incolor}201}]:} \PY{c+c1}{\PYZsh{} Self\PYZhy{}employment vs Loan\PYZus{}Status}
          \PY{n}{self\PYZus{}employed} \PY{o}{=} \PY{n}{pd}\PY{o}{.}\PY{n}{crosstab}\PY{p}{(}\PY{n}{train}\PY{o}{.}\PY{n}{Self\PYZus{}Employed}\PY{p}{,}\PY{n}{train}\PY{o}{.}\PY{n}{Loan\PYZus{}Status}\PY{p}{)}
          \PY{n+nb}{print}\PY{p}{(}\PY{n}{self\PYZus{}employed}\PY{p}{)}
          
          \PY{p}{(}\PY{n}{self\PYZus{}employed}\PY{o}{.}\PY{n}{divide}\PY{p}{(}\PY{n}{self\PYZus{}employed}\PY{o}{.}\PY{n}{sum}\PY{p}{(}\PY{n}{axis}\PY{o}{=}\PY{l+m+mi}{1}\PY{p}{)}
                         \PY{o}{.}\PY{n}{astype}\PY{p}{(}\PY{n+nb}{float}\PY{p}{)}\PY{p}{,} \PY{n}{axis}\PY{o}{=}\PY{l+m+mi}{0}\PY{p}{)}
                          \PY{o}{.}\PY{n}{plot}\PY{p}{(}\PY{n}{kind}\PY{o}{=}\PY{l+s+s2}{\PYZdq{}}\PY{l+s+s2}{bar}\PY{l+s+s2}{\PYZdq{}}\PY{p}{,} \PY{n}{stacked}\PY{o}{=}\PY{k+kc}{True}\PY{p}{,} \PY{n}{figsize}\PY{o}{=}\PY{p}{(}\PY{l+m+mi}{6}\PY{p}{,}\PY{l+m+mi}{4}\PY{p}{)}\PY{p}{)}\PY{p}{)}
          \PY{n}{plt}\PY{o}{.}\PY{n}{title}\PY{p}{(}\PY{l+s+s1}{\PYZsq{}}\PY{l+s+s1}{Relationship between Self\PYZhy{}employment and Loan Status}\PY{l+s+s1}{\PYZsq{}}\PY{p}{)}
          \PY{n}{plt}\PY{o}{.}\PY{n}{xlabel}\PY{p}{(}\PY{l+s+s1}{\PYZsq{}}\PY{l+s+s1}{Self\PYZhy{}employed}\PY{l+s+s1}{\PYZsq{}}\PY{p}{)}
          \PY{n}{plt}\PY{o}{.}\PY{n}{ylabel}\PY{p}{(}\PY{l+s+s1}{\PYZsq{}}\PY{l+s+s1}{Percentage}\PY{l+s+s1}{\PYZsq{}}\PY{p}{)}
          \PY{n}{sns}\PY{o}{.}\PY{n}{despine}\PY{p}{(}\PY{n}{left}\PY{o}{=}\PY{k+kc}{True}\PY{p}{,}\PY{n}{bottom}\PY{o}{=}\PY{k+kc}{True}\PY{p}{)}
          \PY{n}{plt}\PY{o}{.}\PY{n}{text}
\end{Verbatim}


    \begin{Verbatim}[commandchars=\\\{\}]
Loan\_Status      N    Y
Self\_Employed          
No             157  343
Yes             26   56

    \end{Verbatim}

\begin{Verbatim}[commandchars=\\\{\}]
{\color{outcolor}Out[{\color{outcolor}201}]:} <function matplotlib.pyplot.text>
\end{Verbatim}
            
    \begin{center}
    \adjustimage{max size={0.9\linewidth}{0.9\paperheight}}{output_87_2.png}
    \end{center}
    { \hspace*{\fill} \\}
    
    \begin{Verbatim}[commandchars=\\\{\}]
{\color{incolor}In [{\color{incolor}202}]:} \PY{n}{g}\PY{p}{,} \PY{n}{p}\PY{p}{,} \PY{n}{dof}\PY{p}{,} \PY{n}{expctd} \PY{o}{=} \PY{n}{stats}\PY{o}{.}\PY{n}{chi2\PYZus{}contingency}\PY{p}{(}\PY{n}{observed}\PY{o}{=} \PY{n}{self\PYZus{}employed}\PY{p}{)}
          
          \PY{n+nb}{print}\PY{p}{(}\PY{n}{f}\PY{l+s+s1}{\PYZsq{}}\PY{l+s+s1}{The chi square statistic is }\PY{l+s+si}{\PYZob{}g\PYZcb{}}\PY{l+s+s1}{\PYZsq{}}\PY{p}{)}
          \PY{n+nb}{print}\PY{p}{(}\PY{n}{f}\PY{l+s+s1}{\PYZsq{}}\PY{l+s+s1}{The P\PYZhy{}value is }\PY{l+s+si}{\PYZob{}p\PYZcb{}}\PY{l+s+s1}{\PYZsq{}}\PY{p}{)}
          \PY{n+nb}{print}\PY{p}{(}\PY{n}{f}\PY{l+s+s1}{\PYZsq{}}\PY{l+s+s1}{The degree of freedom is }\PY{l+s+si}{\PYZob{}dof\PYZcb{}}\PY{l+s+s1}{\PYZsq{}}\PY{p}{)}
          \PY{n+nb}{print}\PY{p}{(}\PY{n}{f}\PY{l+s+s1}{\PYZsq{}}\PY{l+s+s1}{The expected count table }\PY{l+s+s1}{\PYZsq{}}\PY{p}{)}
          \PY{n}{expctd}
          
          \PY{c+c1}{\PYZsh{} calculate the chi\PYZhy{}square\PYZhy{}statistic}
          \PY{n}{expctd} \PY{o}{=} \PY{n}{pd}\PY{o}{.}\PY{n}{DataFrame}\PY{p}{(}\PY{n}{expctd}\PY{p}{)}
          \PY{n+nb}{print}\PY{p}{(}\PY{n}{expctd}\PY{p}{)}
          \PY{n}{chi\PYZus{}squared\PYZus{}stat} \PY{o}{=} \PY{p}{(}\PY{p}{(}\PY{p}{(}\PY{p}{(}\PY{n}{self\PYZus{}employed}\PY{o}{\PYZhy{}}\PY{n}{expctd}\PY{p}{)}\PY{o}{*}\PY{o}{*}\PY{l+m+mi}{2}\PY{p}{)}\PY{o}{/}\PY{n}{expctd}\PY{p}{)}\PY{o}{.}\PY{n}{sum}\PY{p}{(}\PY{p}{)}\PY{o}{.}\PY{n}{sum}\PY{p}{(}\PY{p}{)}\PY{p}{)}
          
          \PY{c+c1}{\PYZsh{} print(chi\PYZus{}squared\PYZus{}stat)}
          \PY{n}{crit} \PY{o}{=} \PY{n}{stats}\PY{o}{.}\PY{n}{chi2}\PY{o}{.}\PY{n}{ppf}\PY{p}{(}\PY{n}{q} \PY{o}{=} \PY{l+m+mf}{0.95}\PY{p}{,} \PY{c+c1}{\PYZsh{} Find the critical value for 95\PYZpc{} confidence*}
                                \PY{n}{df} \PY{o}{=} \PY{l+m+mi}{1}\PY{p}{)}   \PY{c+c1}{\PYZsh{} *}
          \PY{n+nb}{print}\PY{p}{(}\PY{l+s+s2}{\PYZdq{}}\PY{l+s+s2}{Critical value}\PY{l+s+s2}{\PYZdq{}}\PY{p}{)}
          \PY{n+nb}{print}\PY{p}{(}\PY{n}{crit}\PY{p}{)}
\end{Verbatim}


    \begin{Verbatim}[commandchars=\\\{\}]
The chi square statistic is 0.005292770110001114
The P-value is 0.9420039242223718
The degree of freedom is 1
The expected count table 
            0           1
0  157.216495  342.783505
1   25.783505   56.216495
Critical value
3.8414588206941236

    \end{Verbatim}

    \texttt{There\ is\ nothing\ significant\ we\ can\ infer\ from\ Self\_Employed\ vs\ Loan\_Status\ plot}

\emph{Critical value \textgreater{} chi square statistic, p-value
\textgreater{} 0.05 and nearly 1}

\emph{Also No significance difference in the expected count table.}

    \begin{quote}
\textbf{Though Credit\_History dtypes is float64, it is more of a
categorical variable because it has a unique value of 0 (credit history
does not meets guidelines) and 1 (credit history does meets guideline)}
\end{quote}

    \begin{Verbatim}[commandchars=\\\{\}]
{\color{incolor}In [{\color{incolor}203}]:} \PY{c+c1}{\PYZsh{} credit\PYZus{}history vs Loan\PYZus{}Status}
          \PY{n}{credit\PYZus{}history} \PY{o}{=} \PY{n}{pd}\PY{o}{.}\PY{n}{crosstab}\PY{p}{(}\PY{n}{train}\PY{o}{.}\PY{n}{Credit\PYZus{}History}\PY{p}{,}\PY{n}{train}\PY{o}{.}\PY{n}{Loan\PYZus{}Status}\PY{p}{)}
          \PY{n+nb}{print}\PY{p}{(}\PY{n}{credit\PYZus{}history}\PY{p}{)}
          
          \PY{p}{(}\PY{n}{credit\PYZus{}history}\PY{o}{.}\PY{n}{divide}\PY{p}{(}\PY{n}{credit\PYZus{}history}\PY{o}{.}\PY{n}{sum}\PY{p}{(}\PY{n}{axis}\PY{o}{=}\PY{l+m+mi}{1}\PY{p}{)}
                         \PY{o}{.}\PY{n}{astype}\PY{p}{(}\PY{n+nb}{float}\PY{p}{)}\PY{p}{,} \PY{n}{axis}\PY{o}{=}\PY{l+m+mi}{0}\PY{p}{)}
                          \PY{o}{.}\PY{n}{plot}\PY{p}{(}\PY{n}{kind}\PY{o}{=}\PY{l+s+s2}{\PYZdq{}}\PY{l+s+s2}{bar}\PY{l+s+s2}{\PYZdq{}}\PY{p}{,} \PY{n}{stacked}\PY{o}{=}\PY{k+kc}{True}\PY{p}{,} \PY{n}{figsize}\PY{o}{=}\PY{p}{(}\PY{l+m+mi}{6}\PY{p}{,}\PY{l+m+mi}{4}\PY{p}{)}\PY{p}{)}\PY{p}{)}
          \PY{n}{plt}\PY{o}{.}\PY{n}{title}\PY{p}{(}\PY{l+s+s1}{\PYZsq{}}\PY{l+s+s1}{Relationship between Credit\PYZhy{}History and Loan Status}\PY{l+s+s1}{\PYZsq{}}\PY{p}{)}
          \PY{n}{plt}\PY{o}{.}\PY{n}{xlabel}\PY{p}{(}\PY{l+s+s1}{\PYZsq{}}\PY{l+s+s1}{Credit\PYZhy{}History}\PY{l+s+s1}{\PYZsq{}}\PY{p}{)}
          \PY{n}{plt}\PY{o}{.}\PY{n}{ylabel}\PY{p}{(}\PY{l+s+s1}{\PYZsq{}}\PY{l+s+s1}{Percentage}\PY{l+s+s1}{\PYZsq{}}\PY{p}{)}
          \PY{n}{sns}\PY{o}{.}\PY{n}{despine}\PY{p}{(}\PY{n}{left}\PY{o}{=}\PY{k+kc}{True}\PY{p}{,}\PY{n}{bottom}\PY{o}{=}\PY{k+kc}{True}\PY{p}{)}
          \PY{n}{plt}\PY{o}{.}\PY{n}{text}
\end{Verbatim}


    \begin{Verbatim}[commandchars=\\\{\}]
Loan\_Status      N    Y
Credit\_History         
0.0             82    7
1.0             97  378

    \end{Verbatim}

\begin{Verbatim}[commandchars=\\\{\}]
{\color{outcolor}Out[{\color{outcolor}203}]:} <function matplotlib.pyplot.text>
\end{Verbatim}
            
    \begin{center}
    \adjustimage{max size={0.9\linewidth}{0.9\paperheight}}{output_91_2.png}
    \end{center}
    { \hspace*{\fill} \\}
    
    \begin{Verbatim}[commandchars=\\\{\}]
{\color{incolor}In [{\color{incolor}204}]:} \PY{n}{g}\PY{p}{,} \PY{n}{p}\PY{p}{,} \PY{n}{dof}\PY{p}{,} \PY{n}{expctd} \PY{o}{=} \PY{n}{stats}\PY{o}{.}\PY{n}{chi2\PYZus{}contingency}\PY{p}{(}\PY{n}{observed}\PY{o}{=} \PY{n}{credit\PYZus{}history}\PY{p}{)}
          
          \PY{n+nb}{print}\PY{p}{(}\PY{n}{f}\PY{l+s+s1}{\PYZsq{}}\PY{l+s+s1}{The chi square statistic is }\PY{l+s+si}{\PYZob{}g\PYZcb{}}\PY{l+s+s1}{\PYZsq{}}\PY{p}{)}
          \PY{n+nb}{print}\PY{p}{(}\PY{n}{f}\PY{l+s+s1}{\PYZsq{}}\PY{l+s+s1}{The P\PYZhy{}value is }\PY{l+s+si}{\PYZob{}p\PYZcb{}}\PY{l+s+s1}{\PYZsq{}}\PY{p}{)}
          \PY{n+nb}{print}\PY{p}{(}\PY{n}{f}\PY{l+s+s1}{\PYZsq{}}\PY{l+s+s1}{The degree of freedom is }\PY{l+s+si}{\PYZob{}dof\PYZcb{}}\PY{l+s+s1}{\PYZsq{}}\PY{p}{)}
          \PY{n+nb}{print}\PY{p}{(}\PY{n}{f}\PY{l+s+s1}{\PYZsq{}}\PY{l+s+s1}{The expected count table }\PY{l+s+s1}{\PYZsq{}}\PY{p}{)}
          \PY{n}{expctd}
          
          \PY{c+c1}{\PYZsh{} calculate the chi\PYZhy{}square\PYZhy{}statistic}
          \PY{n}{expctd} \PY{o}{=} \PY{n}{pd}\PY{o}{.}\PY{n}{DataFrame}\PY{p}{(}\PY{n}{expctd}\PY{p}{)}
          \PY{n+nb}{print}\PY{p}{(}\PY{n}{expctd}\PY{p}{)}
          \PY{n}{chi\PYZus{}squared\PYZus{}stat} \PY{o}{=} \PY{p}{(}\PY{p}{(}\PY{p}{(}\PY{p}{(}\PY{n}{credit\PYZus{}history}\PY{o}{\PYZhy{}}\PY{n}{expctd}\PY{p}{)}\PY{o}{*}\PY{o}{*}\PY{l+m+mi}{2}\PY{p}{)}\PY{o}{/}\PY{n}{expctd}\PY{p}{)}\PY{o}{.}\PY{n}{sum}\PY{p}{(}\PY{p}{)}\PY{o}{.}\PY{n}{sum}\PY{p}{(}\PY{p}{)}\PY{p}{)}
          
          \PY{c+c1}{\PYZsh{} print(chi\PYZus{}squared\PYZus{}stat)}
          \PY{n}{crit} \PY{o}{=} \PY{n}{stats}\PY{o}{.}\PY{n}{chi2}\PY{o}{.}\PY{n}{ppf}\PY{p}{(}\PY{n}{q} \PY{o}{=} \PY{l+m+mf}{0.95}\PY{p}{,} \PY{c+c1}{\PYZsh{} Find the critical value for 95\PYZpc{} confidence*}
                                \PY{n}{df} \PY{o}{=} \PY{l+m+mi}{1}\PY{p}{)}   \PY{c+c1}{\PYZsh{} *}
          \PY{n+nb}{print}\PY{p}{(}\PY{l+s+s2}{\PYZdq{}}\PY{l+s+s2}{Critical value}\PY{l+s+s2}{\PYZdq{}}\PY{p}{)}
          \PY{n+nb}{print}\PY{p}{(}\PY{n}{crit}\PY{p}{)}
\end{Verbatim}


    \begin{Verbatim}[commandchars=\\\{\}]
The chi square statistic is 174.63729658142535
The P-value is 7.184759548750746e-40
The degree of freedom is 1
The expected count table 
            0           1
0   28.246454   60.753546
1  150.753546  324.246454
Critical value
3.8414588206941236

    \end{Verbatim}

    It seems people with credit history as 1 are more likely to get their
loans approved.

\textbf{Critical value of 3.84 is far less than 174.64 of chi square
statistic, We have a significant result at 5\% significance level since
the p-value is nearly 0.00 which shows a complete dependency between the
two variables.}

\emph{\texttt{We\ can\ conclude\ that\ Loan\ Status\ of\ an\ applicant\ is\ strongly\ associated\ to\ their\ credit\ history.}}

    \begin{Verbatim}[commandchars=\\\{\}]
{\color{incolor}In [{\color{incolor}205}]:} \PY{c+c1}{\PYZsh{} credit\PYZus{}history vs Loan\PYZus{}Status}
          \PY{n}{property\PYZus{}area} \PY{o}{=} \PY{n}{pd}\PY{o}{.}\PY{n}{crosstab}\PY{p}{(}\PY{n}{train}\PY{o}{.}\PY{n}{Property\PYZus{}Area}\PY{p}{,}\PY{n}{train}\PY{o}{.}\PY{n}{Loan\PYZus{}Status}\PY{p}{)}
          \PY{n+nb}{print}\PY{p}{(}\PY{n}{property\PYZus{}area}\PY{p}{)}
          
          \PY{p}{(}\PY{n}{property\PYZus{}area}\PY{o}{.}\PY{n}{divide}\PY{p}{(}\PY{n}{property\PYZus{}area}\PY{o}{.}\PY{n}{sum}\PY{p}{(}\PY{n}{axis}\PY{o}{=}\PY{l+m+mi}{1}\PY{p}{)}
                         \PY{o}{.}\PY{n}{astype}\PY{p}{(}\PY{n+nb}{float}\PY{p}{)}\PY{p}{,} \PY{n}{axis}\PY{o}{=}\PY{l+m+mi}{0}\PY{p}{)}
                          \PY{o}{.}\PY{n}{plot}\PY{p}{(}\PY{n}{kind}\PY{o}{=}\PY{l+s+s2}{\PYZdq{}}\PY{l+s+s2}{bar}\PY{l+s+s2}{\PYZdq{}}\PY{p}{,} \PY{n}{stacked}\PY{o}{=}\PY{k+kc}{True}\PY{p}{,} \PY{n}{figsize}\PY{o}{=}\PY{p}{(}\PY{l+m+mi}{14}\PY{p}{,}\PY{l+m+mi}{4}\PY{p}{)}\PY{p}{)}\PY{p}{)}
          \PY{n}{plt}\PY{o}{.}\PY{n}{title}\PY{p}{(}\PY{l+s+s1}{\PYZsq{}}\PY{l+s+s1}{Relationship between Property\PYZus{}Area and Loan Status}\PY{l+s+s1}{\PYZsq{}}\PY{p}{)}
          \PY{n}{plt}\PY{o}{.}\PY{n}{xlabel}\PY{p}{(}\PY{l+s+s1}{\PYZsq{}}\PY{l+s+s1}{Property\PYZus{}Area}\PY{l+s+s1}{\PYZsq{}}\PY{p}{)}
          \PY{n}{plt}\PY{o}{.}\PY{n}{ylabel}\PY{p}{(}\PY{l+s+s1}{\PYZsq{}}\PY{l+s+s1}{Percentage}\PY{l+s+s1}{\PYZsq{}}\PY{p}{)}
          \PY{n}{sns}\PY{o}{.}\PY{n}{despine}\PY{p}{(}\PY{n}{left}\PY{o}{=}\PY{k+kc}{True}\PY{p}{,}\PY{n}{bottom}\PY{o}{=}\PY{k+kc}{True}\PY{p}{)}
          \PY{n}{plt}\PY{o}{.}\PY{n}{text}
\end{Verbatim}


    \begin{Verbatim}[commandchars=\\\{\}]
Loan\_Status     N    Y
Property\_Area         
Rural          69  110
Semiurban      54  179
Urban          69  133

    \end{Verbatim}

\begin{Verbatim}[commandchars=\\\{\}]
{\color{outcolor}Out[{\color{outcolor}205}]:} <function matplotlib.pyplot.text>
\end{Verbatim}
            
    \begin{center}
    \adjustimage{max size={0.9\linewidth}{0.9\paperheight}}{output_94_2.png}
    \end{center}
    { \hspace*{\fill} \\}
    
    \begin{Verbatim}[commandchars=\\\{\}]
{\color{incolor}In [{\color{incolor}206}]:} \PY{n}{g}\PY{p}{,} \PY{n}{p}\PY{p}{,} \PY{n}{dof}\PY{p}{,} \PY{n}{expctd} \PY{o}{=} \PY{n}{stats}\PY{o}{.}\PY{n}{chi2\PYZus{}contingency}\PY{p}{(}\PY{n}{observed}\PY{o}{=} \PY{n}{property\PYZus{}area}\PY{p}{)}
          
          \PY{n+nb}{print}\PY{p}{(}\PY{n}{f}\PY{l+s+s1}{\PYZsq{}}\PY{l+s+s1}{The chi square statistic is }\PY{l+s+si}{\PYZob{}g\PYZcb{}}\PY{l+s+s1}{\PYZsq{}}\PY{p}{)}
          \PY{n+nb}{print}\PY{p}{(}\PY{n}{f}\PY{l+s+s1}{\PYZsq{}}\PY{l+s+s1}{The P\PYZhy{}value is }\PY{l+s+si}{\PYZob{}p\PYZcb{}}\PY{l+s+s1}{\PYZsq{}}\PY{p}{)}
          \PY{n+nb}{print}\PY{p}{(}\PY{n}{f}\PY{l+s+s1}{\PYZsq{}}\PY{l+s+s1}{The degree of freedom is }\PY{l+s+si}{\PYZob{}dof\PYZcb{}}\PY{l+s+s1}{\PYZsq{}}\PY{p}{)}
          \PY{n+nb}{print}\PY{p}{(}\PY{n}{f}\PY{l+s+s1}{\PYZsq{}}\PY{l+s+s1}{The expected count table }\PY{l+s+s1}{\PYZsq{}}\PY{p}{)}
          \PY{n}{expctd}
          
          \PY{c+c1}{\PYZsh{} calculate the chi\PYZhy{}square\PYZhy{}statistic}
          \PY{n}{expctd} \PY{o}{=} \PY{n}{pd}\PY{o}{.}\PY{n}{DataFrame}\PY{p}{(}\PY{n}{expctd}\PY{p}{)}
          \PY{n+nb}{print}\PY{p}{(}\PY{n}{expctd}\PY{p}{)}
          \PY{n}{chi\PYZus{}squared\PYZus{}stat} \PY{o}{=} \PY{p}{(}\PY{p}{(}\PY{p}{(}\PY{p}{(}\PY{n}{property\PYZus{}area}\PY{o}{\PYZhy{}}\PY{n}{expctd}\PY{p}{)}\PY{o}{*}\PY{o}{*}\PY{l+m+mi}{2}\PY{p}{)}\PY{o}{/}\PY{n}{expctd}\PY{p}{)}\PY{o}{.}\PY{n}{sum}\PY{p}{(}\PY{p}{)}\PY{o}{.}\PY{n}{sum}\PY{p}{(}\PY{p}{)}\PY{p}{)}
          
          \PY{c+c1}{\PYZsh{} print(chi\PYZus{}squared\PYZus{}stat)}
          \PY{n}{crit} \PY{o}{=} \PY{n}{stats}\PY{o}{.}\PY{n}{chi2}\PY{o}{.}\PY{n}{ppf}\PY{p}{(}\PY{n}{q} \PY{o}{=} \PY{l+m+mf}{0.95}\PY{p}{,} \PY{c+c1}{\PYZsh{} Find the critical value for 95\PYZpc{} confidence*}
                                \PY{n}{df} \PY{o}{=} \PY{l+m+mi}{2}\PY{p}{)}   \PY{c+c1}{\PYZsh{} *}
          \PY{n+nb}{print}\PY{p}{(}\PY{l+s+s2}{\PYZdq{}}\PY{l+s+s2}{Critical value}\PY{l+s+s2}{\PYZdq{}}\PY{p}{)}
          \PY{n+nb}{print}\PY{p}{(}\PY{n}{crit}\PY{p}{)}
\end{Verbatim}


    \begin{Verbatim}[commandchars=\\\{\}]
The chi square statistic is 12.297623130485675
The P-value is 0.0021360187811644937
The degree of freedom is 2
The expected count table 
           0           1
0  55.973941  123.026059
1  72.859935  160.140065
2  63.166124  138.833876
Critical value
5.99146454710798

    \end{Verbatim}

    \emph{It seems people living in semi urban area are more likely to get
their loans approved.}

\textbf{Critical value of 5.99 is less than 12.29 of chi square
statistic, We have a significant result at 5\% significance level since
the p-value is nearly 0.00 which shows a complete dependency between the
two variables.}

\emph{\texttt{We\ can\ conclude\ that\ Loan\ Status\ of\ an\ applicant\ is\ strongly\ associated\ to\ their\ property\ area\ and\ Proportion\ of\ loans\ getting\ approved\ in\ semiurban\ area\ is\ higher\ as\ compared\ to\ that\ in\ rural\ or\ urban\ areas.}}

    \subsubsection{Bivariate analysis of Numerical Independent variable and
the target
variable}\label{bivariate-analysis-of-numerical-independent-variable-and-the-target-variable}

    \texttt{Descriptive\ statistic\ of\ numerical\ variables}

    \begin{Verbatim}[commandchars=\\\{\}]
{\color{incolor}In [{\color{incolor}207}]:} \PY{c+c1}{\PYZsh{} obtain descriptive statistic of numerical variables}
          \PY{n}{train}\PY{o}{.}\PY{n}{describe}\PY{p}{(}\PY{n}{percentiles}\PY{o}{=}\PY{p}{[}\PY{o}{.}\PY{l+m+mi}{01}\PY{p}{,}\PY{o}{.}\PY{l+m+mi}{05}\PY{p}{,}\PY{o}{.}\PY{l+m+mi}{1}\PY{p}{,}\PY{o}{.}\PY{l+m+mi}{2}\PY{p}{,}\PY{o}{.}\PY{l+m+mi}{4}\PY{p}{,}\PY{o}{.}\PY{l+m+mi}{5}\PY{p}{,}\PY{o}{.}\PY{l+m+mi}{7}\PY{p}{,}\PY{o}{.}\PY{l+m+mi}{85}\PY{p}{,}\PY{o}{.}\PY{l+m+mi}{95}\PY{p}{,}\PY{o}{.}\PY{l+m+mi}{99}\PY{p}{]}\PY{p}{)}
\end{Verbatim}


\begin{Verbatim}[commandchars=\\\{\}]
{\color{outcolor}Out[{\color{outcolor}207}]:}        ApplicantIncome  CoapplicantIncome  LoanAmount  Loan\_Amount\_Term  \textbackslash{}
          count       614.000000         614.000000  592.000000         600.00000   
          mean       5403.459283        1621.245798  146.412162         342.00000   
          std        6109.041673        2926.248369   85.587325          65.12041   
          min         150.000000           0.000000    9.000000          12.00000   
          1\%         1025.000000           0.000000   30.000000          84.00000   
          5\%         1897.550000           0.000000   56.000000         180.00000   
          10\%        2216.100000           0.000000   71.000000         294.00000   
          20\%        2605.400000           0.000000   95.000000         360.00000   
          40\%        3406.800000           0.000000  116.000000         360.00000   
          50\%        3812.500000        1188.500000  128.000000         360.00000   
          70\%        5185.600000        2083.000000  158.000000         360.00000   
          85\%        7578.250000        3053.650000  192.000000         360.00000   
          95\%       14583.000000        4997.400000  297.800000         360.00000   
          99\%       32540.410000        8895.890000  496.360000         480.00000   
          max       81000.000000       41667.000000  700.000000         480.00000   
          
                 Credit\_History  
          count      564.000000  
          mean         0.842199  
          std          0.364878  
          min          0.000000  
          1\%           0.000000  
          5\%           0.000000  
          10\%          0.000000  
          20\%          1.000000  
          40\%          1.000000  
          50\%          1.000000  
          70\%          1.000000  
          85\%          1.000000  
          95\%          1.000000  
          99\%          1.000000  
          max          1.000000  
\end{Verbatim}
            
    \subsubsection{Analyzing the Applicant
income}\label{analyzing-the-applicant-income}

\emph{We will try to find the mean income of people for which the loan
has been approved} \textbf{VS} \emph{the mean income of people for which
the loan has not been approved.}

    \begin{Verbatim}[commandchars=\\\{\}]
{\color{incolor}In [{\color{incolor}208}]:} \PY{n}{ax}\PY{o}{=}\PY{n}{train}\PY{o}{.}\PY{n}{groupby}\PY{p}{(}\PY{l+s+s1}{\PYZsq{}}\PY{l+s+s1}{Loan\PYZus{}Status}\PY{l+s+s1}{\PYZsq{}}\PY{p}{)}\PY{p}{[}\PY{l+s+s1}{\PYZsq{}}\PY{l+s+s1}{ApplicantIncome}\PY{l+s+s1}{\PYZsq{}}\PY{p}{]}\PY{o}{.}\PY{n}{mean}\PY{p}{(}\PY{p}{)}
          \PY{n+nb}{print}\PY{p}{(}\PY{n}{ax}\PY{p}{)}
\end{Verbatim}


    \begin{Verbatim}[commandchars=\\\{\}]
Loan\_Status
N    5446.078125
Y    5384.068720
Name: ApplicantIncome, dtype: float64

    \end{Verbatim}

    \begin{Verbatim}[commandchars=\\\{\}]
{\color{incolor}In [{\color{incolor}210}]:} \PY{n}{fig}\PY{o}{=}\PY{n}{plt}\PY{o}{.}\PY{n}{figure}\PY{p}{(}\PY{n}{figsize}\PY{o}{=}\PY{p}{(}\PY{l+m+mi}{8}\PY{p}{,}\PY{l+m+mi}{4}\PY{p}{)}\PY{p}{)}
          \PY{n}{ax}\PY{o}{=}\PY{n}{train}\PY{o}{.}\PY{n}{groupby}\PY{p}{(}\PY{l+s+s1}{\PYZsq{}}\PY{l+s+s1}{Loan\PYZus{}Status}\PY{l+s+s1}{\PYZsq{}}\PY{p}{)}\PY{p}{[}\PY{l+s+s1}{\PYZsq{}}\PY{l+s+s1}{ApplicantIncome}\PY{l+s+s1}{\PYZsq{}}\PY{p}{]}\PY{o}{.}\PY{n}{mean}\PY{p}{(}\PY{p}{)}\PY{o}{.}\PY{n}{plot}\PY{p}{(}\PY{n}{kind}\PY{o}{=}\PY{l+s+s1}{\PYZsq{}}\PY{l+s+s1}{bar}\PY{l+s+s1}{\PYZsq{}}\PY{p}{)}
          \PY{n}{ax}\PY{o}{.}\PY{n}{set\PYZus{}title}\PY{p}{(}\PY{l+s+s1}{\PYZsq{}}\PY{l+s+s1}{Distribution of mean income of approved applicants vs non\PYZhy{}approved applicants}\PY{l+s+s1}{\PYZsq{}}\PY{p}{)}
          \PY{k}{for} \PY{n}{p} \PY{o+ow}{in} \PY{n}{ax}\PY{o}{.}\PY{n}{patches}\PY{p}{:}
              \PY{n}{ax}\PY{o}{.}\PY{n}{annotate}\PY{p}{(}\PY{l+s+s1}{\PYZsq{}}\PY{l+s+si}{\PYZob{}:.2f\PYZcb{}}\PY{l+s+s1}{\PYZsq{}}\PY{o}{.}\PY{n}{format}\PY{p}{(}\PY{n}{p}\PY{o}{.}\PY{n}{get\PYZus{}height}\PY{p}{(}\PY{p}{)}\PY{p}{)}\PY{p}{,}
                          \PY{p}{(}\PY{n}{p}\PY{o}{.}\PY{n}{get\PYZus{}x}\PY{p}{(}\PY{p}{)}\PY{o}{+}\PY{l+m+mf}{0.10}\PY{p}{,} \PY{n}{p}\PY{o}{.}\PY{n}{get\PYZus{}height}\PY{p}{(}\PY{p}{)}\PY{o}{+}\PY{l+m+mi}{50}\PY{p}{)}\PY{p}{)}
\end{Verbatim}


    \begin{center}
    \adjustimage{max size={0.9\linewidth}{0.9\paperheight}}{output_102_0.png}
    \end{center}
    { \hspace*{\fill} \\}
    
    \textbf{We don't see any significant difference in the mean income of
these two categories.}

These can make us reject our null hypotheses which is
\texttt{Applicants\ with\ high\ income\ should\ have\ more\ chances\ of\ loan\ approval.}

This implies that
\texttt{Applicant\ income\ has\ no\ association\ with\ his/her\ loan\ eligibilty}

\textbf{However let dive more into it.}

    \begin{Verbatim}[commandchars=\\\{\}]
{\color{incolor}In [{\color{incolor}211}]:} \PY{k+kn}{from} \PY{n+nn}{scipy}\PY{n+nn}{.}\PY{n+nn}{stats} \PY{k}{import} \PY{n}{norm}
          \PY{n}{sns}\PY{o}{.}\PY{n}{set}\PY{p}{(}\PY{n}{style}\PY{o}{=}\PY{l+s+s2}{\PYZdq{}}\PY{l+s+s2}{dark}\PY{l+s+s2}{\PYZdq{}}\PY{p}{)}
          \PY{n}{ax} \PY{o}{=} \PY{n}{sns}\PY{o}{.}\PY{n}{distplot}\PY{p}{(}\PY{n}{train}\PY{o}{.}\PY{n}{ApplicantIncome}\PY{p}{,} \PY{n}{fit}\PY{o}{=}\PY{n}{norm}\PY{p}{,} \PY{n}{kde}\PY{o}{=}\PY{k+kc}{True}\PY{p}{)}
          \PY{n}{sns}\PY{o}{.}\PY{n}{despine}\PY{p}{(}\PY{n}{left}\PY{o}{=}\PY{k+kc}{True}\PY{p}{,}\PY{n}{bottom}\PY{o}{=}\PY{k+kc}{True}\PY{p}{)}
\end{Verbatim}


    \begin{center}
    \adjustimage{max size={0.9\linewidth}{0.9\paperheight}}{output_104_0.png}
    \end{center}
    { \hspace*{\fill} \\}
    
    \begin{Verbatim}[commandchars=\\\{\}]
{\color{incolor}In [{\color{incolor}212}]:} \PY{n}{plt}\PY{o}{.}\PY{n}{figure}\PY{p}{(}\PY{n}{figsize}\PY{o}{=}\PY{p}{(}\PY{l+m+mi}{12}\PY{p}{,}\PY{l+m+mi}{6}\PY{p}{)}\PY{p}{)}
          \PY{n}{plt}\PY{o}{.}\PY{n}{subplot}\PY{p}{(}\PY{l+m+mi}{121}\PY{p}{)}
          \PY{n}{sns}\PY{o}{.}\PY{n}{distplot}\PY{p}{(}\PY{n}{train}\PY{p}{[}\PY{l+s+s1}{\PYZsq{}}\PY{l+s+s1}{ApplicantIncome}\PY{l+s+s1}{\PYZsq{}}\PY{p}{]}\PY{p}{)}\PY{p}{;}
          
          \PY{n}{plt}\PY{o}{.}\PY{n}{subplot}\PY{p}{(}\PY{l+m+mi}{122}\PY{p}{)}
          \PY{c+c1}{\PYZsh{} train[\PYZsq{}ApplicantIncome\PYZsq{}].plot.box(figsize=(16,5))}
          \PY{n}{a}\PY{o}{=}\PY{n}{sns}\PY{o}{.}\PY{n}{boxplot}\PY{p}{(}\PY{n}{y}\PY{o}{=}\PY{l+s+s1}{\PYZsq{}}\PY{l+s+s1}{ApplicantIncome}\PY{l+s+s1}{\PYZsq{}}\PY{p}{,}\PY{n}{data}\PY{o}{=}\PY{n}{train}\PY{p}{,}\PY{n}{notch}\PY{o}{=}\PY{k+kc}{False}\PY{p}{)}
          \PY{c+c1}{\PYZsh{} a = sns.swarmplot( y=\PYZsq{}ApplicantIncome\PYZsq{}, data=train, color=\PYZdq{}.2\PYZdq{})}
\end{Verbatim}


    \begin{center}
    \adjustimage{max size={0.9\linewidth}{0.9\paperheight}}{output_105_0.png}
    \end{center}
    { \hspace*{\fill} \\}
    
    \textbf{The above charts shows that the ApplicantIncome is not normally
distributed.}

\texttt{It\ can\ be\ inferred\ that\ most\ of\ the\ data\ in\ the\ distribution\ of\ applicant\ income\ is\ towards\ left\ which\ means\ it\ is\ not\ normally\ distributed.\ We\ will\ try\ to\ make\ it\ normal\ in\ later\ sections\ as\ algorithms\ works\ better\ if\ the\ data\ is\ normally\ distributed.}

\texttt{The\ boxplot\ confirms\ the\ presence\ of\ a\ lot\ of\ outliers/extreme\ values.\ This\ can\ be\ attributed\ to\ the\ income\ disparity\ in\ the\ society.\ Part\ of\ this\ can\ be\ driven\ by\ the\ fact\ that\ we\ are\ looking\ at\ people\ with\ different\ education\ levels.}

    \begin{Verbatim}[commandchars=\\\{\}]
{\color{incolor}In [{\color{incolor}213}]:} \PY{c+c1}{\PYZsh{} segregate Applicant Income by Education}
          \PY{n}{a}\PY{o}{=}\PY{n}{sns}\PY{o}{.}\PY{n}{boxplot}\PY{p}{(}\PY{n}{x}\PY{o}{=}\PY{l+s+s1}{\PYZsq{}}\PY{l+s+s1}{Education}\PY{l+s+s1}{\PYZsq{}}\PY{p}{,}\PY{n}{y}\PY{o}{=}\PY{l+s+s1}{\PYZsq{}}\PY{l+s+s1}{ApplicantIncome}\PY{l+s+s1}{\PYZsq{}}\PY{p}{,}\PY{n}{data}\PY{o}{=}\PY{n}{train}\PY{p}{,}\PY{n}{notch}\PY{o}{=}\PY{k+kc}{False}\PY{p}{)}
\end{Verbatim}


    \begin{center}
    \adjustimage{max size={0.9\linewidth}{0.9\paperheight}}{output_107_0.png}
    \end{center}
    { \hspace*{\fill} \\}
    
    \texttt{We\ can\ see\ that\ there\ are\ a\ higher\ number\ of\ graduates\ with\ very\ high\ incomes,\ which\ are\ appearing\ to\ be\ the\ outliers}.

\emph{So, let's make bins for the applicant income variable based on the
values in it and analyze the corresponding loan status for each bin.}

\texttt{We\ can\ decide\ the\ bins\ by\ looking\ into\ its\ descriptive\ statistic}

    \begin{Verbatim}[commandchars=\\\{\}]
{\color{incolor}In [{\color{incolor}214}]:} \PY{c+c1}{\PYZsh{} the describe statistic of ApplicantIncome to decide for the binning range}
          \PY{n}{train}\PY{o}{.}\PY{n}{ApplicantIncome}\PY{o}{.}\PY{n}{describe}\PY{p}{(}\PY{n}{percentiles}\PY{o}{=}\PY{p}{[}\PY{o}{.}\PY{l+m+mi}{01}\PY{p}{,}\PY{o}{.}\PY{l+m+mi}{05}\PY{p}{,}\PY{o}{.}\PY{l+m+mi}{1}\PY{p}{,}\PY{o}{.}\PY{l+m+mi}{25}\PY{p}{,}\PY{o}{.}\PY{l+m+mi}{4}\PY{p}{,}\PY{o}{.}\PY{l+m+mi}{5}\PY{p}{,}\PY{o}{.}\PY{l+m+mi}{75}\PY{p}{,}\PY{o}{.}\PY{l+m+mi}{85}\PY{p}{,}\PY{o}{.}\PY{l+m+mi}{95}\PY{p}{,}\PY{o}{.}\PY{l+m+mi}{99}\PY{p}{]}\PY{p}{)}
\end{Verbatim}


\begin{Verbatim}[commandchars=\\\{\}]
{\color{outcolor}Out[{\color{outcolor}214}]:} count      614.000000
          mean      5403.459283
          std       6109.041673
          min        150.000000
          1\%        1025.000000
          5\%        1897.550000
          10\%       2216.100000
          25\%       2877.500000
          40\%       3406.800000
          50\%       3812.500000
          75\%       5795.000000
          85\%       7578.250000
          95\%      14583.000000
          99\%      32540.410000
          max      81000.000000
          Name: ApplicantIncome, dtype: float64
\end{Verbatim}
            
    \texttt{I\ conclude\ choosing\ the\ following\ bins:}

\begin{longtable}[]{@{}ll@{}}
\toprule
Category & Income range\tabularnewline
\midrule
\endhead
Low & 0-2500\tabularnewline
Average & 2500-4000\tabularnewline
High & 4000-6000\tabularnewline
Very High & \textgreater{}6000\tabularnewline
\bottomrule
\end{longtable}

    \begin{Verbatim}[commandchars=\\\{\}]
{\color{incolor}In [{\color{incolor}215}]:} \PY{n}{bins}\PY{o}{=}\PY{p}{[}\PY{l+m+mi}{0}\PY{p}{,}\PY{l+m+mi}{2500}\PY{p}{,}\PY{l+m+mi}{4000}\PY{p}{,}\PY{l+m+mi}{6000}\PY{p}{,}\PY{l+m+mi}{81000}\PY{p}{]}
          \PY{n}{group}\PY{o}{=}\PY{p}{[}\PY{l+s+s1}{\PYZsq{}}\PY{l+s+s1}{Low}\PY{l+s+s1}{\PYZsq{}}\PY{p}{,}\PY{l+s+s1}{\PYZsq{}}\PY{l+s+s1}{Average}\PY{l+s+s1}{\PYZsq{}}\PY{p}{,}\PY{l+s+s1}{\PYZsq{}}\PY{l+s+s1}{High}\PY{l+s+s1}{\PYZsq{}}\PY{p}{,} \PY{l+s+s1}{\PYZsq{}}\PY{l+s+s1}{Very high}\PY{l+s+s1}{\PYZsq{}}\PY{p}{]}
          \PY{n}{train}\PY{p}{[}\PY{l+s+s1}{\PYZsq{}}\PY{l+s+s1}{ApplicantIncome\PYZus{}bin}\PY{l+s+s1}{\PYZsq{}}\PY{p}{]}\PY{o}{=}\PY{n}{pd}\PY{o}{.}\PY{n}{cut}\PY{p}{(}\PY{n}{train}\PY{p}{[}\PY{l+s+s1}{\PYZsq{}}\PY{l+s+s1}{ApplicantIncome}\PY{l+s+s1}{\PYZsq{}}\PY{p}{]}\PY{p}{,}\PY{n}{bins}\PY{p}{,}\PY{n}{labels}\PY{o}{=}\PY{n}{group}\PY{p}{)}
\end{Verbatim}


    \begin{Verbatim}[commandchars=\\\{\}]
{\color{incolor}In [{\color{incolor}217}]:} \PY{n}{dict\PYZus{}of\PYZus{}value\PYZus{}counts} \PY{o}{=} \PY{p}{\PYZob{}}\PY{n}{a}\PY{p}{:}\PY{n}{b}\PY{o}{.}\PY{n}{value\PYZus{}counts}\PY{p}{(}\PY{p}{)} \PY{k}{for} \PY{n}{a}\PY{p}{,}\PY{n}{b} \PY{o+ow}{in} \PY{n}{train}\PY{o}{.}\PY{n}{items}\PY{p}{(}\PY{p}{)}\PY{p}{\PYZcb{}}
          \PY{n}{fig}\PY{o}{=}\PY{n}{plt}\PY{o}{.}\PY{n}{figure}\PY{p}{(}\PY{n}{figsize}\PY{o}{=}\PY{p}{(}\PY{l+m+mi}{8}\PY{p}{,}\PY{l+m+mi}{6}\PY{p}{)}\PY{p}{)}
          \PY{n}{sns}\PY{o}{.}\PY{n}{set}\PY{p}{(}\PY{n}{style}\PY{o}{=}\PY{l+s+s2}{\PYZdq{}}\PY{l+s+s2}{dark}\PY{l+s+s2}{\PYZdq{}}\PY{p}{)}
          \PY{n}{ax}\PY{o}{=}\PY{n}{sns}\PY{o}{.}\PY{n}{countplot}\PY{p}{(}\PY{n}{x}\PY{o}{=}\PY{l+s+s1}{\PYZsq{}}\PY{l+s+s1}{ApplicantIncome\PYZus{}bin}\PY{l+s+s1}{\PYZsq{}}\PY{p}{,}\PY{n}{data}\PY{o}{=}\PY{n}{train}\PY{p}{,}
                       \PY{n}{order}\PY{o}{=}\PY{n+nb}{list}\PY{p}{(}\PY{n}{dict\PYZus{}of\PYZus{}value\PYZus{}counts}\PY{p}{[}\PY{l+s+s1}{\PYZsq{}}\PY{l+s+s1}{ApplicantIncome\PYZus{}bin}\PY{l+s+s1}{\PYZsq{}}\PY{p}{]}\PY{o}{.}\PY{n}{index}\PY{p}{)}\PY{p}{)}
          \PY{n}{ax}\PY{o}{.}\PY{n}{set\PYZus{}title}\PY{p}{(}\PY{l+s+s1}{\PYZsq{}}\PY{l+s+s1}{Distribution of various categories of Applicant Income}\PY{l+s+s1}{\PYZsq{}}\PY{p}{,} \PY{n}{fontsize}\PY{o}{=}\PY{l+m+mi}{15}\PY{p}{)}
          \PY{n}{sns}\PY{o}{.}\PY{n}{despine}\PY{p}{(}\PY{n}{left}\PY{o}{=}\PY{k+kc}{True}\PY{p}{,}\PY{n}{bottom}\PY{o}{=}\PY{k+kc}{True}\PY{p}{)}
          \PY{k}{for} \PY{n}{p} \PY{o+ow}{in} \PY{n}{ax}\PY{o}{.}\PY{n}{patches}\PY{p}{:}
              \PY{n}{ax}\PY{o}{.}\PY{n}{annotate}\PY{p}{(}\PY{l+s+s1}{\PYZsq{}}\PY{l+s+si}{\PYZob{}:.2f\PYZcb{}}\PY{l+s+s1}{\PYZpc{}}\PY{l+s+s1}{\PYZsq{}}\PY{o}{.}\PY{n}{format}\PY{p}{(}\PY{p}{(}\PY{n}{p}\PY{o}{.}\PY{n}{get\PYZus{}height}\PY{p}{(}\PY{p}{)}\PY{o}{/}\PY{n}{dict\PYZus{}of\PYZus{}value\PYZus{}counts}\PY{p}{[}\PY{l+s+s1}{\PYZsq{}}\PY{l+s+s1}{ApplicantIncome\PYZus{}bin}\PY{l+s+s1}{\PYZsq{}}\PY{p}{]}\PY{o}{.}\PY{n}{sum}\PY{p}{(}\PY{p}{)}\PY{p}{)}\PY{o}{*}\PY{l+m+mi}{100}\PY{p}{)}\PY{p}{,}
                          \PY{p}{(}\PY{n}{p}\PY{o}{.}\PY{n}{get\PYZus{}x}\PY{p}{(}\PY{p}{)}\PY{o}{+}\PY{l+m+mf}{0.30}\PY{p}{,} \PY{n}{p}\PY{o}{.}\PY{n}{get\PYZus{}height}\PY{p}{(}\PY{p}{)}\PY{o}{+}\PY{l+m+mi}{1}\PY{p}{)}\PY{p}{)}
\end{Verbatim}


    \begin{center}
    \adjustimage{max size={0.9\linewidth}{0.9\paperheight}}{output_112_0.png}
    \end{center}
    { \hspace*{\fill} \\}
    
    \begin{Verbatim}[commandchars=\\\{\}]
{\color{incolor}In [{\color{incolor}218}]:} \PY{n}{applicantIncome\PYZus{}bin} \PY{o}{=} \PY{n}{pd}\PY{o}{.}\PY{n}{crosstab}\PY{p}{(}\PY{n}{train}\PY{o}{.}\PY{n}{ApplicantIncome\PYZus{}bin}\PY{p}{,}\PY{n}{train}\PY{o}{.}\PY{n}{Loan\PYZus{}Status}\PY{p}{)}
          \PY{n+nb}{print}\PY{p}{(}\PY{n}{applicantIncome\PYZus{}bin}\PY{p}{)}
          
          \PY{p}{(}\PY{n}{applicantIncome\PYZus{}bin}\PY{o}{.}\PY{n}{divide}\PY{p}{(}\PY{n}{applicantIncome\PYZus{}bin}\PY{o}{.}\PY{n}{sum}\PY{p}{(}\PY{n}{axis}\PY{o}{=}\PY{l+m+mi}{1}\PY{p}{)}
                         \PY{o}{.}\PY{n}{astype}\PY{p}{(}\PY{n+nb}{float}\PY{p}{)}\PY{p}{,} \PY{n}{axis}\PY{o}{=}\PY{l+m+mi}{0}\PY{p}{)}
                          \PY{o}{.}\PY{n}{plot}\PY{p}{(}\PY{n}{kind}\PY{o}{=}\PY{l+s+s2}{\PYZdq{}}\PY{l+s+s2}{bar}\PY{l+s+s2}{\PYZdq{}}\PY{p}{,} \PY{n}{stacked}\PY{o}{=}\PY{k+kc}{True}\PY{p}{,} \PY{n}{figsize}\PY{o}{=}\PY{p}{(}\PY{l+m+mi}{14}\PY{p}{,}\PY{l+m+mi}{4}\PY{p}{)}\PY{p}{)}\PY{p}{)}
          \PY{n}{plt}\PY{o}{.}\PY{n}{title}\PY{p}{(}\PY{l+s+s1}{\PYZsq{}}\PY{l+s+s1}{Relationship between ApplicantIncome\PYZus{}bin and Loan Status}\PY{l+s+s1}{\PYZsq{}}\PY{p}{)}
          \PY{n}{plt}\PY{o}{.}\PY{n}{xlabel}\PY{p}{(}\PY{l+s+s1}{\PYZsq{}}\PY{l+s+s1}{ApplicantIncome\PYZus{}bin}\PY{l+s+s1}{\PYZsq{}}\PY{p}{)}
          \PY{n}{plt}\PY{o}{.}\PY{n}{ylabel}\PY{p}{(}\PY{l+s+s1}{\PYZsq{}}\PY{l+s+s1}{Percentage}\PY{l+s+s1}{\PYZsq{}}\PY{p}{)}
          \PY{n}{sns}\PY{o}{.}\PY{n}{despine}\PY{p}{(}\PY{n}{left}\PY{o}{=}\PY{k+kc}{True}\PY{p}{,}\PY{n}{bottom}\PY{o}{=}\PY{k+kc}{True}\PY{p}{)}
\end{Verbatim}


    \begin{Verbatim}[commandchars=\\\{\}]
Loan\_Status           N    Y
ApplicantIncome\_bin         
Low                  34   74
Average              67  159
High                 45   98
Very high            46   91

    \end{Verbatim}

    \begin{center}
    \adjustimage{max size={0.9\linewidth}{0.9\paperheight}}{output_113_1.png}
    \end{center}
    { \hspace*{\fill} \\}
    
    \texttt{It\ can\ \ now\ be\ inferred\ that\ Applicant\ income\ does\ not\ affect\ the\ chances\ of\ loan\ approval\ which\ contradicts\ our\ hypothesis\ in\ which\ we\ assumed\ that\ if\ the\ applicant\ income\ is\ high\ the\ chances\ of\ loan\ approval\ will\ also\ be\ high.}

    \begin{Verbatim}[commandchars=\\\{\}]
{\color{incolor}In [{\color{incolor}219}]:} \PY{n}{g}\PY{p}{,} \PY{n}{p}\PY{p}{,} \PY{n}{dof}\PY{p}{,} \PY{n}{expctd} \PY{o}{=} \PY{n}{stats}\PY{o}{.}\PY{n}{chi2\PYZus{}contingency}\PY{p}{(}\PY{n}{observed}\PY{o}{=} \PY{n}{applicantIncome\PYZus{}bin}\PY{p}{)}
          
          \PY{n+nb}{print}\PY{p}{(}\PY{n}{f}\PY{l+s+s1}{\PYZsq{}}\PY{l+s+s1}{The chi square statistic is }\PY{l+s+si}{\PYZob{}g\PYZcb{}}\PY{l+s+s1}{\PYZsq{}}\PY{p}{)}
          \PY{n+nb}{print}\PY{p}{(}\PY{n}{f}\PY{l+s+s1}{\PYZsq{}}\PY{l+s+s1}{The P\PYZhy{}value is }\PY{l+s+si}{\PYZob{}p\PYZcb{}}\PY{l+s+s1}{\PYZsq{}}\PY{p}{)}
          \PY{n+nb}{print}\PY{p}{(}\PY{n}{f}\PY{l+s+s1}{\PYZsq{}}\PY{l+s+s1}{The degree of freedom is }\PY{l+s+si}{\PYZob{}dof\PYZcb{}}\PY{l+s+s1}{\PYZsq{}}\PY{p}{)}
          \PY{n+nb}{print}\PY{p}{(}\PY{n}{f}\PY{l+s+s1}{\PYZsq{}}\PY{l+s+s1}{The expected count table }\PY{l+s+s1}{\PYZsq{}}\PY{p}{)}
          \PY{n}{expctd}
          
          \PY{c+c1}{\PYZsh{} calculate the chi\PYZhy{}square\PYZhy{}statistic}
          \PY{n}{expctd} \PY{o}{=} \PY{n}{pd}\PY{o}{.}\PY{n}{DataFrame}\PY{p}{(}\PY{n}{expctd}\PY{p}{)}
          \PY{n+nb}{print}\PY{p}{(}\PY{n}{expctd}\PY{p}{)}
          \PY{n}{chi\PYZus{}squared\PYZus{}stat} \PY{o}{=} \PY{p}{(}\PY{p}{(}\PY{p}{(}\PY{p}{(}\PY{n}{applicantIncome\PYZus{}bin}\PY{o}{\PYZhy{}}\PY{n}{expctd}\PY{p}{)}\PY{o}{*}\PY{o}{*}\PY{l+m+mi}{2}\PY{p}{)}\PY{o}{/}\PY{n}{expctd}\PY{p}{)}\PY{o}{.}\PY{n}{sum}\PY{p}{(}\PY{p}{)}\PY{o}{.}\PY{n}{sum}\PY{p}{(}\PY{p}{)}\PY{p}{)}
          
          \PY{c+c1}{\PYZsh{} print(chi\PYZus{}squared\PYZus{}stat)}
          \PY{n}{crit} \PY{o}{=} \PY{n}{stats}\PY{o}{.}\PY{n}{chi2}\PY{o}{.}\PY{n}{ppf}\PY{p}{(}\PY{n}{q} \PY{o}{=} \PY{l+m+mf}{0.95}\PY{p}{,} \PY{c+c1}{\PYZsh{} Find the critical value for 95\PYZpc{} confidence*}
                                \PY{n}{df} \PY{o}{=} \PY{l+m+mi}{3}\PY{p}{)}   \PY{c+c1}{\PYZsh{} *}
          \PY{n+nb}{print}\PY{p}{(}\PY{l+s+s2}{\PYZdq{}}\PY{l+s+s2}{Critical value}\PY{l+s+s2}{\PYZdq{}}\PY{p}{)}
          \PY{n+nb}{print}\PY{p}{(}\PY{n}{crit}\PY{p}{)}
\end{Verbatim}


    \begin{Verbatim}[commandchars=\\\{\}]
The chi square statistic is 0.6213577939863809
The P-value is 0.8915260581064367
The degree of freedom is 3
The expected count table 
           0           1
0  33.771987   74.228013
1  70.671010  155.328990
2  44.716612   98.283388
3  42.840391   94.159609
Critical value
7.8147279032511765

    \end{Verbatim}

    \texttt{The\ critical\ value\ of\ 7.81\ \textgreater{}\ 0.62\ of\ chi\ square\ statistic\ and\ We\ don\textquotesingle{}t\ have\ significant\ result\ at\ 5\%\ significance\ level\ since\ the\ p-value(0.89)\ is\ nearly\ 1.00\ which\ shows\ a\ complete\ independency\ between\ the\ two\ variables.}

\texttt{Comparing\ the\ expected\ count\ table\ with\ the\ observed\ count\ further\ reveal\ that\ the\ Loan\ Status\ of\ an\ applicant\ with\ average\ and\ very\ high\ income\ are\ partially\ related\ or\ associated\ because\ there\ is\ a\ significant\ change\ in\ their\ values}

    \subsubsection{Coapplicant Income
analyze}\label{coapplicant-income-analyze}

    \begin{Verbatim}[commandchars=\\\{\}]
{\color{incolor}In [{\color{incolor}220}]:} \PY{n}{plt}\PY{o}{.}\PY{n}{figure}\PY{p}{(}\PY{n}{figsize}\PY{o}{=}\PY{p}{(}\PY{l+m+mi}{12}\PY{p}{,}\PY{l+m+mi}{6}\PY{p}{)}\PY{p}{)}
          \PY{n}{plt}\PY{o}{.}\PY{n}{subplot}\PY{p}{(}\PY{l+m+mi}{121}\PY{p}{)}
          \PY{n}{sns}\PY{o}{.}\PY{n}{distplot}\PY{p}{(}\PY{n}{train}\PY{p}{[}\PY{l+s+s1}{\PYZsq{}}\PY{l+s+s1}{CoapplicantIncome}\PY{l+s+s1}{\PYZsq{}}\PY{p}{]}\PY{p}{)}\PY{p}{;}
          
          \PY{n}{plt}\PY{o}{.}\PY{n}{subplot}\PY{p}{(}\PY{l+m+mi}{122}\PY{p}{)}
          \PY{n}{a}\PY{o}{=}\PY{n}{sns}\PY{o}{.}\PY{n}{boxplot}\PY{p}{(}\PY{n}{y}\PY{o}{=}\PY{l+s+s1}{\PYZsq{}}\PY{l+s+s1}{CoapplicantIncome}\PY{l+s+s1}{\PYZsq{}}\PY{p}{,}\PY{n}{data}\PY{o}{=}\PY{n}{train}\PY{p}{,}\PY{n}{notch}\PY{o}{=}\PY{k+kc}{False}\PY{p}{)}
\end{Verbatim}


    \begin{center}
    \adjustimage{max size={0.9\linewidth}{0.9\paperheight}}{output_118_0.png}
    \end{center}
    { \hspace*{\fill} \\}
    
    \texttt{We\ see\ a\ similar\ distribution\ as\ that\ of\ the\ applicant\ income.\ Majority\ (more\ than\ 90\%)\ of\ coapplicant’s\ income\ ranges\ from\ 0\ to\ 5000.\ We\ also\ see\ a\ lot\ of\ outliers\ in\ the\ coapplicant\ income\ and\ it\ is\ not\ normally\ distributed.}

    \begin{Verbatim}[commandchars=\\\{\}]
{\color{incolor}In [{\color{incolor}221}]:} \PY{c+c1}{\PYZsh{} segregate Applicant Income by Education}
          \PY{n}{ax}\PY{o}{=}\PY{n}{train}\PY{o}{.}\PY{n}{groupby}\PY{p}{(}\PY{l+s+s1}{\PYZsq{}}\PY{l+s+s1}{Education}\PY{l+s+s1}{\PYZsq{}}\PY{p}{)}\PY{p}{[}\PY{l+s+s1}{\PYZsq{}}\PY{l+s+s1}{CoapplicantIncome}\PY{l+s+s1}{\PYZsq{}}\PY{p}{]}\PY{o}{.}\PY{n}{count}\PY{p}{(}\PY{p}{)}
          \PY{n+nb}{print}\PY{p}{(}\PY{n}{ax}\PY{p}{)}
          \PY{n}{a}\PY{o}{=}\PY{n}{sns}\PY{o}{.}\PY{n}{violinplot}\PY{p}{(}\PY{n}{x}\PY{o}{=}\PY{l+s+s1}{\PYZsq{}}\PY{l+s+s1}{Education}\PY{l+s+s1}{\PYZsq{}}\PY{p}{,}\PY{n}{y}\PY{o}{=}\PY{l+s+s1}{\PYZsq{}}\PY{l+s+s1}{CoapplicantIncome}\PY{l+s+s1}{\PYZsq{}}\PY{p}{,}\PY{n}{data}\PY{o}{=}\PY{n}{train}\PY{p}{,}\PY{n}{inner}\PY{o}{=}\PY{k+kc}{None}\PY{p}{)}
          \PY{n}{a} \PY{o}{=} \PY{n}{sns}\PY{o}{.}\PY{n}{swarmplot}\PY{p}{(}\PY{n}{x}\PY{o}{=}\PY{l+s+s1}{\PYZsq{}}\PY{l+s+s1}{Education}\PY{l+s+s1}{\PYZsq{}}\PY{p}{,} \PY{n}{y}\PY{o}{=}\PY{l+s+s2}{\PYZdq{}}\PY{l+s+s2}{CoapplicantIncome}\PY{l+s+s2}{\PYZdq{}}\PY{p}{,} \PY{n}{data}\PY{o}{=}\PY{n}{train}\PY{p}{,} \PY{n}{color}\PY{o}{=}\PY{l+s+s2}{\PYZdq{}}\PY{l+s+s2}{.2}\PY{l+s+s2}{\PYZdq{}}\PY{p}{)}
\end{Verbatim}


    \begin{Verbatim}[commandchars=\\\{\}]
Education
Graduate        480
Not Graduate    134
Name: CoapplicantIncome, dtype: int64

    \end{Verbatim}

    \begin{center}
    \adjustimage{max size={0.9\linewidth}{0.9\paperheight}}{output_120_1.png}
    \end{center}
    { \hspace*{\fill} \\}
    
    \emph{It is shown above that majority of the Coapplicant are graduate
and they have low income}

    \texttt{The\ descriptive\ statistic\ reveal\ that\ more\ than\ 40\%\ of\ Coapplicant\ has\ 0.0\ income\ and\ it\ was\ earlier\ observed\ that\ 273\ loan\ obervations\ have\ 0.0\ coapplicantIncome}

\texttt{This\ can\ implies\ that\ those\ loan\ requests\ with\ 0.0\ coapplicant\ income\ involved\ only\ one\ person\ OR\ the\ coapplicant\ has\ no\ income\ value\ OR\ he/she\ is\ unemployed}

    \begin{Verbatim}[commandchars=\\\{\}]
{\color{incolor}In [{\color{incolor}222}]:} \PY{n}{coapplicant}\PY{o}{=}\PY{p}{(}\PY{n}{train}\PY{o}{.}\PY{n}{CoapplicantIncome}
                       \PY{o}{.}\PY{n}{apply}\PY{p}{(}\PY{k}{lambda} \PY{n}{x}\PY{p}{:} \PY{l+s+s1}{\PYZsq{}}\PY{l+s+s1}{No coapplicant or Coapplicant has no income}\PY{l+s+s1}{\PYZsq{}} \PY{k}{if} \PY{n}{x}\PY{o}{==}\PY{l+m+mi}{0} \PY{k}{else} \PY{l+s+s1}{\PYZsq{}}\PY{l+s+s1}{Coapplicant exist and has an income}\PY{l+s+s1}{\PYZsq{}}\PY{p}{)}
                       \PY{o}{.}\PY{n}{value\PYZus{}counts}\PY{p}{(}\PY{p}{)}
                       \PY{o}{.}\PY{n}{to\PYZus{}frame}\PY{p}{(}\PY{n}{name}\PY{o}{=}\PY{l+s+s1}{\PYZsq{}}\PY{l+s+s1}{count}\PY{l+s+s1}{\PYZsq{}}\PY{p}{)}\PY{p}{)}
          \PY{n}{coapplicant}
\end{Verbatim}


\begin{Verbatim}[commandchars=\\\{\}]
{\color{outcolor}Out[{\color{outcolor}222}]:}                                              count
          Coapplicant exist and has an income            341
          No coapplicant or Coapplicant has no income    273
\end{Verbatim}
            
    \begin{Verbatim}[commandchars=\\\{\}]
{\color{incolor}In [{\color{incolor}223}]:} \PY{c+c1}{\PYZsh{} sns.set(style=\PYZdq{}white\PYZdq{})}
          \PY{n}{fig}\PY{o}{=}\PY{n}{plt}\PY{o}{.}\PY{n}{figure}\PY{p}{(}\PY{n}{figsize}\PY{o}{=}\PY{p}{(}\PY{l+m+mi}{10}\PY{p}{,}\PY{l+m+mi}{6}\PY{p}{)}\PY{p}{)}
          \PY{n}{ax}\PY{o}{=} \PY{n}{sns}\PY{o}{.}\PY{n}{barplot}\PY{p}{(}\PY{n}{x}\PY{o}{=}\PY{n}{coapplicant}\PY{o}{.}\PY{n}{index}\PY{p}{,}\PY{n}{y}\PY{o}{=}\PY{l+s+s1}{\PYZsq{}}\PY{l+s+s1}{count}\PY{l+s+s1}{\PYZsq{}}\PY{p}{,}
                      \PY{n}{data}\PY{o}{=}\PY{n}{coapplicant}\PY{p}{,}
                       \PY{n}{order}\PY{o}{=}\PY{n+nb}{list}\PY{p}{(}\PY{n}{coapplicant}\PY{o}{.}\PY{n}{index}\PY{p}{)}\PY{p}{)}
          \PY{n}{ax}\PY{o}{.}\PY{n}{set\PYZus{}title}\PY{p}{(}\PY{l+s+s1}{\PYZsq{}}\PY{l+s+s1}{Distribution of various categories of CoApplicant Income}\PY{l+s+s1}{\PYZsq{}}\PY{p}{,} \PY{n}{fontsize}\PY{o}{=}\PY{l+m+mi}{15}\PY{p}{)}
          \PY{n}{sns}\PY{o}{.}\PY{n}{despine}\PY{p}{(}\PY{n}{left}\PY{o}{=}\PY{k+kc}{True}\PY{p}{,}\PY{n}{bottom}\PY{o}{=}\PY{k+kc}{True}\PY{p}{)}
          \PY{k}{for} \PY{n}{p} \PY{o+ow}{in} \PY{n}{ax}\PY{o}{.}\PY{n}{patches}\PY{p}{:}
              \PY{n}{ax}\PY{o}{.}\PY{n}{annotate}\PY{p}{(}\PY{l+s+s1}{\PYZsq{}}\PY{l+s+si}{\PYZob{}:.2f\PYZcb{}}\PY{l+s+s1}{\PYZpc{}}\PY{l+s+s1}{\PYZsq{}}\PY{o}{.}\PY{n}{format}\PY{p}{(}\PY{p}{(}\PY{n}{p}\PY{o}{.}\PY{n}{get\PYZus{}height}\PY{p}{(}\PY{p}{)}\PY{o}{/}\PY{n}{dict\PYZus{}of\PYZus{}value\PYZus{}counts}\PY{p}{[}\PY{l+s+s1}{\PYZsq{}}\PY{l+s+s1}{CoapplicantIncome}\PY{l+s+s1}{\PYZsq{}}\PY{p}{]}\PY{o}{.}\PY{n}{sum}\PY{p}{(}\PY{p}{)}\PY{p}{)}\PY{o}{*}\PY{l+m+mi}{100}\PY{p}{)}\PY{p}{,}
                          \PY{p}{(}\PY{n}{p}\PY{o}{.}\PY{n}{get\PYZus{}x}\PY{p}{(}\PY{p}{)}\PY{o}{+}\PY{l+m+mf}{0.30}\PY{p}{,} \PY{n}{p}\PY{o}{.}\PY{n}{get\PYZus{}height}\PY{p}{(}\PY{p}{)}\PY{o}{+}\PY{l+m+mi}{1}\PY{p}{)}\PY{p}{)}
\end{Verbatim}


    \begin{center}
    \adjustimage{max size={0.9\linewidth}{0.9\paperheight}}{output_124_0.png}
    \end{center}
    { \hspace*{\fill} \\}
    
    \textbf{It can be observed that 44.46\% of the loan applicant has no
coapplicant}

    \begin{Verbatim}[commandchars=\\\{\}]
{\color{incolor}In [{\color{incolor}224}]:} \PY{c+c1}{\PYZsh{} The average coaplicant income excluding those of zero }
          \PY{n}{a} \PY{o}{=} \PY{p}{(}\PY{n}{train}\PY{o}{.}\PY{n}{CoapplicantIncome}
           \PY{o}{.}\PY{n}{map}\PY{p}{(}\PY{k}{lambda} \PY{n}{x}\PY{p}{:}\PY{n}{x} \PY{k}{if} \PY{n}{x}\PY{o}{\PYZgt{}}\PY{l+m+mi}{0} \PY{k}{else} \PY{n}{np}\PY{o}{.}\PY{n}{nan}\PY{p}{)}
           \PY{o}{.}\PY{n}{mean}\PY{p}{(}\PY{n}{skipna}\PY{o}{=}\PY{k+kc}{True}\PY{p}{)}\PY{p}{)}
          
          \PY{n+nb}{print}\PY{p}{(}\PY{p}{(}\PY{l+s+s1}{\PYZsq{}\PYZsq{}\PYZsq{}}\PY{l+s+s1}{The true mean of coapplicant income is }\PY{l+s+si}{\PYZob{}:.2f\PYZcb{}}\PY{l+s+s1}{ aganist }\PY{l+s+si}{\PYZob{}:.2f\PYZcb{}}\PY{l+s+s1}{ that is }
          \PY{l+s+s1}{obtained if those zeros of no coapplicant or unemployed coapplicant are included in computation}\PY{l+s+s1}{\PYZsq{}\PYZsq{}\PYZsq{}}  
              
          \PY{o}{.}\PY{n}{format}\PY{p}{(}\PY{n}{a}\PY{p}{,}\PY{n}{train}\PY{o}{.}\PY{n}{CoapplicantIncome}\PY{o}{.}\PY{n}{mean}\PY{p}{(}\PY{p}{)} \PY{p}{)}\PY{p}{)}\PY{p}{)}
\end{Verbatim}


    \begin{Verbatim}[commandchars=\\\{\}]
The true mean of coapplicant income is 2919.19 aganist 1621.25 that is 
obtained if those zeros of no coapplicant or unemployed coapplicant are included in computation

    \end{Verbatim}

    \begin{Verbatim}[commandchars=\\\{\}]
{\color{incolor}In [{\color{incolor}225}]:} \PY{n}{employed\PYZus{}coapplicant}\PY{o}{=}\PY{p}{(}\PY{n}{train}\PY{o}{.}\PY{n}{CoapplicantIncome}
           \PY{o}{.}\PY{n}{map}\PY{p}{(}\PY{k}{lambda} \PY{n}{x}\PY{p}{:}\PY{n}{x} \PY{k}{if} \PY{n}{x}\PY{o}{\PYZgt{}}\PY{l+m+mi}{0} \PY{k}{else} \PY{n}{np}\PY{o}{.}\PY{n}{nan}\PY{p}{)}
           \PY{o}{.}\PY{n}{agg}\PY{p}{(}\PY{p}{[}\PY{n}{np}\PY{o}{.}\PY{n}{min}\PY{p}{,}\PY{n}{np}\PY{o}{.}\PY{n}{mean}\PY{p}{,}\PY{n}{np}\PY{o}{.}\PY{n}{max}\PY{p}{,}\PY{n}{np}\PY{o}{.}\PY{n}{std}\PY{p}{]}\PY{p}{)}
            \PY{o}{.}\PY{n}{to\PYZus{}frame}\PY{p}{(}\PY{p}{)}\PY{p}{)}
          
          \PY{n}{employed\PYZus{}coapplicant}\PY{o}{.}\PY{n}{index} \PY{o}{=} \PY{p}{[}\PY{l+s+s1}{\PYZsq{}}\PY{l+s+s1}{min\PYZus{}income of the employed}\PY{l+s+s1}{\PYZsq{}}\PY{p}{,}
                                        \PY{l+s+s1}{\PYZsq{}}\PY{l+s+s1}{mean\PYZus{}income of the employed}\PY{l+s+s1}{\PYZsq{}}\PY{p}{,}\PY{l+s+s1}{\PYZsq{}}\PY{l+s+s1}{max\PYZus{}income of the employed}\PY{l+s+s1}{\PYZsq{}}\PY{p}{,}\PY{l+s+s1}{\PYZsq{}}\PY{l+s+s1}{std\PYZus{}value of employed}\PY{l+s+s1}{\PYZsq{}}\PY{p}{]}
          \PY{n}{employed\PYZus{}coapplicant}
\end{Verbatim}


\begin{Verbatim}[commandchars=\\\{\}]
{\color{outcolor}Out[{\color{outcolor}225}]:}                              CoapplicantIncome
          min\_income of the employed           16.120001
          mean\_income of the employed        2919.193314
          max\_income of the employed        41667.000000
          std\_value of employed              3411.503263
\end{Verbatim}
            
    \texttt{Describe\ statistic\ of\ CoApplicantIncome\ to\ decide\ for\ the\ binning\ range}

    \begin{Verbatim}[commandchars=\\\{\}]
{\color{incolor}In [{\color{incolor}226}]:} \PY{c+c1}{\PYZsh{} the describe statistic of CoApplicantIncome to decide for the binning range}
          \PY{n}{train}\PY{o}{.}\PY{n}{CoapplicantIncome}\PY{o}{.}\PY{n}{describe}\PY{p}{(}\PY{n}{percentiles}\PY{o}{=}\PY{p}{[}\PY{o}{.}\PY{l+m+mi}{01}\PY{p}{,}\PY{o}{.}\PY{l+m+mi}{05}\PY{p}{,}\PY{o}{.}\PY{l+m+mi}{1}\PY{p}{,}\PY{o}{.}\PY{l+m+mi}{25}\PY{p}{,}\PY{o}{.}\PY{l+m+mi}{4}\PY{p}{,}\PY{o}{.}\PY{l+m+mi}{5}\PY{p}{,}\PY{o}{.}\PY{l+m+mi}{75}\PY{p}{,}\PY{o}{.}\PY{l+m+mi}{85}\PY{p}{,}\PY{o}{.}\PY{l+m+mi}{95}\PY{p}{,}\PY{o}{.}\PY{l+m+mi}{99}\PY{p}{]}\PY{p}{)}
\end{Verbatim}


\begin{Verbatim}[commandchars=\\\{\}]
{\color{outcolor}Out[{\color{outcolor}226}]:} count      614.000000
          mean      1621.245798
          std       2926.248369
          min          0.000000
          1\%           0.000000
          5\%           0.000000
          10\%          0.000000
          25\%          0.000000
          40\%          0.000000
          50\%       1188.500000
          75\%       2297.250000
          85\%       3053.650000
          95\%       4997.400000
          99\%       8895.890000
          max      41667.000000
          Name: CoapplicantIncome, dtype: float64
\end{Verbatim}
            
    \texttt{I\ concluded\ from\ the\ two\ descriptive\ statistic\ above\ to\ choose\ the\ following\ bins:}

\begin{longtable}[]{@{}ll@{}}
\toprule
Category & Income range\tabularnewline
\midrule
\endhead
Low & 0-1000\tabularnewline
Average & 1000-3000\tabularnewline
High & \textgreater{}3000\tabularnewline
\bottomrule
\end{longtable}

    \begin{Verbatim}[commandchars=\\\{\}]
{\color{incolor}In [{\color{incolor}227}]:} \PY{n}{bins}\PY{o}{=}\PY{p}{[}\PY{l+m+mi}{0}\PY{p}{,}\PY{l+m+mi}{1000}\PY{p}{,}\PY{l+m+mi}{3000}\PY{p}{,}\PY{l+m+mi}{42000}\PY{p}{]}
          \PY{n}{group}\PY{o}{=}\PY{p}{[}\PY{l+s+s1}{\PYZsq{}}\PY{l+s+s1}{Low}\PY{l+s+s1}{\PYZsq{}}\PY{p}{,}\PY{l+s+s1}{\PYZsq{}}\PY{l+s+s1}{Average}\PY{l+s+s1}{\PYZsq{}}\PY{p}{,}\PY{l+s+s1}{\PYZsq{}}\PY{l+s+s1}{High}\PY{l+s+s1}{\PYZsq{}}\PY{p}{]}
          \PY{n}{train}\PY{p}{[}\PY{l+s+s1}{\PYZsq{}}\PY{l+s+s1}{Coapplicant\PYZus{}Income\PYZus{}bin}\PY{l+s+s1}{\PYZsq{}}\PY{p}{]}\PY{o}{=}\PY{n}{pd}\PY{o}{.}\PY{n}{cut}\PY{p}{(}\PY{n}{train}\PY{p}{[}\PY{l+s+s1}{\PYZsq{}}\PY{l+s+s1}{CoapplicantIncome}\PY{l+s+s1}{\PYZsq{}}\PY{p}{]}\PY{p}{,}\PY{n}{bins}\PY{p}{,}\PY{n}{labels}\PY{o}{=}\PY{n}{group}\PY{p}{)}
\end{Verbatim}


    \begin{Verbatim}[commandchars=\\\{\}]
{\color{incolor}In [{\color{incolor}228}]:} \PY{n}{coapplicantIncome\PYZus{}bin} \PY{o}{=} \PY{n}{pd}\PY{o}{.}\PY{n}{crosstab}\PY{p}{(}\PY{n}{train}\PY{o}{.}\PY{n}{Coapplicant\PYZus{}Income\PYZus{}bin}\PY{p}{,}\PY{n}{train}\PY{o}{.}\PY{n}{Loan\PYZus{}Status}\PY{p}{)}
          \PY{n+nb}{print}\PY{p}{(}\PY{n}{coapplicantIncome\PYZus{}bin}\PY{p}{)}
          
          \PY{p}{(}\PY{n}{coapplicantIncome\PYZus{}bin}\PY{o}{.}\PY{n}{divide}\PY{p}{(}\PY{n}{coapplicantIncome\PYZus{}bin}\PY{o}{.}\PY{n}{sum}\PY{p}{(}\PY{n}{axis}\PY{o}{=}\PY{l+m+mi}{1}\PY{p}{)}
                         \PY{o}{.}\PY{n}{astype}\PY{p}{(}\PY{n+nb}{float}\PY{p}{)}\PY{p}{,} \PY{n}{axis}\PY{o}{=}\PY{l+m+mi}{0}\PY{p}{)}
                          \PY{o}{.}\PY{n}{plot}\PY{p}{(}\PY{n}{kind}\PY{o}{=}\PY{l+s+s2}{\PYZdq{}}\PY{l+s+s2}{bar}\PY{l+s+s2}{\PYZdq{}}\PY{p}{,} \PY{n}{stacked}\PY{o}{=}\PY{k+kc}{True}\PY{p}{,} \PY{n}{figsize}\PY{o}{=}\PY{p}{(}\PY{l+m+mi}{10}\PY{p}{,}\PY{l+m+mi}{4}\PY{p}{)}\PY{p}{)}\PY{p}{)}
          \PY{n}{plt}\PY{o}{.}\PY{n}{title}\PY{p}{(}\PY{l+s+s1}{\PYZsq{}}\PY{l+s+s1}{Relationship between CoapplicantIncome\PYZus{}bin and Loan Status}\PY{l+s+s1}{\PYZsq{}}\PY{p}{)}
          \PY{n}{plt}\PY{o}{.}\PY{n}{xlabel}\PY{p}{(}\PY{l+s+s1}{\PYZsq{}}\PY{l+s+s1}{CoapplicantIncome\PYZus{}bin}\PY{l+s+s1}{\PYZsq{}}\PY{p}{)}
          \PY{n}{plt}\PY{o}{.}\PY{n}{ylabel}\PY{p}{(}\PY{l+s+s1}{\PYZsq{}}\PY{l+s+s1}{Percentage}\PY{l+s+s1}{\PYZsq{}}\PY{p}{)}
          \PY{n}{sns}\PY{o}{.}\PY{n}{despine}\PY{p}{(}\PY{n}{left}\PY{o}{=}\PY{k+kc}{True}\PY{p}{,}\PY{n}{bottom}\PY{o}{=}\PY{k+kc}{True}\PY{p}{)}
\end{Verbatim}


    \begin{Verbatim}[commandchars=\\\{\}]
Loan\_Status              N    Y
Coapplicant\_Income\_bin         
Low                      3   19
Average                 61  161
High                    32   65

    \end{Verbatim}

    \begin{center}
    \adjustimage{max size={0.9\linewidth}{0.9\paperheight}}{output_132_1.png}
    \end{center}
    { \hspace*{\fill} \\}
    
    \begin{Verbatim}[commandchars=\\\{\}]
{\color{incolor}In [{\color{incolor}229}]:} \PY{n}{g}\PY{p}{,} \PY{n}{p}\PY{p}{,} \PY{n}{dof}\PY{p}{,} \PY{n}{expctd} \PY{o}{=} \PY{n}{stats}\PY{o}{.}\PY{n}{chi2\PYZus{}contingency}\PY{p}{(}\PY{n}{observed}\PY{o}{=} \PY{n}{coapplicantIncome\PYZus{}bin}\PY{p}{)}
          
          \PY{n+nb}{print}\PY{p}{(}\PY{n}{f}\PY{l+s+s1}{\PYZsq{}}\PY{l+s+s1}{The chi square statistic is }\PY{l+s+si}{\PYZob{}g\PYZcb{}}\PY{l+s+s1}{\PYZsq{}}\PY{p}{)}
          \PY{n+nb}{print}\PY{p}{(}\PY{n}{f}\PY{l+s+s1}{\PYZsq{}}\PY{l+s+s1}{The P\PYZhy{}value is }\PY{l+s+si}{\PYZob{}p\PYZcb{}}\PY{l+s+s1}{\PYZsq{}}\PY{p}{)}
          \PY{n+nb}{print}\PY{p}{(}\PY{n}{f}\PY{l+s+s1}{\PYZsq{}}\PY{l+s+s1}{The degree of freedom is }\PY{l+s+si}{\PYZob{}dof\PYZcb{}}\PY{l+s+s1}{\PYZsq{}}\PY{p}{)}
          \PY{n+nb}{print}\PY{p}{(}\PY{n}{f}\PY{l+s+s1}{\PYZsq{}}\PY{l+s+s1}{The expected count table }\PY{l+s+s1}{\PYZsq{}}\PY{p}{)}
          \PY{n}{expctd}
          
          \PY{c+c1}{\PYZsh{} calculate the chi\PYZhy{}square\PYZhy{}statistic}
          \PY{n}{expctd} \PY{o}{=} \PY{n}{pd}\PY{o}{.}\PY{n}{DataFrame}\PY{p}{(}\PY{n}{expctd}\PY{p}{)}
          \PY{n+nb}{print}\PY{p}{(}\PY{n}{expctd}\PY{p}{)}
          \PY{n}{chi\PYZus{}squared\PYZus{}stat} \PY{o}{=} \PY{p}{(}\PY{p}{(}\PY{p}{(}\PY{p}{(}\PY{n}{coapplicantIncome\PYZus{}bin}\PY{o}{\PYZhy{}}\PY{n}{expctd}\PY{p}{)}\PY{o}{*}\PY{o}{*}\PY{l+m+mi}{2}\PY{p}{)}\PY{o}{/}\PY{n}{expctd}\PY{p}{)}\PY{o}{.}\PY{n}{sum}\PY{p}{(}\PY{p}{)}\PY{o}{.}\PY{n}{sum}\PY{p}{(}\PY{p}{)}\PY{p}{)}
          
          \PY{c+c1}{\PYZsh{} print(chi\PYZus{}squared\PYZus{}stat)}
          \PY{n}{crit} \PY{o}{=} \PY{n}{stats}\PY{o}{.}\PY{n}{chi2}\PY{o}{.}\PY{n}{ppf}\PY{p}{(}\PY{n}{q} \PY{o}{=} \PY{l+m+mf}{0.95}\PY{p}{,} \PY{c+c1}{\PYZsh{} Find the critical value for 95\PYZpc{} confidence*}
                                \PY{n}{df} \PY{o}{=} \PY{l+m+mi}{2}\PY{p}{)}   \PY{c+c1}{\PYZsh{} *}
          \PY{n+nb}{print}\PY{p}{(}\PY{l+s+s2}{\PYZdq{}}\PY{l+s+s2}{Critical value}\PY{l+s+s2}{\PYZdq{}}\PY{p}{)}
          \PY{n+nb}{print}\PY{p}{(}\PY{n}{crit}\PY{p}{)}
\end{Verbatim}


    \begin{Verbatim}[commandchars=\\\{\}]
The chi square statistic is 3.4640082184540937
The P-value is 0.17692946843764604
The degree of freedom is 2
The expected count table 
           0           1
0   6.193548   15.806452
1  62.498534  159.501466
2  27.307918   69.692082
Critical value
5.99146454710798

    \end{Verbatim}

    \texttt{It\ shows\ that\ if\ coapplicant’s\ income\ is\ less\ the\ chances\ of\ loan\ approval\ are\ high.\ But\ this\ does\ not\ look\ right.\ The\ possible\ reason\ behind\ this\ may\ be\ that\ most\ of\ the\ applicants\ don’t\ have\ any\ coapplicant\ so\ the\ coapplicant\ income\ for\ such\ applicants\ is\ 0\ and\ hence\ the\ loan\ approval\ is\ not\ dependent\ on\ it.}

\texttt{The\ expected\ count\ table\ show\ that\ there\ is\ significant\ difference\ in\ coapplicant\ with\ low\ and\ high\ income\ when\ compared\ with\ their\ observed\ value.\ This\ tell\ us\ that\ these\ categories\ seems\ to\ be\ associated\ with\ the\ loan\ eligibilty\ status.}

    \texttt{On\ the\ premise\ that\ nearly\ half\ of\ the\ loan\ applicant\ has\ no\ coapplicant\ and\ that\ there\ seems\ to\ be\ an\ association\ between\ loan\ status\ an\ coapplicant\ with\ low\ or\ high\ income.}
\textbf{\texttt{Let\ assume\ to\ \ combine\ the\ applicant’s\ and\ coapplicant’s\ income\ to\ visualize\ the\ combined\ effect\ of\ income\ on\ loan\ approval.}}

    \begin{Verbatim}[commandchars=\\\{\}]
{\color{incolor}In [{\color{incolor}230}]:} \PY{n}{train}\PY{p}{[}\PY{l+s+s1}{\PYZsq{}}\PY{l+s+s1}{Total\PYZus{}Income}\PY{l+s+s1}{\PYZsq{}}\PY{p}{]}\PY{o}{=}\PY{n}{train}\PY{p}{[}\PY{l+s+s1}{\PYZsq{}}\PY{l+s+s1}{ApplicantIncome}\PY{l+s+s1}{\PYZsq{}}\PY{p}{]} \PY{o}{+} \PY{n}{train}\PY{p}{[}\PY{l+s+s1}{\PYZsq{}}\PY{l+s+s1}{CoapplicantIncome}\PY{l+s+s1}{\PYZsq{}}\PY{p}{]}
\end{Verbatim}


    \emph{\texttt{Describe\ statistic\ of\ numerical\ variables\ after\ Total\_Income\ has\ been\ computed\ by\ adding\ Applicant\ Income\ and\ Coapplicant\ Income}}

    \begin{Verbatim}[commandchars=\\\{\}]
{\color{incolor}In [{\color{incolor}231}]:} \PY{n}{train}\PY{o}{.}\PY{n}{describe}\PY{p}{(}\PY{n}{percentiles}\PY{o}{=}\PY{p}{[}\PY{o}{.}\PY{l+m+mi}{01}\PY{p}{,}\PY{o}{.}\PY{l+m+mi}{05}\PY{p}{,}\PY{o}{.}\PY{l+m+mi}{1}\PY{p}{,}\PY{o}{.}\PY{l+m+mi}{25}\PY{p}{,}\PY{o}{.}\PY{l+m+mi}{4}\PY{p}{,}\PY{o}{.}\PY{l+m+mi}{5}\PY{p}{,}\PY{o}{.}\PY{l+m+mi}{75}\PY{p}{,}\PY{o}{.}\PY{l+m+mi}{85}\PY{p}{,}\PY{o}{.}\PY{l+m+mi}{95}\PY{p}{,}\PY{o}{.}\PY{l+m+mi}{99}\PY{p}{]}\PY{p}{)}
\end{Verbatim}


\begin{Verbatim}[commandchars=\\\{\}]
{\color{outcolor}Out[{\color{outcolor}231}]:}        ApplicantIncome  CoapplicantIncome  LoanAmount  Loan\_Amount\_Term  \textbackslash{}
          count       614.000000         614.000000  592.000000         600.00000   
          mean       5403.459283        1621.245798  146.412162         342.00000   
          std        6109.041673        2926.248369   85.587325          65.12041   
          min         150.000000           0.000000    9.000000          12.00000   
          1\%         1025.000000           0.000000   30.000000          84.00000   
          5\%         1897.550000           0.000000   56.000000         180.00000   
          10\%        2216.100000           0.000000   71.000000         294.00000   
          25\%        2877.500000           0.000000  100.000000         360.00000   
          40\%        3406.800000           0.000000  116.000000         360.00000   
          50\%        3812.500000        1188.500000  128.000000         360.00000   
          75\%        5795.000000        2297.250000  168.000000         360.00000   
          85\%        7578.250000        3053.650000  192.000000         360.00000   
          95\%       14583.000000        4997.400000  297.800000         360.00000   
          99\%       32540.410000        8895.890000  496.360000         480.00000   
          max       81000.000000       41667.000000  700.000000         480.00000   
          
                 Credit\_History  Total\_Income  
          count      564.000000    614.000000  
          mean         0.842199   7024.705081  
          std          0.364878   6458.663872  
          min          0.000000   1442.000000  
          1\%           0.000000   2141.510000  
          5\%           0.000000   2800.500000  
          10\%          0.000000   3245.800000  
          25\%          1.000000   4166.000000  
          40\%          1.000000   4807.400000  
          50\%          1.000000   5416.500000  
          75\%          1.000000   7521.750000  
          85\%          1.000000   9767.000000  
          95\%          1.000000  16165.500000  
          99\%          1.000000  37453.020000  
          max          1.000000  81000.000000  
\end{Verbatim}
            
    \begin{Verbatim}[commandchars=\\\{\}]
{\color{incolor}In [{\color{incolor}232}]:} \PY{n}{ax} \PY{o}{=} \PY{n}{sns}\PY{o}{.}\PY{n}{distplot}\PY{p}{(}\PY{n}{train}\PY{o}{.}\PY{n}{Total\PYZus{}Income}\PY{p}{,} \PY{n}{fit}\PY{o}{=}\PY{n}{norm}\PY{p}{,} \PY{n}{kde}\PY{o}{=}\PY{k+kc}{True}\PY{p}{)}
\end{Verbatim}


    \begin{center}
    \adjustimage{max size={0.9\linewidth}{0.9\paperheight}}{output_139_0.png}
    \end{center}
    { \hspace*{\fill} \\}
    
    \texttt{Though\ The\ Total\_Income\ is\ not\ normally\ distributed.\ Its\ value\ still\ give\ a\ reasonable\ evalution\ than\ that\ of\ Applicant\_Income\ or\ Coapplicant\_Income\ alone}

    \begin{Verbatim}[commandchars=\\\{\}]
{\color{incolor}In [{\color{incolor}233}]:} \PY{n}{plt}\PY{o}{.}\PY{n}{figure}\PY{p}{(}\PY{n}{figsize}\PY{o}{=}\PY{p}{(}\PY{l+m+mi}{12}\PY{p}{,}\PY{l+m+mi}{6}\PY{p}{)}\PY{p}{)}
          \PY{n}{plt}\PY{o}{.}\PY{n}{subplot}\PY{p}{(}\PY{l+m+mi}{121}\PY{p}{)}
          \PY{n}{sns}\PY{o}{.}\PY{n}{distplot}\PY{p}{(}\PY{n}{train}\PY{p}{[}\PY{l+s+s1}{\PYZsq{}}\PY{l+s+s1}{Total\PYZus{}Income}\PY{l+s+s1}{\PYZsq{}}\PY{p}{]}\PY{p}{)}\PY{p}{;}
          
          \PY{n}{plt}\PY{o}{.}\PY{n}{subplot}\PY{p}{(}\PY{l+m+mi}{122}\PY{p}{)}
          \PY{c+c1}{\PYZsh{} train[\PYZsq{}ApplicantIncome\PYZsq{}].plot.box(figsize=(16,5))}
          \PY{n}{a}\PY{o}{=}\PY{n}{sns}\PY{o}{.}\PY{n}{boxplot}\PY{p}{(}\PY{n}{y}\PY{o}{=}\PY{l+s+s1}{\PYZsq{}}\PY{l+s+s1}{Total\PYZus{}Income}\PY{l+s+s1}{\PYZsq{}}\PY{p}{,}\PY{n}{data}\PY{o}{=}\PY{n}{train}\PY{p}{,}\PY{n}{notch}\PY{o}{=}\PY{k+kc}{False}\PY{p}{)}
          \PY{c+c1}{\PYZsh{} a = sns.swarmplot( y=\PYZsq{}ApplicantIncome\PYZsq{}, data=train, color=\PYZdq{}.2\PYZdq{})}
\end{Verbatim}


    \begin{center}
    \adjustimage{max size={0.9\linewidth}{0.9\paperheight}}{output_141_0.png}
    \end{center}
    { \hspace*{\fill} \\}
    
    \texttt{I\ conclude\ choosing\ the\ following\ bins:}

\begin{longtable}[]{@{}ll@{}}
\toprule
Category & Income range\tabularnewline
\midrule
\endhead
Low & 0-2500\tabularnewline
Average & 2500-4000\tabularnewline
High & 4000-6000\tabularnewline
Very High & \textgreater{}6000\tabularnewline
\bottomrule
\end{longtable}

    \begin{Verbatim}[commandchars=\\\{\}]
{\color{incolor}In [{\color{incolor}234}]:} \PY{n}{bins}\PY{o}{=}\PY{p}{[}\PY{l+m+mi}{0}\PY{p}{,}\PY{l+m+mi}{2500}\PY{p}{,}\PY{l+m+mi}{4000}\PY{p}{,}\PY{l+m+mi}{6000}\PY{p}{,}\PY{l+m+mi}{81000}\PY{p}{]}
          \PY{n}{group}\PY{o}{=}\PY{p}{[}\PY{l+s+s1}{\PYZsq{}}\PY{l+s+s1}{Low}\PY{l+s+s1}{\PYZsq{}}\PY{p}{,}\PY{l+s+s1}{\PYZsq{}}\PY{l+s+s1}{Average}\PY{l+s+s1}{\PYZsq{}}\PY{p}{,}\PY{l+s+s1}{\PYZsq{}}\PY{l+s+s1}{High}\PY{l+s+s1}{\PYZsq{}}\PY{p}{,} \PY{l+s+s1}{\PYZsq{}}\PY{l+s+s1}{Very high}\PY{l+s+s1}{\PYZsq{}}\PY{p}{]}
          \PY{n}{train}\PY{p}{[}\PY{l+s+s1}{\PYZsq{}}\PY{l+s+s1}{Total\PYZus{}Income\PYZus{}bin}\PY{l+s+s1}{\PYZsq{}}\PY{p}{]}\PY{o}{=}\PY{n}{pd}\PY{o}{.}\PY{n}{cut}\PY{p}{(}\PY{n}{train}\PY{p}{[}\PY{l+s+s1}{\PYZsq{}}\PY{l+s+s1}{Total\PYZus{}Income}\PY{l+s+s1}{\PYZsq{}}\PY{p}{]}\PY{p}{,}\PY{n}{bins}\PY{p}{,}\PY{n}{labels}\PY{o}{=}\PY{n}{group}\PY{p}{)}
\end{Verbatim}


    \begin{Verbatim}[commandchars=\\\{\}]
{\color{incolor}In [{\color{incolor}235}]:} \PY{n}{total\PYZus{}Income\PYZus{}bin} \PY{o}{=} \PY{n}{pd}\PY{o}{.}\PY{n}{crosstab}\PY{p}{(}\PY{n}{train}\PY{o}{.}\PY{n}{Total\PYZus{}Income\PYZus{}bin}\PY{p}{,}\PY{n}{train}\PY{o}{.}\PY{n}{Loan\PYZus{}Status}\PY{p}{)}
          \PY{n+nb}{print}\PY{p}{(}\PY{n}{total\PYZus{}Income\PYZus{}bin}\PY{p}{)}
          
          \PY{p}{(}\PY{n}{total\PYZus{}Income\PYZus{}bin}\PY{o}{.}\PY{n}{divide}\PY{p}{(}\PY{n}{total\PYZus{}Income\PYZus{}bin}\PY{o}{.}\PY{n}{sum}\PY{p}{(}\PY{n}{axis}\PY{o}{=}\PY{l+m+mi}{1}\PY{p}{)}
                         \PY{o}{.}\PY{n}{astype}\PY{p}{(}\PY{n+nb}{float}\PY{p}{)}\PY{p}{,} \PY{n}{axis}\PY{o}{=}\PY{l+m+mi}{0}\PY{p}{)}
                          \PY{o}{.}\PY{n}{plot}\PY{p}{(}\PY{n}{kind}\PY{o}{=}\PY{l+s+s2}{\PYZdq{}}\PY{l+s+s2}{bar}\PY{l+s+s2}{\PYZdq{}}\PY{p}{,} \PY{n}{stacked}\PY{o}{=}\PY{k+kc}{True}\PY{p}{,} \PY{n}{figsize}\PY{o}{=}\PY{p}{(}\PY{l+m+mi}{12}\PY{p}{,}\PY{l+m+mi}{4}\PY{p}{)}\PY{p}{)}\PY{p}{)}
          \PY{n}{plt}\PY{o}{.}\PY{n}{title}\PY{p}{(}\PY{l+s+s1}{\PYZsq{}}\PY{l+s+s1}{Relationship between total\PYZus{}Income\PYZus{}bin and Loan Status}\PY{l+s+s1}{\PYZsq{}}\PY{p}{)}
          \PY{n}{plt}\PY{o}{.}\PY{n}{xlabel}\PY{p}{(}\PY{l+s+s1}{\PYZsq{}}\PY{l+s+s1}{total\PYZus{}Income\PYZus{}bin}\PY{l+s+s1}{\PYZsq{}}\PY{p}{)}
          \PY{n}{plt}\PY{o}{.}\PY{n}{ylabel}\PY{p}{(}\PY{l+s+s1}{\PYZsq{}}\PY{l+s+s1}{Percentage}\PY{l+s+s1}{\PYZsq{}}\PY{p}{)}
          \PY{n}{sns}\PY{o}{.}\PY{n}{despine}\PY{p}{(}\PY{n}{left}\PY{o}{=}\PY{k+kc}{True}\PY{p}{,}\PY{n}{bottom}\PY{o}{=}\PY{k+kc}{True}\PY{p}{)}
\end{Verbatim}


    \begin{Verbatim}[commandchars=\\\{\}]
Loan\_Status        N    Y
Total\_Income\_bin         
Low               14   10
Average           32   87
High              65  159
Very high         81  166

    \end{Verbatim}

    \begin{center}
    \adjustimage{max size={0.9\linewidth}{0.9\paperheight}}{output_144_1.png}
    \end{center}
    { \hspace*{\fill} \\}
    
    \texttt{We\ can\ see\ that\ Proportion\ of\ loans\ getting\ approved\ for\ applicants\ having\ low\ Total\_Income\ is\ very\ less\ as\ compared\ to\ that\ of\ applicants\ with\ Average,\ High\ and\ Very\ High\ Income.}

    \begin{Verbatim}[commandchars=\\\{\}]
{\color{incolor}In [{\color{incolor}236}]:} \PY{n}{g}\PY{p}{,} \PY{n}{p}\PY{p}{,} \PY{n}{dof}\PY{p}{,} \PY{n}{expctd} \PY{o}{=} \PY{n}{stats}\PY{o}{.}\PY{n}{chi2\PYZus{}contingency}\PY{p}{(}\PY{n}{observed}\PY{o}{=} \PY{n}{total\PYZus{}Income\PYZus{}bin}\PY{p}{)}
          
          \PY{n+nb}{print}\PY{p}{(}\PY{n}{f}\PY{l+s+s1}{\PYZsq{}}\PY{l+s+s1}{The chi square statistic is }\PY{l+s+si}{\PYZob{}g\PYZcb{}}\PY{l+s+s1}{\PYZsq{}}\PY{p}{)}
          \PY{n+nb}{print}\PY{p}{(}\PY{n}{f}\PY{l+s+s1}{\PYZsq{}}\PY{l+s+s1}{The P\PYZhy{}value is }\PY{l+s+si}{\PYZob{}p\PYZcb{}}\PY{l+s+s1}{\PYZsq{}}\PY{p}{)}
          \PY{n+nb}{print}\PY{p}{(}\PY{n}{f}\PY{l+s+s1}{\PYZsq{}}\PY{l+s+s1}{The degree of freedom is }\PY{l+s+si}{\PYZob{}dof\PYZcb{}}\PY{l+s+s1}{\PYZsq{}}\PY{p}{)}
          \PY{n+nb}{print}\PY{p}{(}\PY{n}{f}\PY{l+s+s1}{\PYZsq{}}\PY{l+s+s1}{The expected count table }\PY{l+s+s1}{\PYZsq{}}\PY{p}{)}
          \PY{n}{expctd}
          
          \PY{c+c1}{\PYZsh{} calculate the chi\PYZhy{}square\PYZhy{}statistic}
          \PY{n}{expctd} \PY{o}{=} \PY{n}{pd}\PY{o}{.}\PY{n}{DataFrame}\PY{p}{(}\PY{n}{expctd}\PY{p}{)}
          \PY{n+nb}{print}\PY{p}{(}\PY{n}{expctd}\PY{p}{)}
          \PY{n}{chi\PYZus{}squared\PYZus{}stat} \PY{o}{=} \PY{p}{(}\PY{p}{(}\PY{p}{(}\PY{p}{(}\PY{n}{total\PYZus{}Income\PYZus{}bin}\PY{o}{\PYZhy{}}\PY{n}{expctd}\PY{p}{)}\PY{o}{*}\PY{o}{*}\PY{l+m+mi}{2}\PY{p}{)}\PY{o}{/}\PY{n}{expctd}\PY{p}{)}\PY{o}{.}\PY{n}{sum}\PY{p}{(}\PY{p}{)}\PY{o}{.}\PY{n}{sum}\PY{p}{(}\PY{p}{)}\PY{p}{)}
          
          \PY{c+c1}{\PYZsh{} print(chi\PYZus{}squared\PYZus{}stat)}
          \PY{n}{crit} \PY{o}{=} \PY{n}{stats}\PY{o}{.}\PY{n}{chi2}\PY{o}{.}\PY{n}{ppf}\PY{p}{(}\PY{n}{q} \PY{o}{=} \PY{l+m+mf}{0.95}\PY{p}{,} \PY{c+c1}{\PYZsh{} Find the critical value for 95\PYZpc{} confidence*}
                                \PY{n}{df} \PY{o}{=} \PY{l+m+mi}{3}\PY{p}{)}   \PY{c+c1}{\PYZsh{} *}
          \PY{n+nb}{print}\PY{p}{(}\PY{l+s+s2}{\PYZdq{}}\PY{l+s+s2}{Critical value}\PY{l+s+s2}{\PYZdq{}}\PY{p}{)}
          \PY{n+nb}{print}\PY{p}{(}\PY{n}{crit}\PY{p}{)}
\end{Verbatim}


    \begin{Verbatim}[commandchars=\\\{\}]
The chi square statistic is 10.036201335258713
The P-value is 0.018260904127128785
The degree of freedom is 3
The expected count table 
           0           1
0   7.504886   16.495114
1  37.211726   81.788274
2  70.045603  153.954397
3  77.237785  169.762215
Critical value
7.8147279032511765

    \end{Verbatim}

    \textbf{\emph{The above statistic and chart show that the Total\_Income
of applicant is associated with the loan eligibility status}}

    \subsubsection{Considering Loan Amount
variable}\label{considering-loan-amount-variable}

    \begin{Verbatim}[commandchars=\\\{\}]
{\color{incolor}In [{\color{incolor}238}]:} \PY{n}{plt}\PY{o}{.}\PY{n}{figure}\PY{p}{(}\PY{n}{figsize}\PY{o}{=}\PY{p}{(}\PY{l+m+mi}{16}\PY{p}{,}\PY{l+m+mi}{5}\PY{p}{)}\PY{p}{)}
          \PY{n}{plt}\PY{o}{.}\PY{n}{subplot}\PY{p}{(}\PY{l+m+mi}{121}\PY{p}{)}
          \PY{n}{train}\PY{o}{=}\PY{n}{train}\PY{o}{.}\PY{n}{dropna}\PY{p}{(}\PY{p}{)}
          \PY{n}{sns}\PY{o}{.}\PY{n}{distplot}\PY{p}{(}\PY{n}{train}\PY{p}{[}\PY{l+s+s1}{\PYZsq{}}\PY{l+s+s1}{LoanAmount}\PY{l+s+s1}{\PYZsq{}}\PY{p}{]}\PY{p}{)}\PY{p}{;}
          
          \PY{n}{plt}\PY{o}{.}\PY{n}{subplot}\PY{p}{(}\PY{l+m+mi}{122}\PY{p}{)}
          \PY{n}{a}\PY{o}{=}\PY{n}{sns}\PY{o}{.}\PY{n}{boxplot}\PY{p}{(}\PY{n}{y}\PY{o}{=}\PY{l+s+s1}{\PYZsq{}}\PY{l+s+s1}{LoanAmount}\PY{l+s+s1}{\PYZsq{}}\PY{p}{,}\PY{n}{data}\PY{o}{=}\PY{n}{train}\PY{p}{,}\PY{n}{notch}\PY{o}{=}\PY{k+kc}{False}\PY{p}{)}
          \PY{c+c1}{\PYZsh{} train[\PYZsq{}LoanAmount\PYZsq{}].plot.box(figsize=(16,5))}
\end{Verbatim}


    \begin{center}
    \adjustimage{max size={0.9\linewidth}{0.9\paperheight}}{output_149_0.png}
    \end{center}
    { \hspace*{\fill} \\}
    
    \textbf{We see a lot of outliers in this variable and the distribution
is fairly normal.}

    \begin{Verbatim}[commandchars=\\\{\}]
{\color{incolor}In [{\color{incolor}239}]:} \PY{n}{bins}\PY{o}{=}\PY{p}{[}\PY{l+m+mi}{0}\PY{p}{,}\PY{l+m+mi}{100}\PY{p}{,}\PY{l+m+mi}{200}\PY{p}{,}\PY{l+m+mi}{700}\PY{p}{]}
          \PY{n}{group}\PY{o}{=}\PY{p}{[}\PY{l+s+s1}{\PYZsq{}}\PY{l+s+s1}{Low}\PY{l+s+s1}{\PYZsq{}}\PY{p}{,}\PY{l+s+s1}{\PYZsq{}}\PY{l+s+s1}{Average}\PY{l+s+s1}{\PYZsq{}}\PY{p}{,}\PY{l+s+s1}{\PYZsq{}}\PY{l+s+s1}{High}\PY{l+s+s1}{\PYZsq{}}\PY{p}{]}
          \PY{n}{train}\PY{p}{[}\PY{l+s+s1}{\PYZsq{}}\PY{l+s+s1}{LoanAmount\PYZus{}bin}\PY{l+s+s1}{\PYZsq{}}\PY{p}{]}\PY{o}{=}\PY{n}{pd}\PY{o}{.}\PY{n}{cut}\PY{p}{(}\PY{n}{train}\PY{p}{[}\PY{l+s+s1}{\PYZsq{}}\PY{l+s+s1}{LoanAmount}\PY{l+s+s1}{\PYZsq{}}\PY{p}{]}\PY{p}{,}\PY{n}{bins}\PY{p}{,}\PY{n}{labels}\PY{o}{=}\PY{n}{group}\PY{p}{)}
\end{Verbatim}


    \begin{Verbatim}[commandchars=\\\{\}]
{\color{incolor}In [{\color{incolor}240}]:} \PY{n}{loanAmount\PYZus{}bin} \PY{o}{=} \PY{n}{pd}\PY{o}{.}\PY{n}{crosstab}\PY{p}{(}\PY{n}{train}\PY{o}{.}\PY{n}{LoanAmount\PYZus{}bin}\PY{p}{,}\PY{n}{train}\PY{o}{.}\PY{n}{Loan\PYZus{}Status}\PY{p}{)}
          \PY{n+nb}{print}\PY{p}{(}\PY{n}{total\PYZus{}Income\PYZus{}bin}\PY{p}{)}
          
          \PY{p}{(}\PY{n}{loanAmount\PYZus{}bin}\PY{o}{.}\PY{n}{divide}\PY{p}{(}\PY{n}{loanAmount\PYZus{}bin}\PY{o}{.}\PY{n}{sum}\PY{p}{(}\PY{n}{axis}\PY{o}{=}\PY{l+m+mi}{1}\PY{p}{)}
                         \PY{o}{.}\PY{n}{astype}\PY{p}{(}\PY{n+nb}{float}\PY{p}{)}\PY{p}{,} \PY{n}{axis}\PY{o}{=}\PY{l+m+mi}{0}\PY{p}{)}
                          \PY{o}{.}\PY{n}{plot}\PY{p}{(}\PY{n}{kind}\PY{o}{=}\PY{l+s+s2}{\PYZdq{}}\PY{l+s+s2}{bar}\PY{l+s+s2}{\PYZdq{}}\PY{p}{,} \PY{n}{stacked}\PY{o}{=}\PY{k+kc}{True}\PY{p}{,} \PY{n}{figsize}\PY{o}{=}\PY{p}{(}\PY{l+m+mi}{10}\PY{p}{,}\PY{l+m+mi}{4}\PY{p}{)}\PY{p}{)}\PY{p}{)}
          \PY{n}{plt}\PY{o}{.}\PY{n}{title}\PY{p}{(}\PY{l+s+s1}{\PYZsq{}}\PY{l+s+s1}{Relationship between loanAmount\PYZus{}bin and Loan Status}\PY{l+s+s1}{\PYZsq{}}\PY{p}{)}
          \PY{n}{plt}\PY{o}{.}\PY{n}{xlabel}\PY{p}{(}\PY{l+s+s1}{\PYZsq{}}\PY{l+s+s1}{loanAmount\PYZus{}bin}\PY{l+s+s1}{\PYZsq{}}\PY{p}{)}
          \PY{n}{plt}\PY{o}{.}\PY{n}{ylabel}\PY{p}{(}\PY{l+s+s1}{\PYZsq{}}\PY{l+s+s1}{Percentage}\PY{l+s+s1}{\PYZsq{}}\PY{p}{)}
          \PY{n}{sns}\PY{o}{.}\PY{n}{despine}\PY{p}{(}\PY{n}{left}\PY{o}{=}\PY{k+kc}{True}\PY{p}{,}\PY{n}{bottom}\PY{o}{=}\PY{k+kc}{True}\PY{p}{)}
\end{Verbatim}


    \begin{Verbatim}[commandchars=\\\{\}]
Loan\_Status        N    Y
Total\_Income\_bin         
Low               14   10
Average           32   87
High              65  159
Very high         81  166

    \end{Verbatim}

    \begin{center}
    \adjustimage{max size={0.9\linewidth}{0.9\paperheight}}{output_152_1.png}
    \end{center}
    { \hspace*{\fill} \\}
    
    \texttt{It\ can\ be\ seen\ that\ the\ proportion\ of\ approved\ loans\ is\ higher\ for\ Low\ and\ Average\ Loan\ Amount\ as\ compared\ to\ that\ of\ High\ Loan\ Amount\ which\ supports\ our\ hypothesis\ in\ which\ I\ considered\ earlier\ that\ the\ chances\ of\ loan\ approval\ will\ be\ high\ when\ the\ loan\ amount\ is\ less.}

    \begin{Verbatim}[commandchars=\\\{\}]
{\color{incolor}In [{\color{incolor}242}]:} \PY{n}{g}\PY{p}{,} \PY{n}{p}\PY{p}{,} \PY{n}{dof}\PY{p}{,} \PY{n}{expctd} \PY{o}{=} \PY{n}{stats}\PY{o}{.}\PY{n}{chi2\PYZus{}contingency}\PY{p}{(}\PY{n}{observed}\PY{o}{=} \PY{n}{loanAmount\PYZus{}bin}\PY{p}{)}
          
          \PY{n+nb}{print}\PY{p}{(}\PY{n}{f}\PY{l+s+s1}{\PYZsq{}}\PY{l+s+s1}{The chi square statistic is }\PY{l+s+si}{\PYZob{}g\PYZcb{}}\PY{l+s+s1}{\PYZsq{}}\PY{p}{)}
          \PY{n+nb}{print}\PY{p}{(}\PY{n}{f}\PY{l+s+s1}{\PYZsq{}}\PY{l+s+s1}{The P\PYZhy{}value is }\PY{l+s+si}{\PYZob{}p\PYZcb{}}\PY{l+s+s1}{\PYZsq{}}\PY{p}{)}
          \PY{n+nb}{print}\PY{p}{(}\PY{n}{f}\PY{l+s+s1}{\PYZsq{}}\PY{l+s+s1}{The degree of freedom is }\PY{l+s+si}{\PYZob{}dof\PYZcb{}}\PY{l+s+s1}{\PYZsq{}}\PY{p}{)}
          \PY{n+nb}{print}\PY{p}{(}\PY{n}{f}\PY{l+s+s1}{\PYZsq{}}\PY{l+s+s1}{The expected count table }\PY{l+s+s1}{\PYZsq{}}\PY{p}{)}
          \PY{n}{expctd}
          
          \PY{c+c1}{\PYZsh{} calculate the chi\PYZhy{}square\PYZhy{}statistic}
          \PY{n}{expctd} \PY{o}{=} \PY{n}{pd}\PY{o}{.}\PY{n}{DataFrame}\PY{p}{(}\PY{n}{expctd}\PY{p}{)}
          \PY{n+nb}{print}\PY{p}{(}\PY{n}{expctd}\PY{p}{)}
          \PY{n}{chi\PYZus{}squared\PYZus{}stat} \PY{o}{=} \PY{p}{(}\PY{p}{(}\PY{p}{(}\PY{p}{(}\PY{n}{loanAmount\PYZus{}bin}\PY{o}{\PYZhy{}}\PY{n}{expctd}\PY{p}{)}\PY{o}{*}\PY{o}{*}\PY{l+m+mi}{2}\PY{p}{)}\PY{o}{/}\PY{n}{expctd}\PY{p}{)}\PY{o}{.}\PY{n}{sum}\PY{p}{(}\PY{p}{)}\PY{o}{.}\PY{n}{sum}\PY{p}{(}\PY{p}{)}\PY{p}{)}
          
          \PY{c+c1}{\PYZsh{} print(chi\PYZus{}squared\PYZus{}stat)}
          \PY{n}{crit} \PY{o}{=} \PY{n}{stats}\PY{o}{.}\PY{n}{chi2}\PY{o}{.}\PY{n}{ppf}\PY{p}{(}\PY{n}{q} \PY{o}{=} \PY{l+m+mf}{0.95}\PY{p}{,} \PY{c+c1}{\PYZsh{} Find the critical value for 95\PYZpc{} confidence*}
                                \PY{n}{df} \PY{o}{=} \PY{l+m+mi}{2}\PY{p}{)}   \PY{c+c1}{\PYZsh{} *}
          \PY{n+nb}{print}\PY{p}{(}\PY{l+s+s2}{\PYZdq{}}\PY{l+s+s2}{Critical value}\PY{l+s+s2}{\PYZdq{}}\PY{p}{)}
          \PY{n+nb}{print}\PY{p}{(}\PY{n}{crit}\PY{p}{)}
\end{Verbatim}


    \begin{Verbatim}[commandchars=\\\{\}]
The chi square statistic is 7.199509007986851
The P-value is 0.027330431135482496
The degree of freedom is 2
The expected count table 
           0           1
0  13.272727   34.727273
1  50.602273  132.397727
2   9.125000   23.875000
Critical value
5.99146454710798

    \end{Verbatim}

    \textbf{\emph{The above statistic and chart show that the applicant Loan
amount is associated with the loan eligibility status}}

    \subsubsection{Looking at the correlation between all the numerical
variables.}\label{looking-at-the-correlation-between-all-the-numerical-variables.}

\texttt{I\ will\ change\ the\ 3+\ in\ dependents\ variable\ to\ 3\ to\ make\ it\ a\ numerical\ variable.I\ will\ also\ convert\ the\ target\ variable’s\ categories\ into\ 0\ and\ 1\ i.e.\ map\ N\ with\ 0\ and\ Y\ with\ 1\ so\ that\ we\ can\ find\ its\ correlation\ with\ numerical\ variables.\ One\ more\ reason\ to\ do\ so\ is\ few\ models\ like\ logistic\ regression\ takes\ only\ numeric\ values\ as\ input.}

    \begin{Verbatim}[commandchars=\\\{\}]
{\color{incolor}In [{\color{incolor}121}]:} \PY{n}{train}\PY{p}{[}\PY{l+s+s1}{\PYZsq{}}\PY{l+s+s1}{Dependents}\PY{l+s+s1}{\PYZsq{}}\PY{p}{]}\PY{o}{.}\PY{n}{replace}\PY{p}{(}\PY{l+s+s1}{\PYZsq{}}\PY{l+s+s1}{3+}\PY{l+s+s1}{\PYZsq{}}\PY{p}{,} \PY{l+m+mi}{3}\PY{p}{,}\PY{n}{inplace}\PY{o}{=}\PY{k+kc}{True}\PY{p}{)}
          \PY{n}{test}\PY{p}{[}\PY{l+s+s1}{\PYZsq{}}\PY{l+s+s1}{Dependents}\PY{l+s+s1}{\PYZsq{}}\PY{p}{]}\PY{o}{.}\PY{n}{replace}\PY{p}{(}\PY{l+s+s1}{\PYZsq{}}\PY{l+s+s1}{3+}\PY{l+s+s1}{\PYZsq{}}\PY{p}{,} \PY{l+m+mi}{3}\PY{p}{,}\PY{n}{inplace}\PY{o}{=}\PY{k+kc}{True}\PY{p}{)}
          \PY{n}{train}\PY{p}{[}\PY{l+s+s1}{\PYZsq{}}\PY{l+s+s1}{Loan\PYZus{}Status}\PY{l+s+s1}{\PYZsq{}}\PY{p}{]}\PY{o}{.}\PY{n}{replace}\PY{p}{(}\PY{l+s+s1}{\PYZsq{}}\PY{l+s+s1}{N}\PY{l+s+s1}{\PYZsq{}}\PY{p}{,} \PY{l+m+mi}{0}\PY{p}{,}\PY{n}{inplace}\PY{o}{=}\PY{k+kc}{True}\PY{p}{)}
          \PY{n}{train}\PY{p}{[}\PY{l+s+s1}{\PYZsq{}}\PY{l+s+s1}{Loan\PYZus{}Status}\PY{l+s+s1}{\PYZsq{}}\PY{p}{]}\PY{o}{.}\PY{n}{replace}\PY{p}{(}\PY{l+s+s1}{\PYZsq{}}\PY{l+s+s1}{Y}\PY{l+s+s1}{\PYZsq{}}\PY{p}{,} \PY{l+m+mi}{1}\PY{p}{,}\PY{n}{inplace}\PY{o}{=}\PY{k+kc}{True}\PY{p}{)}
\end{Verbatim}


    \begin{Verbatim}[commandchars=\\\{\}]
{\color{incolor}In [{\color{incolor}123}]:} \PY{n}{matrix} \PY{o}{=} \PY{n}{train}\PY{o}{.}\PY{n}{corr}\PY{p}{(}\PY{p}{)}
          \PY{n}{matrix}
\end{Verbatim}


\begin{Verbatim}[commandchars=\\\{\}]
{\color{outcolor}Out[{\color{outcolor}123}]:}                    ApplicantIncome  CoapplicantIncome  LoanAmount  \textbackslash{}
          ApplicantIncome           1.000000          -0.116605    0.570909   
          CoapplicantIncome        -0.116605           1.000000    0.188619   
          LoanAmount                0.570909           0.188619    1.000000   
          Loan\_Amount\_Term         -0.045306          -0.059878    0.039447   
          Credit\_History           -0.014715          -0.002056   -0.008433   
          Loan\_Status              -0.004710          -0.059187   -0.037318   
          Total\_Income              0.893037           0.342781    0.624621   
          
                             Loan\_Amount\_Term  Credit\_History  Loan\_Status  Total\_Income  
          ApplicantIncome           -0.045306       -0.014715    -0.004710      0.893037  
          CoapplicantIncome         -0.059878       -0.002056    -0.059187      0.342781  
          LoanAmount                 0.039447       -0.008433    -0.037318      0.624621  
          Loan\_Amount\_Term           1.000000        0.001470    -0.021268     -0.069948  
          Credit\_History             0.001470        1.000000     0.561678     -0.015109  
          Loan\_Status               -0.021268        0.561678     1.000000     -0.031271  
          Total\_Income              -0.069948       -0.015109    -0.031271      1.000000  
\end{Verbatim}
            
    \emph{Stastically absolute value of 0.25 is used as benchmark for
correlation between two features}

\texttt{So\ get\ correlation\ between\ variables\ and\ set\ those\ less\ than\ 0.25\ to\ 0}

    \begin{Verbatim}[commandchars=\\\{\}]
{\color{incolor}In [{\color{incolor}124}]:} \PY{n}{matrix\PYZus{}corr}\PY{o}{=}\PY{n}{train}\PY{o}{.}\PY{n}{corr}\PY{p}{(}\PY{p}{)}\PY{o}{.}\PY{n}{applymap}\PY{p}{(}\PY{k}{lambda} \PY{n}{x}\PY{p}{:} \PY{n}{x} \PY{k}{if} \PY{n+nb}{abs}\PY{p}{(}\PY{n}{x}\PY{p}{)}\PY{o}{\PYZgt{}}\PY{l+m+mf}{0.25} \PY{k}{else} \PY{l+m+mi}{0}\PY{p}{)}
          \PY{n}{matrix\PYZus{}corr}
\end{Verbatim}


\begin{Verbatim}[commandchars=\\\{\}]
{\color{outcolor}Out[{\color{outcolor}124}]:}                    ApplicantIncome  CoapplicantIncome  LoanAmount  \textbackslash{}
          ApplicantIncome           1.000000           0.000000    0.570909   
          CoapplicantIncome         0.000000           1.000000    0.000000   
          LoanAmount                0.570909           0.000000    1.000000   
          Loan\_Amount\_Term          0.000000           0.000000    0.000000   
          Credit\_History            0.000000           0.000000    0.000000   
          Loan\_Status               0.000000           0.000000    0.000000   
          Total\_Income              0.893037           0.342781    0.624621   
          
                             Loan\_Amount\_Term  Credit\_History  Loan\_Status  Total\_Income  
          ApplicantIncome                 0.0        0.000000     0.000000      0.893037  
          CoapplicantIncome               0.0        0.000000     0.000000      0.342781  
          LoanAmount                      0.0        0.000000     0.000000      0.624621  
          Loan\_Amount\_Term                1.0        0.000000     0.000000      0.000000  
          Credit\_History                  0.0        1.000000     0.561678      0.000000  
          Loan\_Status                     0.0        0.561678     1.000000      0.000000  
          Total\_Income                    0.0        0.000000     0.000000      1.000000  
\end{Verbatim}
            
    \begin{Verbatim}[commandchars=\\\{\}]
{\color{incolor}In [{\color{incolor}126}]:} \PY{n}{f}\PY{p}{,} \PY{n}{ax} \PY{o}{=} \PY{n}{plt}\PY{o}{.}\PY{n}{subplots}\PY{p}{(}\PY{n}{figsize}\PY{o}{=}\PY{p}{(}\PY{l+m+mi}{9}\PY{p}{,} \PY{l+m+mi}{6}\PY{p}{)}\PY{p}{)}
          \PY{n}{sns}\PY{o}{.}\PY{n}{heatmap}\PY{p}{(}\PY{n}{matrix}\PY{p}{,} \PY{n}{vmax}\PY{o}{=}\PY{o}{.}\PY{l+m+mi}{8}\PY{p}{,} \PY{n}{square}\PY{o}{=}\PY{k+kc}{True}\PY{p}{,} \PY{n}{annot}\PY{o}{=}\PY{k+kc}{True}\PY{p}{,}\PY{n}{cmap}\PY{o}{=}\PY{l+s+s2}{\PYZdq{}}\PY{l+s+s2}{BuPu}\PY{l+s+s2}{\PYZdq{}}\PY{p}{)}\PY{p}{;}
\end{Verbatim}


    \begin{center}
    \adjustimage{max size={0.9\linewidth}{0.9\paperheight}}{output_161_0.png}
    \end{center}
    { \hspace*{\fill} \\}
    
    \begin{Verbatim}[commandchars=\\\{\}]
{\color{incolor}In [{\color{incolor}128}]:} \PY{n}{matrix\PYZus{}corr\PYZus{}sorted} \PY{o}{=} \PY{n}{matrix\PYZus{}corr}\PY{o}{.}\PY{n}{reindex}\PY{p}{(}
              \PY{n}{columns}\PY{o}{=}\PY{p}{(} \PY{n+nb}{list}\PY{p}{(}\PY{p}{[}\PY{n}{a} \PY{k}{for} \PY{n}{a} \PY{o+ow}{in} \PY{n}{matrix\PYZus{}corr}\PY{o}{.}\PY{n}{columns} \PY{k}{if} \PY{n}{a} \PY{o}{!=} \PY{l+s+s1}{\PYZsq{}}\PY{l+s+s1}{Loan\PYZus{}Status}\PY{l+s+s1}{\PYZsq{}}\PY{p}{]}\PY{p}{)} \PY{o}{+}\PY{p}{[}\PY{l+s+s1}{\PYZsq{}}\PY{l+s+s1}{Loan\PYZus{}Status}\PY{l+s+s1}{\PYZsq{}}\PY{p}{]} \PY{p}{)}\PY{p}{,}
                                \PY{n}{index}\PY{o}{=}\PY{p}{(} \PY{n+nb}{list}\PY{p}{(}\PY{p}{[}\PY{n}{a} \PY{k}{for} \PY{n}{a} \PY{o+ow}{in} \PY{n}{matrix\PYZus{}corr}\PY{o}{.}\PY{n}{columns} \PY{k}{if} \PY{n}{a} \PY{o}{!=} \PY{l+s+s1}{\PYZsq{}}\PY{l+s+s1}{Loan\PYZus{}Status}\PY{l+s+s1}{\PYZsq{}}\PY{p}{]}\PY{p}{)} \PY{o}{+}\PY{p}{[}\PY{l+s+s1}{\PYZsq{}}\PY{l+s+s1}{Loan\PYZus{}Status}\PY{l+s+s1}{\PYZsq{}}\PY{p}{]}\PY{p}{)}\PY{p}{)}
          
          \PY{n}{matrix\PYZus{}corr\PYZus{}sorted}
\end{Verbatim}


\begin{Verbatim}[commandchars=\\\{\}]
{\color{outcolor}Out[{\color{outcolor}128}]:}                    ApplicantIncome  CoapplicantIncome  LoanAmount  \textbackslash{}
          ApplicantIncome           1.000000           0.000000    0.570909   
          CoapplicantIncome         0.000000           1.000000    0.000000   
          LoanAmount                0.570909           0.000000    1.000000   
          Loan\_Amount\_Term          0.000000           0.000000    0.000000   
          Credit\_History            0.000000           0.000000    0.000000   
          Total\_Income              0.893037           0.342781    0.624621   
          Loan\_Status               0.000000           0.000000    0.000000   
          
                             Loan\_Amount\_Term  Credit\_History  Total\_Income  Loan\_Status  
          ApplicantIncome                 0.0        0.000000      0.893037     0.000000  
          CoapplicantIncome               0.0        0.000000      0.342781     0.000000  
          LoanAmount                      0.0        0.000000      0.624621     0.000000  
          Loan\_Amount\_Term                1.0        0.000000      0.000000     0.000000  
          Credit\_History                  0.0        1.000000      0.000000     0.561678  
          Total\_Income                    0.0        0.000000      1.000000     0.000000  
          Loan\_Status                     0.0        0.561678      0.000000     1.000000  
\end{Verbatim}
            
    \begin{Verbatim}[commandchars=\\\{\}]
{\color{incolor}In [{\color{incolor}131}]:} \PY{n}{plt}\PY{o}{.}\PY{n}{figure}\PY{p}{(}\PY{n}{figsize}\PY{o}{=}\PY{p}{(}\PY{l+m+mi}{8}\PY{p}{,}\PY{l+m+mi}{8}\PY{p}{)}\PY{p}{)}
          \PY{c+c1}{\PYZsh{} plt.title(\PYZsq{}Visualization chart showing strength of correlated variables\PYZsq{})}
          \PY{n}{a}\PY{o}{=}\PY{n}{sns}\PY{o}{.}\PY{n}{heatmap}\PY{p}{(}\PY{n}{matrix\PYZus{}corr\PYZus{}sorted}\PY{p}{,}\PY{n}{cmap}\PY{o}{=}\PY{l+s+s1}{\PYZsq{}}\PY{l+s+s1}{BuPu}\PY{l+s+s1}{\PYZsq{}}\PY{p}{,}\PY{n}{annot}\PY{o}{=}\PY{k+kc}{True}\PY{p}{,}
                       \PY{n}{linewidths}\PY{o}{=}\PY{o}{.}\PY{l+m+mi}{5}\PY{p}{,}\PY{n}{linecolor}\PY{o}{=}\PY{l+s+s1}{\PYZsq{}}\PY{l+s+s1}{red}\PY{l+s+s1}{\PYZsq{}}\PY{p}{)}
          \PY{n}{plt}\PY{o}{.}\PY{n}{setp}\PY{p}{(}\PY{n}{a}\PY{o}{.}\PY{n}{get\PYZus{}xticklabels}\PY{p}{(}\PY{p}{)}\PY{p}{,} \PY{n}{fontsize}\PY{o}{=}\PY{l+m+mi}{12}\PY{p}{,}\PY{n}{rotation}\PY{o}{=}\PY{l+m+mi}{75}\PY{p}{)}
          \PY{n}{plt}\PY{o}{.}\PY{n}{text} 
\end{Verbatim}


\begin{Verbatim}[commandchars=\\\{\}]
{\color{outcolor}Out[{\color{outcolor}131}]:} <function matplotlib.pyplot.text>
\end{Verbatim}
            
    \begin{center}
    \adjustimage{max size={0.9\linewidth}{0.9\paperheight}}{output_163_1.png}
    \end{center}
    { \hspace*{\fill} \\}
    
    \texttt{We\ \ use\ \ heat\ map\ to\ visualize\ the\ correlation.\ Heatmaps\ visualize\ data\ through\ variations\ in\ coloring.\ The\ variables\ with\ darker\ color\ means\ their\ correlation\ is\ more.}

    \texttt{We\ see\ that\ the\ most\ correlated\ variables\ are\ (ApplicantIncome\ -\ LoanAmount)\ ,(TotalIncome\ -\ LoanAmount)\ and\ (Credit\_History\ -\ Loan\_Status).\ LoanAmount\ is\ also\ correlated\ with\ CoapplicantIncome.}

    \textbf{\emph{We now have a better idea of what our data looks like and
which variables are important to take into account when predicting loan
eligibility status. We have narrowed it down to the following
variables:}}

\begin{itemize}
\tightlist
\item
  \texttt{Married\ Status}
\item
  \texttt{Education}
\item
  \texttt{Credit\_History}
\item
  \texttt{Property\_Area}
\item
  \texttt{Coapplicant\_Income}
\item
  \texttt{Total\_Income}
\item
  \texttt{Loan\ Amount}
\end{itemize}

\texttt{AS\ we\ now\ move\ into\ building\ machine\ learning\ models\ to\ automate\ our\ analysis,\ dealing\ with\ missing\ values,\ feeding\ the\ model\ with\ variables\ that\ meaningfully\ affect\ our\ target\ variable\ will\ improve\ our\ model\textquotesingle{}s\ prediction\ performance.}

    \begin{Verbatim}[commandchars=\\\{\}]
{\color{incolor}In [{\color{incolor}60}]:} \PY{c+c1}{\PYZsh{} check for missing values in numerical variables}
         \PY{n}{train}\PY{o}{.}\PY{n}{select\PYZus{}dtypes}\PY{p}{(}\PY{n}{include}\PY{o}{=}\PY{n}{np}\PY{o}{.}\PY{n}{number}\PY{p}{)}\PY{o}{.}\PY{n}{info}\PY{p}{(}\PY{n}{memory\PYZus{}usage}\PY{o}{=}\PY{l+s+s1}{\PYZsq{}}\PY{l+s+s1}{deep}\PY{l+s+s1}{\PYZsq{}}\PY{p}{)}
\end{Verbatim}


    \begin{Verbatim}[commandchars=\\\{\}]
<class 'pandas.core.frame.DataFrame'>
RangeIndex: 614 entries, 0 to 613
Data columns (total 5 columns):
ApplicantIncome      614 non-null int64
CoapplicantIncome    614 non-null float64
LoanAmount           592 non-null float64
Loan\_Amount\_Term     600 non-null float64
Credit\_History       564 non-null float64
dtypes: float64(4), int64(1)
memory usage: 24.1 KB

    \end{Verbatim}


    % Add a bibliography block to the postdoc
    
    
    
    \end{document}
